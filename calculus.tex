%Calculo no R^n
%Curso de Ver\~ao IME-USP 2008
% rubber: makeidx.style indexstyle.ist
\documentclass{book}
\usepackage[paperwidth=18cm,paperheight=24cm,margin=3cm]{geometry}
\usepackage{preambulo}
\usepackage[a4,center]{crop}	%sem marcas de corte
\usepackage{microtype}
\usepackage[normalem]{ulem}
\usepackage{hyperref}
\hypersetup{pdfpagelayout=TwoPageLeft, %SinglePage
  bookmarksopen=true,
  colorlinks=true,
  urlcolor=blue,
  linkcolor=black,
  pdftitle={Calculo no Rn 2008},
  pdfauthor={R\'egis S. Santos}
}
\everymath{\displaystyle}
\graphicspath{{figuras/}}

%%somatoria de i (troca o indice), ex: \Soma[j]{a_j}
\newcommand{\SomaR}[2][i]{\sum\limits_{#1} {#2}}

\newcommand{\rot}{\operatorname{rot}}	%rotacional

\fancyfoot[LO,RE]{R\'egis {\Large \smiley} 2008}
\fancyfoot[C]{\scriptsize{Curso de Ver\~ao IME USP 2008}}
%*******************************************************
\title{C\'alculo no $\R^n$
\large{Curso de Ver\~ao $2008$}\\
\small{IME - USP}}
\author{Prof. Alexandre Lymberopoulos\\
www.ime.usp.br/$\sim $lymber/verao\\
\small{R\'egis da Silva Santos}}
\date{ }    %Oculta a data
%*******************************************************
\makeindex
\begin{document}
% capa
\thispagestyle{empty}
\begin{tikzpicture}[remember picture,overlay,inner sep=0pt]
  \node[above right] at (current page.south west) {
      \includegraphics[width=210mm]{capa/capa02}
  };
\end{tikzpicture}
%*******************************************************
\pagestyle{fancy}
\frontmatter   %book
\maketitle

%*******************************************************
\chapter*{Pref\'acio}

Este material foi criado a partir de notas de aula do Curso de Ver\~ao 2008 no IME-USP.

\'E permitida a reprodu\c c\~ao total ou parcial deste material desde que indicada a autoria.

Este material foi criado para uso pessoal, portanto, adaptado para tal fim, podendo posteriormente ser adaptado para uso coletivo. E est\'a sujeito a conter erros, portanto, s\~ao aceitas sugest\~oes e cr\'iticas construtivas para melhoria do mesmo.

\vfill

\begin{flushright}
\textit{R\'egis da Silva Santos}

Mar\c co de 2008.
\end{flushright}
%*****************************************
\tableofcontents
\listoffigures
\mainmatter
\pagestyle{fancy}

\chapter{Revis\~ao de C\'alculo} \label{chap01}

\begin{defn}
Uma bola\index{Bola aberta} de centro $x_0$ e raio $r$ \'e o conjunto $B_r \left( {x_0 } \right) = \left\{ {x \in \R:\left| {x - x_0 } \right| < r} \right\}$\footnote{http://cr.yp.to/papers/calculus.pdf}
\end{defn}

\tkzfig{fig01}{Bola aberta}

Seja $S$ conjunto no dom\'inio de $f$ tal que $f\left( s \right) = \left\{ {f\left( x \right);x \in S} \right\}$.


\section{Continuidade} \label{sec01}
\begin{defn}
Seja $f$ uma fun\c c\~ao. Dizemos que $f$ \'e \textit{cont\'inua}\index{Fun\c c\~ao!cont\'inua} em $x_0$ se para toda bola centrada em $f\left( {x_0 } \right)$, $F$, existe bola centrada em $x_0$, $B$ tal que $f\left( B \right) \subset F$.
\end{defn}

\begin{figure}[!h]
  \centering
  \subfloat[Fun\c c\~ao cont\'inua]{
    \includegraphics[height=3cm]{figuras/fig02}
    \label{fig02}
  }
  \quad
  \subfloat[Fun\c c\~ao descont\'inua]{
    \includegraphics[height=3cm]{figuras/fig03}
    \label{fig03}
  }
\end{figure}

\begin{defn}
$f$ \'e cont\'inua se for cont\'inua em todos os pontos onde est\'a definida.
\end{defn}

\begin{ex}
$f(x) = 3x$
\end{ex}

\begin{sol}
Seja $x_0  \in \R;f\left( {x_0 } \right) = 3x_0$

\[
\begin{gathered}
F = B_\varepsilon  \left( {3x_0 } \right) \hfill \\
B = B_{\varepsilon /3} \left( {x_0 } \right) \Rightarrow f\left( B \right) \subset F \hfill \\
B = \left\{ {x \in \R:\left| {x - x_0 } \right| < {\raise0.7ex\hbox{$\varepsilon $} \!\mathord{\left/
{\vphantom {\varepsilon  3}}\right.\kern-\nulldelimiterspace}
\!\lower0.7ex\hbox{$3$}}} \right\} \hfill \\
\left| {x - x_0 } \right| < {\raise0.7ex\hbox{$\varepsilon $} \!\mathord{\left/
{\vphantom {\varepsilon  3}}\right.\kern-\nulldelimiterspace}
\!\lower0.7ex\hbox{$3$}} \Rightarrow  \hfill \\
3\left| {x - x_0 } \right| < \varepsilon  \hfill \\
\left| {3x - 3x_0 } \right| < \varepsilon  \hfill \\
\left| {f\left( x \right) - f\left( {x_0 } \right)} \right| < \varepsilon  \hfill \\
\Rightarrow f\left( B \right) \subset F \hfill \\
\end{gathered}
\]

$\therefore f$ \'e cont\'inua.
\end{sol}

\begin{ex}
$f\left( x \right) = \left\{ \begin{gathered}
5,{\text{ se }}x \geqslant 2 \hfill \\
3,{\text{ se }}x < 2 \hfill \\
\end{gathered}  \right.$

\end{ex}

\myfig[height=3cm]{fig04}{Fun\c c\~ao de duas senten\c c as}

\begin{sol}
contra-exemplo, $F = B_1 \left( {f\left( 2 \right)} \right)$. $\nexists B_r \left( 2 \right)$ tal que $f\left( {B_r \left( 2 \right)} \right) \subset F$.
\end{sol}

\begin{teo}
Sejam $f$ e $g$ fun\c c\~oes cont\'inuas tais que $f(x)=g(x)$, para todo $x \ne x_0$, ent\~ao $f(x_0)=g(x_0)$.
\end{teo}

\begin{dem}
Vamos provar que $\left| {f\left( {x_0 } \right) - g\left( {x_0 } \right)} \right| < \varepsilon$, para todo $\varepsilon  > 0$.

$f$ \'e cont\'inua em $x_0 \Rightarrow$ dada $F = B_{\varepsilon /2} \left( {f\left( {x_0 } \right)} \right) \Rightarrow \exists A$ bola centrada em $X_0$ tal que $f\left( A \right) \subset F$.

$g$ \'e cont\'inua em $x_0 \Rightarrow$ dada $F = B_{\varepsilon /2} \left( {g\left( {x_0 } \right)} \right) \Rightarrow \exists B$ bola centrada em $X_0$ tal que $g\left( B \right) \subset G$.

Existe $x \ne x_0$ tal que $x \in A \cap B$

$f\left( x \right) \in F$ e $\underbrace {g\left( x \right)}_{f\left( x \right)} \in F$

\begin{eqnarray*}
\left| {f\left( {x_0 } \right) - g\left( {x_0 } \right)} \right| &=& \left| {f\left( {x_0 } \right) - f\left( x \right) + \underbrace {f\left( x \right)}_{g\left( x \right)} - g\left( {x_0 } \right)} \right| \\
&\leqslant& \left| {f\left( {x_0 } \right) - f\left( x \right) + g\left( x \right) - g\left( {x_0 } \right)} \right| \\
&\leqslant& \frac{\varepsilon }
{2} + \frac{\varepsilon }
{2} \\
\left| {f\left( {x_0 } \right) - g\left( {x_0 } \right)} \right| &<& \varepsilon  \\
\Rightarrow f\left( {x_0 } \right) - g\left( {x_0 } \right) &=& 0 \\
\Rightarrow f\left( {x_0 } \right) &=& g\left( {x_0 } \right) \\
\end{eqnarray*}
\end{dem}

\begin{teo}
Sejam $f$ e $g$ fun\c c\~oes cont\'inuas em $x_0$, ent\~ao $f+g$ e $f.g$ s\~ao cont\'inuas em $x_0$.
\end{teo}

\begin{teo}
Sejam $f$ e $g$ fun\c c\~oes cont\'inuas em $g(x_0)$ e $x_0$, respectivamente, ent\~ao $f \circ g$ \'e cont\'inua em $x_0$.
\end{teo}

\begin{dem}

$f$ \'{e} cont\'{\i}nua em $g\left( {x_0 } \right)$

$ \Rightarrow $ dada $F = B_\varepsilon \left( {f\left( {f\left( {x_0 }
\right)} \right)} \right),\exists B$ bola centrada em $g\left( {x_0 }
\right)$ tal que $f\left( B \right) \subset F$.

$g$ \'{e} cont\'{\i}nua em $x_0 $

$ \Rightarrow $ para $B$ acima $\exists A$ bola centrada em $x_0 $ tal que
$g\left( A \right) \subset B$.

\[
\begin{array}{l}
g\left( A \right) \subset B \\
\Rightarrow f\left( {g\left( A \right)} \right) \subset f\left( B \right)
\subset F \\
\Rightarrow \left( {f \circ g} \right)\left( A \right) \subset F \\
\end{array}
\]

$ \Rightarrow f \circ g$ \'{e} cont\'{\i}nua em $x_0 $.
\end{dem}

\begin{ex}
$f\left( x \right) = k,k \in \R$
\end{ex}

\begin{sol}
dada $F = B_\varepsilon \left( k \right)$ escolha $B = B_r \left( {x_0 }\right),r > 0$ qualquer $ \Rightarrow f\left( B \right) \subset F$.
\end{sol}

\begin{ex}
$f\left( x \right) = x$
\end{ex}

\begin{sol}
dada $F = B_\varepsilon \left( k \right)$ escolha $B = F \Rightarrow f\left(
B \right) \subset F$.
\end{sol}

\begin{itemize}
\item \textbf{Conseq\"{u}\^{e}ncia}

\[
f\left( x \right) = a_n x^n + a_{n - 1} x^{n - 1} + ... + a_1 x + a_0 ;n \in
\mathbb{Z}^ + ;a_i \in \R
\]

s\~ao cont\'{\i}nuas.

\end{itemize}

\begin{ex}
$f\left( x \right) = 1 / x$ \'{e} cont\'{\i}nua para $x_0 \ne 0$.
\end{ex}

\begin{sol}
dada $B_\varepsilon \left( {1 / x_0 } \right)$ escolha $B = B_{\frac{1 +    \varepsilon \left| {x_0 } \right|}{\left| {x_0 } \right|}} \left( {x_0 }    \right)$.

exerc\'{\i}cio
\end{sol}

\begin{itemize}

\item \textbf{Conseq\"{u}\^{e}ncia}

\begin{enumerate}

\item se $f$ \'{e} cont\'{\i}nua e $f\left( x \right) \ne 0,\forall x$,
ent\~ao $\left( {\frac{1}{x} \circ f} \right)\left( x \right) =
\frac{1}{f\left( x \right)}$ \'{e} cont\'{\i}nua.

\item se $p\left( x \right)$ e $q\left( x \right)$ s\~ao cont\'{\i}nuas e
$q\left( x \right) \ne 0,\forall x$, ent\~ao $f\left( x \right) =
\frac{p\left( x \right)}{q\left( x \right)}$ \'{e} cont\'{\i}nua.

\end{enumerate}

\end{itemize}



\section{Derivadas} \label{sec02}

\begin{defn}
Seja $f$ uma fun\c c\~ao definida em $x_0 $. Dizemos que $f$ \'{e} \textit{deriv\'avel}\index{Fun\c c\~ao!deriv\'avel} em $x_0 $ se existir $f_1 $, fun\c c\~ao cont\'{\i}nua em $x_0 $ tal que

\[\boxed{
f\left( x \right) = f\left( {x_0 } \right) + \left( {x - x_0 } \right)f_1
\left( x \right)}
\]

\end{defn}

\begin{defn}

Seja $f$ uma fun\c c\~ao definida em $x_0 $. Ent\~ao $f$ tem derivada
$d \in \R$ em $x_0 $ se

$f\left( x \right) = f\left( {x_0 } \right) + \left( {x - x_0 } \right)f_1
\left( x \right)$ e $f_1 \left( {x_0 } \right) = d$

\end{defn}

\newpage 

\begin{ex}

$f\left( x \right) = x^2;x_0 = 2$

\end{ex}

\begin{sol}

\[
\begin{array}{l}
f\left( x \right) = f\left( {x_0 } \right) + \left( {x - x_0 } \right)f_1
\left( x \right) \\
x^2 = 2^2 + \left( {x - 2} \right)f_1 \left( x \right) \\
f_1 \left( x \right) = \frac{x^2 - 4}{x - 2} = \frac{\left( {x + 2}
\right)\left( {x - 2} \right)}{x - 2} = x + 2 \\
\therefore f'\left( 2 \right) = f_1 \left( 2 \right) = 2 + 2 = 4 \\
\end{array}
\]

E ainda, num ponto $x_0 $ qualquer, $\forall x_0 $, temos:

\[
\begin{array}{l}
f\left( x \right) = f\left( {x_0 } \right) + \left( {x - x_0 } \right)f_1
\left( x \right) \\
\Rightarrow x^2 = x_0^2 + \left( {x - x_0 } \right)f_1 \left( x \right) \\
\Rightarrow f_1 \left( x \right) = \frac{x^2 - x_0^2 }{x - x_0 } = x + x_0
\\
\Rightarrow f'\left( {x_0 } \right) = f_1 \left( {x_0 } \right) = x_0 + x_0 = 2x_0 \\
\end{array}
\]

\end{sol}

\begin{teo}
Se $f$ \'{e} deriv\'avel em $x_0 $, ent\~ao $f$ \'{e} cont\'{\i}nua em $x_0 $.
\end{teo}

\begin{dem}

$f$ \'{e} deriv\'avel em $x_0 $, ent\~ao $f\left( x \right) = f\left( {x_0} \right) + \left( {x - x_0 } \right)f_1 \left( x \right)$, como $f_1 \left(x \right)$ \'{e} cont\'{\i}nua, ent\~ao, $f$ \'{e} cont\'{\i}nua em $x_0$.

\end{dem}


\section{Regras de Deriva\c c\~ao} \label{sec03}



\begin{teo}[Derivada da Soma]

Sejam $f,g$ deriv\'aveis em $x_0 $. Ent\~ao $f + g$ \'{e} deriv\'avel
em $x_0 $.

\end{teo}

\begin{dem}

$f$ \'{e} deriv\'avel em $x_0 $, ent\~ao $f\left( x \right) = f\left( {x_0
} \right) + \left( {x - x_0 } \right)f_1 \left( x \right)$ e

$g$ \'{e} deriv\'avel em $x_0 $, ent\~ao $g\left( x \right) = g\left( {x_0
} \right) + \left( {x - x_0 } \right)g_1 \left( x \right)$

ent\~ao, $h\left( x \right) = f\left( x \right) + g\left( x \right)$

\[
\begin{array}{l}
f\left( x \right) + g\left( x \right) = f\left( {x_0 } \right) + g\left(
{x_0 } \right) + \left( {x - x_0 } \right)\left( {f_1 \left( x \right) + g_1
\left( x \right)} \right) \\
\Rightarrow h\left( x \right) = h\left( {x_0 } \right) + \left( {x - x_0 }
\right)h_1 \left( x \right) \\
\end{array}
\]

$h_1 $ \'{e} cont\'{\i}nua em $x_0 $ e

\[
\begin{array}{l}
h'\left( x \right) = \left( {f + g} \right)'\left( x \right) \\
h_1 \left( {x_0 } \right) = f_1 \left( {x_0 } \right) + g_1 \left( {x_0 }
\right) = f'\left( {x_0 } \right) + g'\left( {x_0 } \right) \\
\end{array}
\]

\end{dem}

\begin{teo}[Derivada do Produto]

Sejam $f,g$ deriv\'aveis em $x_0 $. Ent\~ao $h\left( x \right) = f\left(
x \right).g\left( x \right)$ \'{e} deriv\'avel em $x_0 $ e $\left( {fg}
\right)'\left( {x_0 } \right) = f'\left( {x_0 } \right)g\left( {x_0 }
\right) + f\left( {x_0 } \right)g'\left( {x_0 } \right)$.

\end{teo}

\begin{dem}

$f$ \'{e} deriv\'avel em $x_0 $, ent\~ao $f\left( x \right) = f\left( {x_0
} \right) + \left( {x - x_0 } \right)f_1 \left( x \right)$ e

$g$ \'{e} deriv\'avel em $x_0 $, ent\~ao $g\left( x \right) = g\left( {x_0
} \right) + \left( {x - x_0 } \right)g_1 \left( x \right)$

ent\~ao, $h\left( x \right) = f\left( x \right).g\left( x \right)$

\[
\begin{array}{l}
f\left( x \right)g\left( x \right) = \left( {f\left( {x_0 } \right) +
\left( {x - x_0 } \right)f_1 \left( x \right)} \right)\left( {g\left( {x_0 }
\right) + \left( {x - x_0 } \right)g_1 \left( x \right)} \right) \\
= f\left( {x_0 } \right)g\left( {x_0 } \right) + \left( {x - x_0 }
\right)\left[ {f\left( {x_0 } \right)g_1 \left( x \right) + f_1 \left( x
\right)g\left( {x_0 } \right) + f_1 \left( x \right)g_1 \left( x
\right)\left( {x - x_0 } \right)} \right] \\
= h\left( {x_0 } \right) + \left( {x - x_0 } \right)h_1 \left( x \right) \\
\Rightarrow \left( {fg} \right)'\left( {x_0 } \right) = h_1 \left( {x_0 }
\right) = f\left( {x_0 } \right)g_1 \left( {x_0 } \right) + f_1 \left( {x_0
} \right)g\left( {x_0 } \right) \\
= f\left( {x_0 } \right)g'\left( {x_0 } \right) + f'\left( {x_0 }
\right)g\left( {x_0 } \right) \\
\end{array}
\]

\end{dem}

\begin{teo}[Regra da Cadeia]

Se $f$ \'{e} deriv\'avel em $g\left( {x_0 } \right)$ e $g$ \'{e}
deriv\'avel em $x_0 $, ent\~ao $f \circ g$ \'{e} deriv\'avel em $x_0 $
e $\left( {f \circ g} \right)'\left( {x_0 } \right) = f'\left( {g\left( {x_0
} \right)} \right).g'\left( {x_0 } \right)$.

\end{teo}

\begin{dem}

Fa\c camos $y = g\left( x \right)$ e $y_0 = g\left( {x_0 } \right)$

$f$ \'{e} deriv\'avel em $y_0 $, ent\~ao $f\left( y \right) = f\left( {y_0
} \right) + \left( {y - y_0 } \right)f_1 \left( y \right)$ e

$g$ \'{e} deriv\'avel em $x_0 $, ent\~ao $g\left( x \right) = g\left( {x_0
} \right) + \left( {x - x_0 } \right)g_1 \left( x \right)$

\begin{eqnarray*}
f\left( y \right) = f\left( {y_0 } \right) + \left( {g\left( {x_0 } \right)
+ \left( {x - x_0 } \right)g_1 \left( x \right) - \underbrace {y_0
}_{g\left( {x_0 } \right)}} \right).f_1 \left( y \right) \\
f\left( {g\left( x \right)} \right) = f\left( {g\left( {x_0 } \right)}
\right) + \left( {x - x_0 } \right)\underbrace {g_1 \left( x \right)f_1
\left( {g\left( x \right)} \right)}
\end{eqnarray*}

$f \circ g$ \'{e} deriv\'avel em $x_0 $ e

\[
\begin{array}{l}
\left( {f \circ g} \right)'\left( {x_0 } \right) = f_1 \left( {g\left( {x_0
} \right)} \right).g_1 \left( {x_0 } \right) \\
= f'\left( {g\left( {x_0 } \right)} \right).g'\left( {x_0 } \right) \\
\end{array}
\]

\end{dem}

\begin{ex}

\[
\begin{array}{l}
f\left( x \right) = k \\
f\left( x \right) = f\left( {x_0 } \right) + \left( {x - x_0 } \right)f_1
\left( x \right) \\
k = k + \left( {x - x_0 } \right)f_1 \left( x \right) \\
\left( {x - x_0 } \right)f_1 \left( x \right) = 0 \Rightarrow f_1 \left( x
\right) = 0 \\
\Rightarrow f'\left( {x_0 } \right) = f_1 \left( {x_0 } \right) = 0 \\
\end{array}
\]

\end{ex}

\begin{ex}

\[
\begin{array}{l}
f\left( x \right) = x^n \\
f\left( x \right) = f\left( {x_0 } \right) + \left( {x - x_0 } \right)f_1
\left( x \right) \\
x^n = x_0^n + \left( {x - x_0 } \right)f_1 \left( x \right) \\
f_1 \left( x \right) = \frac{x^n - x_0^n }{x - x_0 } = x^{n - 2}x_0^1 +
x^{n - 3}x_0^2 + ... + x^1x_0^{n - 2} + x_0^{n - 1} \\
\Rightarrow f'\left( {x_0 } \right) = f_1 \left( {x_0 } \right) = nx_0^{n -
1} \\
\end{array}
\]

\end{ex}

\begin{ex}

\[
\begin{array}{l}
f\left( x \right) = \frac{1}{x} \\
f\left( x \right) = f\left( {x_0 } \right) + \left( {x - x_0 } \right)f_1
\left( x \right) \\
\frac{1}{x} = \frac{1}{x_0 } + \left( {x - x_0 } \right)f_1 \left( x
\right) \\
f_1 \left( x \right) = - \frac{1}{x.x_0 } \\
f'\left( {x_0 } \right) = f_1 \left( {x_0 } \right) = - \frac{1}{x_0^2 } \\
\end{array}
\]

\end{ex}

\begin{teo}[Derivada do Quociente]

Sejam $f$ e $g$ deriv\'aveis em $x_0 $ com $g\left( {x_0 } \right) \ne 0$,
ent\~ao $f / g$ \'{e} deriv\'avel em $x_0 $ e

\[
\left( {\frac{f}{g}} \right)^{'}\left( {x_0 } \right) = \frac{f'\left( {x_0 }
\right)g\left( {x_0 } \right) - f\left( {x_0 } \right)g'\left( {x_0 }
\right)}{g\left( {x_0 } \right)^2}
\]

\end{teo}

\begin{dem}

\[
\begin{array}{l}
\left( {\frac{f}{g}} \right)^{'}\left( {x_0 } \right) = \left(
{f.\frac{1}{g}} \right)^{'}\left( {x_0 } \right) = f'\left( {x_0 }
\right).\frac{1}{g\left( {x_0 } \right)} + f\left( {x_0 } \right). -
\frac{1}{g\left( {x_0 } \right)^2}.g'\left( {x_0 } \right) \\
 = \displaystyle\frac{f'\left( {x_0 } \right)g\left( {x_0 } \right) - f\left( {x_0 }
\right)g'\left( {x_0 } \right)}{g\left( {x_0 } \right)^2} \\
\end{array}
\]

\end{dem}


\section{A Completude de $\R$ e suas Conseq\"{u}\^{e}ncias} \label{sec04}


\begin{defn}
Seja $S$ um conjunto de reais. Um n\'{u}mero real $c$ \'{e} quota superior\index{Quota superior}
de $S$ se $x \leqslant c,\forall x \in S$.

\end{defn}

\begin{ex}

Seja $c \geqslant \pi $, $c$ \'{e} quota superior para $S = \left\{
{3,3.1,3.14,3.141,...} \right\}$. A menor quota superior \'{e} $\pi $.

\end{ex}

Os n\'{u}meros reais s\~ao completos: todo $S \subset \R,S \ne
\emptyset $, com quota superior admite menor quota superior. Essa menor
quota superior \'{e} o \textbf{supremo}\index{Supremo} de $S$, $\sup S$.

\begin{ex}

$\left\{ {x \in \mathbb{Q}:x < \sqrt 2 } \right\}$tem quota superior, mas
n\~ao tem supremo em $\mathbb{Q}$.

\end{ex}

\begin{teo}[valor intermedi\'ario]
\label{tvi}
\index{Teorema!do valor intermedi\'ario}

Seja $f$ uma fun\c c\~ao cont\'{\i}nua com valores reais. Sejam $b,c$ e
$y$ reais tais que $b \leqslant c$ e $f$ definida em $\left[ {b,c} \right]$
com $f\left( b \right) \leqslant y \leqslant f\left( c \right)$.

Ent\~ao, existe $x \in \left[ {b,c} \right]$ tal que $f\left( x \right) =
y$.

\end{teo}

% \documentclass[10pt]{article}
\usepackage[utf8]{inputenc}
\usepackage{pstricks,pstricks-add,pst-math,pst-xkey,pst-eps}
\pagestyle{empty}
\begin{document}
\begin{TeXtoEPS}
%\pstVerb{/func {180 mul 3.1416 div 4 sub sin neg 4 add} def}
%sin(x*180/pi-4)+4

% \begin{figure}[!htbp]
% \begin{center}

\psset{unit=0.75cm}

\begin{pspicture}(-2,-1)(8,6.5)
    %Fun��o
    \psplot[linewidth=2pt]{2}{6}{x 180 mul 3.1416 div 4 sub sin neg 4 add} %sin(x*180/pi-4)+4

    %Eixo Ox e Oy
    %requer o pacote pstricks-add
    \psaxes[ticks=none]{->}(0,0)(-1,-1)(8,6)
    \uput[-90](8,0){$x$}
    \uput[180](0,6){$y$}

    %R�tulos e intervalos do eixo Ox
    \psline{(-)}(3.5,0)(4.5,0)
    \psdot[dotscale=1,dotstyle=|](4,0)
    \uput[-90](3.5,0){$b$}
    \uput[-90](4,0){$x$}
    \uput[-90](4.5,0){$c$}

    %R�tulos e intervalos do eixo Oy
    \psdot[dotscale=1,dotstyle=+](!0 3.5 180 mul 3.1416 div 4 sub sin neg 4 add)
    \psdot[dotscale=1,dotstyle=+](!0 4 180 mul 3.1416 div 4 sub sin neg 4 add)
    \psdot[dotscale=1,dotstyle=+](!0 4.5 180 mul 3.1416 div 4 sub sin neg 4 add)
    \uput[-135](!0 3.5 180 mul 3.1416 div 4 sub sin neg 4 add){$f(b)$}
    \uput[180](!0 4 180 mul 3.1416 div 4 sub sin neg 4 add){$f(x)=y$}
    \uput[135](!0 4.5 180 mul 3.1416 div 4 sub sin neg 4 add){$f(c)$}

    %Retas verticais
    \psline[linestyle=dashed](3.5,0)(!3.5 3.5 180 mul 3.1416 div 4 sub sin neg 4 add)
    \psline[linestyle=dashed](4,0)(!4 4 180 mul 3.1416 div 4 sub sin neg 4 add)
    \psline[linestyle=dashed](4.5,0)(!4.5 4.5 180 mul 3.1416 div 4 sub sin neg 4 add)

    %Retas horizontais
    \psline[linestyle=dashed](!0 3.5 180 mul 3.1416 div 4 sub sin neg 4 add)(!3.5 3.5 180 mul 3.1416 div 4 sub sin neg 4 add)
    \psline[linestyle=dashed](!0 4 180 mul 3.1416 div 4 sub sin neg 4 add)(!4 4 180 mul 3.1416 div 4 sub sin neg 4 add)
    \psline[linestyle=dashed](!0 4.5 180 mul 3.1416 div 4 sub sin neg 4 add)(!4.5 4.5 180 mul 3.1416 div 4 sub sin neg 4 add)

\end{pspicture}
% \caption{TVI} \label{figura05}
% \end{center}
% \end{figure} 
\end{TeXtoEPS}
\end{document} 

\myfig{fig05}{TVI}

\begin{dem}


Seja $S = \left\{ {x \in \left[ {b,c} \right]:f\left( x \right) \leqslant y}
\right\}$

$S \ne \emptyset ,$pois $b \in S\left( {f\left( b \right) \leqslant y}
\right)$

$S$tem $c$ como quota superior.

Portanto, $S$ tem supremo: $u = \sup S$.

Vamos provar que $f\left( u \right) = y$.

Suponha $f\left( u \right) > y$, existe $D$ bola aberta centrada em $u$ tal
que $f\left( x \right) > y,\forall x \in D$. Seja $t \in D,t < u\left(
{f\left( t \right) > y} \right)$, logo $t$ \'{e} quota superior para $S$ e
$t < u$.

Contradi\c c\~ao, achamos quota superior menor que o supremo, logo,
$f\left( u \right) \leqslant y$.

Suponha $f\left( u \right) < y$, isto implica que $u \ne c$ e ent\~ao $u <
c$.

Por continuidade de $f$ existe $D$ bola aberta centrada em $u$ tal que \\
$f\left( x \right) < y,\forall x \in D$.

Seja $x \in D$ tal que $u < x < c$, mas $x \ne S \Rightarrow f\left( x
\right) > y$.

Contradi\c c\~ao, logo, $f\left( u \right) \geqslant y$.

Portanto, $f\left( u \right) = y$.

\end{dem}

\begin{teo}

Seja $f$ cont\'{\i}nua em $\left[ {b,c} \right]$ tal que $f\left( b
\right).f\left( c \right) < 0$. Ent\~ao existe $x \in \left[ {b,c}
\right]$ tal que $f\left( x \right) = 0$.

\end{teo}

\begin{dem}

Fa\c ca $y=0$ no teorema \ref{tvi}.

\end{dem}

\begin{teo} \label{t1}

Seja $f$ fun\c c\~ao cont\'{\i}nua e $b,c$ reais tais que $b \leqslant
c$. Ent\~ao $f\left( {\left[ {b,c} \right]} \right)$ tem quota superior.

\end{teo}

\begin{dem}

Seja $S = \left\{ {x \in \left[ {b,c} \right]:f\left( {\left[ {b,x} \right]}
\right)} \right\}$ \'{e} limitada.

$S = \emptyset ;b \in S;f\left( {\left[ {b,b} \right]} \right) = f\left( b
\right)$(limitada).

$S$tem quota superior; $x \leqslant c,\forall x \in S$. Ent\~ao, $u = \sup
S$.

Vamos provar que $u = c$.

Pela continuidade de $f$ existe $D$, bola centrada em $u$ com $f\left( D
\right) \subset B_1 \left( {f\left( u \right)} \right)$. Seja $t \in D,t <
u \Rightarrow \exists x,t < x < u$ tal que $f\left( {\left[ {b,x} \right]}
\right)$ \'{e} limitada e $f\left( {\left[ {x,u} \right]} \right) \subset
B_1 \left( {f\left( u \right)} \right)$ \'{e} limitada.

Logo, $f\left( {\left[ {b,u} \right]} \right)$ \'{e} limitada.

Suponha $u < c \Rightarrow \exists v \in D$ tal que $u < v < c$, logo,
$f\left( {\left[ {u,v} \right]} \right)$ \'{e} limitada. Ent\~ao, $f\left(
{\left[ {b,v} \right]} \right)$ \'{e} limitada. Contradi\c c\~ao.

Logo, $u \geqslant c \Rightarrow u = c$.

\end{dem}

\begin{teo}[m\'aximo - Weierstrass]
\label{t2}
\index{Teorema!de Weierstrass}

Seja $f$ fun\c c\~ao cont\'{\i}nua e $b,c$ reais com $b \leqslant c$.
Ent\~ao existe $z \in \left[ {b,c} \right]$ tal que $f\left( x \right)
\leqslant f\left( z \right),\forall x \in \left[ {b,c} \right]$.

\end{teo}

\begin{dem}

$f\left( {\left[ {b,c} \right]} \right)$ tem quota superior, teorema \ref{t1},
logo existe $M = \sup \left\{ {f\left( {\left[ {b,c} \right]} \right)}
\right\}$.

Seja $S = \left\{ {x \in \left[ {b,c} \right]:\sup \left\{ {f\left( {\left[
{x,c} \right]} \right)} \right\} = M} \right\}$

\[
S \ne \emptyset :b \in S.
\]

$S$tem quota superior, ent\~ao, $c$ \'{e} quota. $u = \sup S$.

Suponha $f\left( u \right) < M$.

Continuidade de $f$ d\'a bola aberta $D$ centrada em $u$ com $f\left( D
\right) \subset B_{\frac{M - f\left( u \right)}{2}} \left( {f\left( u
\right)} \right)$.

Logo, $\sup \left\{ {f\left( D \right)} \right\} < M$.

Seja $t \in D,t < u$ e $x \in D$ com $t < x \leqslant u$.

Ent\~ao $\sup \left\{ {f\left( {\left[ {x,c} \right]} \right)} \right\} =
M$, pois $x \in S,t < x \leqslant u$.

Mas $\sup \left\{ {f\left( {\left[ {x,u} \right]} \right)} \right\} < M$ e
ent\~ao, $u < c$.

Seja $v \in D$ com $u < v < c$ e $\sup \left\{ {f\left( {\left[ {x,v}
\right]} \right)} \right\} < M$, (pois $\left[ {x,v} \right] \subset D)$,
ent\~ao, $\sup \left\{ {f\left( {\left[ {v,c} \right]} \right)} \right\} =
M$.

Contradi\c c\~ao, logo, $f\left( u \right) \geqslant M \Rightarrow
f\left( u \right) = M$.

\end{dem}

\begin{teo}

Seja $f$ fun\c c\~ao cont\'{\i}nua e $b,c$ reais com $b \leqslant c$.

Ent\~ao existe $u \in \left[ {b,c} \right]$ tal que $f\left( u \right)
\leqslant f\left( x \right),\forall x \in \left[ {b,c} \right]$.

\end{teo}

\begin{dem}

$ - f$\'{e} continua e, pelo teorema \ref{t2}, tem m\'aximo em $\left[ {b,c}
\right]$, isto \'{e}, $\exists u \in \left[ {b,c} \right]$ tal que

$\begin{array}{l}
- f\left( x \right) \leqslant - f\left( u \right),\forall x \in \left[
{b,c} \right] \\
\Rightarrow f\left( u \right) \leqslant f\left( x \right),\forall x \in
\left[ {b,c} \right] \\
\end{array}$

\end{dem}

\begin{teo}[Teorema de Fermat]
\label{t3}
\index{Teorema!de Fermat}
Seja $f$ uma fun\c c\~ao deriv\'avel em $x_0$. Suponha $f\left( {x_0 } \right) \geqslant f\left( x \right)$, para todo $x$ numa bola aberta $B$ centrada em $x_0$. Ent\~ao $f'\left( {x_0 } \right) = 0$.
\end{teo}

\begin{dem}
Seja $f$ deriv\'avel em $x_0$, ent\~ao

\[
    f\left( x \right) = f\left( {x_0 } \right) + \left( {x - x_0 } \right)f_1 \left( x \right)
\]

$f_1$ \'e cont\'inua em $x_0$.

    Suponha $\underbrace {f'\left( {x_0 } \right)}_{f_1 \left( {x_0 } \right)} > 0$.

    Existe $D$ bola centrada em $x_0$ tal que $f_1 \left( x \right) > 0,\forall x \in D$.

Seja $x \in B \cap D,x > x_0$

\[
\begin{gathered}
      f\left( {x_0 } \right) \geqslant f\left( x \right) = f\left( {x_0 } \right) + \underbrace {\left( {x - x_0 } \right)}_{ > 0}\underbrace {f_1 \left( x \right)}_{ > 0} > f\left( {x_0 } \right) \hfill \\
       \Rightarrow f\left( {x_0 } \right) \geqslant f\left( {x_0 } \right) \hfill \\
\end{gathered}
\]

Contradi\c c\~ao. Logo, $f'\left( {x_0 } \right) \leqslant 0$.

Analogamente, supondo $f'\left( {x_0 } \right) < 0$.

E $x \in B \cap D$, com $x < x_0$, obtemos a mesma contradi\c c\~ao.

$f'\left( {x_0 } \right) \geqslant 0$.

Portanto, $f'\left( {x_0 } \right) = 0$.
\end{dem}

\begin{teo}
    Seja $f$ uma fun\c c\~ao deriv\'avel com $f\left( {x_0 } \right) \leqslant f\left( x \right)$, para todo $x \in B$, bola centrada em $x_0$. Ent\~ao, $f'\left( {x_0 } \right) = 0$.
\end{teo}

\begin{dem}
Aplique o Teorema \ref{t3} para $-f$.

\[
\begin{gathered}
  f\left( {x_0 } \right) \leqslant f\left( x \right) \Rightarrow  - f\left( {x_0 } \right) \geqslant  - f\left( x \right) \hfill \\
   \Rightarrow  - f'\left( {x_0 } \right) = 0 \Rightarrow f'\left( {x_0 } \right) = 0 \hfill \\
\end{gathered}
\]

\end{dem}

\begin{teo}[Teorema de Rolle]
\label{tr}
\index{Teorema!de Rolle}
\begin{sloppypar}
Seja $f$ uma fun\c c\~ao deriv\'avel. Sejam ${b < c \in \R}$. Se $f\left( b \right) = f\left( c \right)$, ent\~ao existe $x \in \left] {b,c} \right[$ tal que $f'\left( {x_0 } \right) = 0$.
\end{sloppypar}
\end{teo}

\begin{dem}
$f$ \'e deriv\'avel, portanto, $f$ \'e cont\'inua.

Portanto, assume m\'aximo em algum $x \in \left[ {b,c} \right]$.

    Se $f\left( x \right) > f\left( b \right) = f\left( c \right)$, ent\~ao, $x \ne b$ e $x \ne c$, logo, existe $B$ centrada em $x$ tal que

\[
    f\left( x \right) \geqslant f\left( t \right);\forall t \in B \Rightarrow f'\left( x \right) = 0
\]

    Analogamente $f$ assume m\'inimo em algum $u \in \left[ {b,c} \right]$ se $f\left( u \right) < f\left( b \right) = f\left( c \right)$.

Ent\~ao, $u \ne b$ e $u \ne c \Rightarrow f'\left( u \right) = 0$.

    Por fim, se $f\left( x \right) \leqslant f\left( b \right) = f\left( c \right)$ e $f\left( u \right) \geqslant f\left( b \right) = f\left( c \right)$, ent\~ao, $f\left( b \right) = f\left( c \right)$ \'e m\'aximo e m\'inimo simultaneamente, logo, $f$ \'e constante

\[
\Rightarrow f'\left( x \right) = 0,\forall x \in \left] {b,c} \right[
\]

\end{dem}

\newpage 

\begin{teo}[Teorema do Valor M\'edio]
\label{tvm}
\index{Teorema!do valor m\'edio}
Seja $f$ deriv\'avel e $b,c$ reais com $b \leqslant c$. Ent\~ao, existe $d \in \left] {b,c} \right[$ tal que $\displaystyle f'\left( d \right) = \frac{{f\left( c \right) - f\left( b \right)}}{{c - b}}$.
\end{teo}

\begin{dem}
Seja

\[
\begin{gathered}
  g\left( x \right) = \left( {c - b} \right)f\left( x \right) - \left( {x - b} \right)\left( {f\left( c \right) - f\left( b \right)} \right) \hfill \\
g\left( b \right) = \left( {c - b} \right)f\left( b \right) \hfill \\
  g\left( c \right) = \left( {c - b} \right)f\left( c \right) - \left( {c - b} \right)\left( {f\left( c \right) - f\left( b \right)} \right) \hfill \\
g\left( c \right) = \left( {c - b} \right)f\left( b \right) \hfill \\
\Rightarrow g\left( b \right) = g\left( c \right) \hfill \\
\end{gathered}
\]

    $g$ \'e deriv\'avel $\Rightarrow \exists d \in \left] {b,c} \right[$ tal que $g'\left( d \right) = 0$.

\[
\begin{gathered}
  g'\left( x \right) = \left( {c - b} \right)f'\left( x \right) - \left( {f\left( c \right) - f\left( b \right)} \right) \hfill \\
   \Rightarrow 0 = g'\left( d \right) = \left( {c - b} \right)f'\left( d \right) - \left( {f\left( c \right) - f\left( b \right)} \right) \hfill \\
   \Rightarrow \displaystyle f'\left( d \right) = \frac{{f\left( c \right) - f\left( b \right)}}
{{c - b}} \hfill \\
\end{gathered}
\]

\end{dem}


\section{Integral de Riemann} \label{sec05}
\index{Integral!de Riemann}

\begin{defn}
    Seja $\left[ {a,b} \right] \subset \R$ um intervalo. Uma \textit{parti\c c\~ao} de $\left[ {a,b} \right]$ \'e a escolha de $\left\{ {t_i } \right\}_{i = 0}^n  \in \left[ {a,b} \right]$ tais que $t_0  = a < t_1  < t_2  < ... < t_n  = b$ e $P = \left( {t_0 ,t_1 ,...,t_n } \right)$.

    A norma de $P$ \'e $\left| P \right| = \max \left\{ {t_i  - t_{i - 1} ;i = 1,...,n} \right\}$.
\end{defn}

\begin{defn}
Seja $f:\left[ {a,b} \right] \to \R$ fun\c c\~ao limitada.

A \textit{Soma de Riemann}\index{Soma de Riemann} de $f$ relativa \`a parti\c c\~ao $P = \left( {t_0 ,t_1 ,...,t_n } \right)$ de $\left[ {a,b} \right]$ \'e a escolha $\left\{ {c_i } \right\}_{i = 1}^n ,c_i  \in \left[ {t_{i - 1} ,t_i } \right]$ \'e

\[\boxed{
    S\left( {f,P,\left\{ {c_i } \right\}} \right) = \sum\limits_{i = 1}^n {f\left( {c_i } \right)\left( {t_i  - t_{i - 1} } \right)}}
\]

\end{defn}

\begin{defn}
Dizemos que $f$ \'e \textit{Riemann Integr\'avel}\index{Fun\c c\~ao!Riemann integr\'avel} se existir

\[
    \mathop {\lim }\limits_{\left| P \right| \to 0} S\left( {f,P,\left\{ {c_i } \right\}} \right)
\]

    para toda parti\c c\~ao $P$ de $\left[ {a,b} \right]$ e escolha dos $c_i  \in \left[ {t_{i - 1} ,t_i } \right]$.

Neste caso escrevemos

\[
    \int\limits_a^b {f\left( x \right)dx}  = \mathop {\lim }\limits_{\left| P \right| \to 0} \sum\limits_{i = 1}^n {f\left( {c_i } \right)\left( {t_i  - t_{i - 1} } \right)}
\]

\end{defn}

\textbf{Exerc\'icio}

\begin{exerc}
\begin{sloppypar}
 Calcule $\int\limits_0^1 {x^2 dx}$, usando $P = \left( {t_i } \right)$, onde $t_i  = i/n,i = 0,...,n$ e ${c_i  = t_i \left( {\sum\limits_{i = 1}^n {f\left( {c_i } \right)\left( {t_i  - t_{i - 1} } \right)} } \right)}$.
\end{sloppypar}
\end{exerc}

\begin{defn}
Uma \textit{primitiva} de $f$ \'e uma fun\c c\~ao $F$ tal que $F' = f$.
\end{defn}

\begin{teo}[Teorema Fundamental do C\'alculo] \label{tfc}
\index{Teorema!fundamental do c\'alculo}
Seja $f:\left[ {a,b} \right] \to \R$ fun\c c\~ao integr\'avel e $F$ sua primitiva. Ent\~ao

\[\boxed{
    \int\limits_a^b {f\left( x \right)dx}  = F\left( b \right) - F\left( a \right)}
\]

\end{teo}

\begin{dem}
    Seja $P = \left( {t_0 ,t_1 ,...,t_n } \right)$ parti\c c\~ao de $\left[ {a,b} \right]$.

\[
\begin{gathered}
  \scriptstyle {F\left( b \right) - F\left( a \right) = F\left( {t_n } \right) - F\left( {t_{n - 1} } \right) + F\left( {t_{n - 1} } \right) - F\left( {t_{n - 2} } \right) + F\left( {t_{n - 2} } \right) + ... + F\left( {t_1 } \right) - F\left( {t_0 } \right)} \hfill \\
   = \sum\limits_{i = 1}^n {F\left( {t_i } \right) - F\left( {t_{i - 1} } \right)} \left( 1 \right) \hfill \\
\end{gathered}
\]

    De $\displaystyle \frac{{F\left( {t_i } \right) - F\left( {t_{i - 1} } \right)}}
{{t_i  - t_{i - 1} }} = F'\left( {t_i } \right)$ (Teorema \ref{tvm} do Valor M\'edio), temos:

\begin{eqnarray*}
      \left( 1 \right) &=& \sum\limits_{i = 1}^n {F'\left( {c_i } \right)\left( {t_i  - t_{i - 1} } \right)}  \hfill \\
       &=& \sum\limits_{i = 1}^n {f\left( {c_i } \right)\left( {t_i  - t_{i - 1} } \right)}  \hfill \\
\end{eqnarray*}

O lado esquerdo independe de $P$ e $\left\{ {c_i } \right\}$

\[
     \Rightarrow \int\limits_a^b {f\left( x \right)dx}  = F\left( b \right) - F\left( a \right)
\]

\end{dem}

\begin{ex}

Seja $f\left( x \right) = \left\{ {\begin{array}{l}
1,\mbox{ se }x \in \mathbb{Q} \cap \left[ {0,1} \right] \\
0,\mbox{ se }x \in \left( {\R\backslash \mathbb{Q}} \right) \cap
\left[ {0,1} \right] \\
\end{array}} \right.$

\end{ex}



\begin{sol}

\[
P = \left( {t_i } \right),t_i = i / n
\]


\begin{enumerate} [i)]

\item $c_i = $ algum irracional entre $t_{i - 1} $ e $t_i $

\[
\sum\limits_{i = 1}^n {\underbrace {f\left( {c_i } \right)}_0\left( {t_i -
t_{i - 1} } \right)} = 0
\]


\item $c_i = $ algum racional entre $t_{i - 1} $ e $t_i $

\[
\sum\limits_{i = 1}^n {\underbrace {f\left( {c_i } \right)}_1\left( {t_i -
t_{i - 1} } \right)} = 1 - 0 = 1
\]


\end{enumerate}

$\therefore f$n\~ao \'{e} integr\'avel.

\end{sol}


\section{Fun\c c\~{o}es dadas por Integrais} \label{sec06}

% figura 09

\myfig{fig09}{Integral}

Seja $f$ integr\'avel em $\left[ {a,b} \right]$

\[
F\left( x \right) = \int\limits_a^b {f\left( t \right)dt}
\]


\begin{teo}

Seja $f$ integr\'avel em $\left[ {a,b} \right]$ e cont\'{\i}nua.

Ent\~ao existe $c \in \left[ {a,b} \right]$ tal que $\int\limits_a^b
{f\left( x \right)dx} = f\left( c \right)\left( {b - a} \right)$

\end{teo}

% Figura 10

\myfig{fig10}{}

\begin{dem}

Seja $f$ cont\'{\i}nua em $\left[ {a,b} \right]$.

Ent\~ao, $f$ tem m\'aximo $\left( M \right)$ e m\'{\i}nimo $\left( m
\right)$ em $\left[ {a,b} \right]$.

\[
\begin{array}{l}
\Rightarrow m \leqslant f\left( x \right) \leqslant M \\
\Rightarrow \int\limits_a^b {mdx} \leqslant \int\limits_a^b {f\left( x
\right)dx} \leqslant \int\limits_a^b {Mdx} \\
\Rightarrow m\left( {b - a} \right) \leqslant \int\limits_a^b {f\left( x
\right)dx} \leqslant M\left( {b - a} \right) \\
\Rightarrow m \leqslant \frac{\int\limits_a^b {f\left( x \right)dx} }{b -
a} \leqslant M \\
\end{array}
\]


Pelo, teorema \ref{tvi} do Valor Intermedi\'ario, temos

$\exists c \in \left[ {a,b} \right]$tal que $f\left( c \right) =
\frac{\int\limits_a^b {f\left( x \right)dx} }{b - a}$

\[
\Rightarrow \int\limits_a^b {f\left( x \right)dx} = f\left( c \right)\left(
{b - a} \right)
\]


\end{dem}


\begin{teo}

Seja $f$ integr\'avel em $\left[ {a,b} \right]$, ent\~ao, $F\left( x
\right) = \int\limits_a^x {f\left( t \right)dt} $ \'{e} cont\'{\i}nua em
$\left[ {a,b} \right]$.

\end{teo}



\begin{dem}

Exerc\'{\i}cio

\end{dem}



\begin{teo}

Seja $f$ cont\'{\i}nua em $\left[ {a,b} \right]$, ent\~ao, $F\left( x
\right) = \int\limits_a^x {f\left( t \right)dt} $ \'{e} deriv\'avel em
$\left[ {a,b} \right]$ e $F'\left( x \right) = f\left( x \right)$.

\end{teo}

\section{Exerc\'icios Resolvidos}

\begin{exerc}
Sejam $f$ uma fun\c c\~ao deriv\'avel e $a < b$ n\'umeros reais tais que $f(a) = f(b) = 0$ e $f'(a)f'(b) > 0$. Prove que existe $c \in (a,b)$ tal que $f(c) = 0$.
\end{exerc}

\begin{sol}
$f'(a)$ e $f'(b)$ s\~ao de mesmo sinal, vamos supor que $f'(a)$ e $f'(b) > 0$.

\tkzfigonly{fig28}

Neste caso, a inclina\c c\~ao das retas tangentes em $a$ e $b$ s\~ao crescentes. Basta mostrar que existem $x,y \in (a,b)$ tais que $f(x) > 0$ e $f(y) < 0$. A tese sai do Teorema do Valor Intermedi\'ario (\ref{tvi}).

\begin{sloppypar}
$f$ \'e deriv\'avel em $a$ e $f'(a) > 0$, ent\~ao existe $f_1$ cont\'inua em $a$ tal que ${f(x) = f(a) + (x - a)f_1(x)}$ e $f_1(a) = f'(a)$.
\end{sloppypar}

Como $f(a) = 0$, ent\~ao $f(x) = (x - a)f_1(x)$. Como $f'(a) > 0$, fixe $\varepsilon = \frac{f'(a)}{2} > 0$.

Como $f_1$ \'e cont\'inua em $a$, existe $\delta_1 > 0$ tal que se $|x - a| < \delta_1$, ent\~ao $\left| {f_1 (x) - \underbrace {f_1 (a)}_{f'(a)}} \right| < \varepsilon  = \frac{{f'(a)}}{2}$.

\[
\begin{gathered}
  |f_1(x) - f'(a)| < \frac{f'(a)}{2} \sse \frac{-f'(a)}{2} < f_1(x) - f'(a) < f'(a) \hfill \\
  \sse 0 < \frac{f'(a)}{2} < f_1(x) < \frac{3f'(a)}{2} \hfill \\ 
\end{gathered} 
\]

Tome $x \in (a,a + \delta_1)$. Como $f(x) = \underbrace{(x - a)}_{>0} \underbrace{f_1(x)}_{>0}$, ent\~ao $f(x) > 0$.

\begin{sloppypar}
$f$ \'e deriv\'avel em $b$ e $f'(b) > 0$, ent\~ao existe $f_2$ cont\'inua em $b$ tal que ${f(x) = f(b) + (x - b)f_2(x)}$ e $f_2(b) = f'(b)$.
\end{sloppypar}

Como $f'(b) > 0$, fixe $\varepsilon = \frac{f'(b)}{2} > 0$.

\begin{sloppypar}
Como $f_2$ \'e cont\'inua em $b$, existe $\delta_2 > 0$ tal que se $|x - b| < \delta_2$, ent\~ao ${\left| {f_2 (x) - \underbrace {f_2 (b)}_{f'(b)}} \right| < \varepsilon  = \frac{{f'(b)}}{2}}$.
\end{sloppypar}

\[
\begin{gathered}
  |f_2(x) - f'(b)| < \frac{f'(b)}{2} \sse \frac{-f'(b)}{2} < f_2(x) - f'(b) < \frac{f'(b)}{2} \hfill \\
  \sse 0 < \frac{f'(b)}{2} < f_2(x) < \frac{3f'(b)}{2} \hfill \\ 
\end{gathered} 
\]

Tome $y \in (b - \delta_2,b)$. Como $f(y) = \underbrace{(y - b)}_{<0} \underbrace{f_2(y)}_{>0}$, ent\~ao $f(y) < 0$.

Conclus\~ao: Como $f$ \'e deriv\'avel, $f$ \'e cont\'inua, portanto, usando o TVI. (\ref{tvi}), $\exists c \in (x,y) \subset (a,b)$ tal que $f(c) = 0$.
\end{sol}

\begin{exerc}
Sejam $f$ uma fun\c c\~ao deriv\'avel em $I$ e $a < b$ n\'umeros reais em $I$ tais que $f'(a)f'(b) < 0$. Prove que existe $c \in (a,b)$ tal que $f'(c) = 0$.
\end{exerc}

\begin{sol}
Note que $f'(a)$ e $f'(b)$ possuem sinais contr\'arios. Suponhamos $f'(a) > 0$ e $f'(b) < 0$, e suponha ainda que $f(a) \ne f(b)$.

\begin{figure}[!h]
  \centering
  \subfloat[fig29]{
    \input{figuras/fig29}
    \label{fig29}
  }
  \quad
  \subfloat[fig30]{
    \input{figuras/fig30}
    \label{fig30}
  }
\end{figure}

Precisamos encontrar $x,y \in (a,b)$ com $x < y$ tal que $f(x) = f(y)$, e a tese sai do Teorema de Rolle (\ref{tr}).

Suponha que $f$ \'e estritamente crescente, isto \'e, $\forall x,y; x < y \imp f(x) < f(y)$.

\begin{sloppypar}
Por hip\'otese, $f$ \'e deriv\'avel em $b$, ent\~ao existe $f_1$ cont\'inua em $a$ tal que ${f(x) = f(b) + (x - b)f_1(x)}$ e $f_1(b) = f'(b)$.
\end{sloppypar}

Lembrando que $f'(b) = f_1(b) < 0$, ent\~ao $\exists \delta$ tal que $\forall y \in (b - \delta,b), f_1(y) < 0$.

Ent\~ao, se $f$ \'e estritamente crescente, $f(y) - f(b) < 0 \imp y - b < 0$ \\
$\imp \frac{f(y) - f(b)}{y - b} > 0 \imp f_(y) > 0$. Absurdo.

Conclu\'imos que $f$ n\~ao \'e estritamente crescente. Ent\~ao, $\exists x,y \in (a,b)$ com $x < y$ tal que $f(x) \mai f(y)$.

1\textordmasculine caso) $f(x) = f(y)$. \'E imediato com o Teorema de Rolle.

2\textordmasculine caso) $f(x) > f(y)$. Se $f(a) < f(y)$, aplique o TVI e conclu\'imos que $\exists z \in (a,x)$ tal que $f(z) = f(y)$.

Tome $z,y: z < y$ e conclui-se com o Teorema de Rolle (Fig. fig30).

Aplicando o TVI encontramos $c \in (x,y)$ tal que $f(c) = f(a)$.

\end{sol}





\chapter{C\'alculo no $\R^n$} \label{chap02}

\section{Curvas no $\R^n$} \label{sec07}

Seja $\R^n = \left\{ {\left( {x_1 ,x_2 ,...,x_n } \right);x_i  \in \R;1 \leqslant i \leqslant n} \right\}$

Escrevemos $x = \left( {x_1 ,x_2 ,...,x_n } \right)$

\begin{defn}
    Uma curva em $\R$ \'e uma fun\c c\~ao $\gamma :I \subset \R \to \R^n$.

    $\gamma \left( t \right) = \left( {x_1 \left( t \right),x_2 \left( t \right),...,x_n \left( t \right)} \right)$ onde $x_i :I \to \R$.

$\gamma$ \'e cont\'inua se cada $x_i$ for cont\'inua.

    $\gamma$ \'e deriv\'avel se cada $x_i$ for deriv\'avel e $\gamma '\left( t \right) = \left( {x_1 '\left( t \right),x_2 '\left( t \right),...,x_n '\left( t \right)} \right)$

    $\operatorname{Im} \gamma  = \left\{ {\gamma \left( t \right):t \in I} \right\} \subset \R^n$

    gr\'afico $\gamma  = \left\{ {\left( {t,\gamma \left( t \right)} \right):t \in I} \right\} \subset \R \times \R^n  \approx \R^{n + 1}$

    onde: $\R \times \R^n  = \left( {x_0 ,\left( {x_1 ,x_2 ,...,x_n } \right)} \right)$ e $\R^{n + 1}  = \left( {x_0 ,x_1 ,...,x_n } \right)$.
\end{defn}

\begin{ex}
Seja

\[
\begin{gathered}
\gamma :\R \to \R^2  \hfill \\
  \gamma \left( t \right) = \left( {\cos t,\sin t} \right);t \in \left[ {0,2\pi } \right] \hfill \\
\end{gathered}
\]
\end{ex}

\begin{sol}
    Note que $\gamma \left( t \right) = \left( {x\left( t \right),y\left( t \right)} \right)$

\[
\begin{gathered}
      \left\| {\gamma \left( t \right)} \right\| = \sqrt {\cos ^2 t + \sin ^2 t}  = 1 \hfill \\
\Rightarrow x^2 \left( t \right) + y^2 \left( t \right) = 1 \hfill \\
\end{gathered}
\]

\myfig{fig11}{Circunfer\^encia}

Para ver o ponto inicial basta ver que

para $t=0 \Rightarrow \left( {1,0} \right)$ e

para $t=1 \Rightarrow \left( {0,1} \right)$

    ou por derivadas $\gamma '\left( t \right) = \left( { - \sin t,\cos t} \right)$ em $\left[ {0,{\raise0.5ex\hbox{$\scriptstyle \pi $}
\kern-0.1em/\kern-0.15em
    \lower0.25ex\hbox{$\scriptstyle 2$}}} \right]$ a derivada decresce, ent\~ao, $x$ tende a $0$, portanto, o sentido de rota\c c\~ao \'e anti-hor\'ario.

    Ent\~ao, a $\operatorname{Im}  \in \R^2$ \'e uma circunfer\^encia de raio unit\'ario.

    Seu gr\'afico \'e $\gamma  \subset \R^3$ $\gamma \left( t \right)= \left\{ {\left( {t,\cos t,\sin t} \right);t \in \left[ {0,2\pi } \right]} \right\}$

\myfig{fig12}{H\'elice}
\end{sol}

\begin{ex}
    Seja $\gamma \left( t \right) = \left( {e^{ - t} \cos t,e^{ - t} \sin t} \right),t \geqslant 0$
\end{ex}

\begin{sol}
Calculando a norma do vetor podemos ver a regi\~ao onde $t$ varia.

\[
\left\| {\gamma \left( t \right)} \right\| = e^{ - t}
\]

Calculando a derivada podemos ver o sentido de crescimento da curva.

\begin{eqnarray*}
      \gamma '\left( t \right) &=& \left( { - e^{ - t} \cos t - e^{ - t} \sin t, - e^{ - t} \sin t + e^{ - t} \cos t} \right) \hfill \\
       &=& e^{ - t} \left( { - \cos t - \sin t, - \sin t + \cos t} \right) \hfill \\
\end{eqnarray*}

Portanto a $\operatorname{Im} \gamma$ \'e dada pela figura a seguir.

% figura 13

\myfig{fig13}{}

\end{sol}

\begin{ex}
Seja

\[
    \gamma \left( t \right) = \left( {2\cos t,\sin t} \right),t \in \left[ {0,2\pi } \right]
\]

\end{ex}

\begin{sol}
\[
\begin{gathered}
  \left\| {\gamma \left( t \right)} \right\| = \sqrt {4\cos ^2 t + \sin ^2 t}  \hfill \\
x^2 \left( t \right) = 4\cos ^2 t \Rightarrow \frac{{x^2 \left( t \right)}}
{4} = \cos ^2 t \hfill \\
y^2 \left( t \right) = \sin ^2 t \hfill \\
\left[ {\cos ^2 t + \sin ^2 t = 1} \right] \hfill \\
  \operatorname{Im} \gamma  \subset \left\{ {\left( {x,y} \right) \in \R^2 :\frac{{x^2 }}
{4} + y^2  = 1} \right\} \hfill \\
\end{gathered}
\]

% figura 14

\myfig{fig14}{Elipse}

A reta tangente a $\gamma \left( t \right)$ em $\gamma \left( t_0 \right)$ \'e

\[
\begin{gathered}
r:x = \gamma \left( {t_0 } \right) + s.\gamma '\left( {t_0 } \right) \hfill \\
{\text{p/ }}t_0  = {\raise0.5ex\hbox{$\scriptstyle 3$}
\kern-0.1em/\kern-0.15em
\lower0.25ex\hbox{$\scriptstyle 4$}} \Rightarrow \gamma \left( {t_0 } \right) = \left( {2. - \tfrac{{\sqrt 2 }}
{2},\tfrac{{\sqrt 2 }}
{2}} \right) = \left( { - \sqrt 2 ,\tfrac{{\sqrt 2 }}
{2}} \right) \hfill \\
\gamma '\left( t \right) = \left( { - 2\sin t,\cos t} \right) \hfill \\
\gamma '\left( {t_0 } \right) = \left( { - 2\tfrac{{\sqrt 2 }}
{2}, - \tfrac{{\sqrt 2 }}
{2}} \right) \hfill \\
r:x = \left( { - \sqrt 2 ,\tfrac{{\sqrt 2 }}
{2}} \right) + s\left( { - \sqrt 2 , - \tfrac{{\sqrt 2 }}
{2}} \right),s \in \R \hfill \\
\end{gathered}
\]

\end{sol}

\begin{ex}
Seja

\[
\gamma \left( t \right) = \left( {t,\cos t,\sin t} \right),t \geqslant 0
\]

\end{ex}

\begin{sol}
\[
\begin{gathered}
\gamma '\left( t \right) = \left( {1, - \sin t,\cos t} \right) \hfill \\
y^2  + z^2  = 1 \hfill \\
\end{gathered}
\]

% figura 15
\myfig{fig15}{H\'elice}
\end{sol}

\newpage 

\begin{ex}
Seja

\[
    \gamma \left( t \right) = \left( {e^{ - t} \cos t,e^{ - t} \sin t,e^{ - t} } \right),t \geqslant 0
\]

\end{ex}

\begin{sol}
\[
\begin{gathered}
\left\| {\gamma \left( t \right)} \right\| = \sqrt 2 e^{ - t}  \hfill \\
x^2  + y^2  = \left( {e^{ - t} } \right)^2  = z^2  \hfill \\
\end{gathered}
\]

% figura 16
\myfig{fig16}{H\'elice crescente}
\end{sol}

\section{Comprimento de Curvas} \label{sec08}

\begin{defn}
    Seja $\gamma :\left[ {a,b} \right] \to \R^n$ uma curva com derivada cont\'inua em $\left[ {a,b} \right]$. O comprimento de $\gamma$ \'e:

\[\boxed{
    L\left( \gamma  \right) = \int\limits_a^b {\left\| {\gamma '\left( t \right)} \right\|dt}}
\]
\end{defn}

% figura 17
\myfig{fig17}{Comprimento de curva}

\begin{dem}
\[
    \sum\limits_{i = 1}^n {\left\| {\gamma \left( {t_i } \right) - \gamma \left( {t_{i - 1} } \right)} \right\|}  = \sum\limits_{i = 1}^n {\left\| {\gamma '\left( {c_i } \right)} \right\|\Delta t_i }
\]

fazendo $n \to \infty$ temos

\[
    L\left( \gamma  \right) = \int\limits_a^b {\left\| {\gamma '\left( t \right)} \right\|dt}
\]

\end{dem}

\begin{ex}
Seja

\[
    \gamma \left( t \right) = \left( {t,\cos t,\sin t} \right),t \in \left[ {0,2\pi } \right]
\]

\end{ex}

\begin{sol}
\[
\begin{gathered}
  L\left( t \right) = \int\limits_0^{2\pi } {\left\| {\gamma '\left( t \right)} \right\|dt}  \hfill \\
\gamma '\left( t \right) = \left( {1, - \sin t,\cos t} \right) \hfill \\
\left\| {\gamma '\left( t \right)} \right\| = \sqrt 2  \hfill \\
   \Rightarrow L\left( t \right) = \int\limits_0^{2\pi } {\sqrt 2 dt}  = \left. {\sqrt 2 t} \right|_0^{2\pi }  = 2\pi \sqrt 2  \hfill \\
\end{gathered}
\]

\end{sol}

\newpage 

\begin{ex}
Calcule o comprimento de uma circunfer\^encia de raio $r$.

\end{ex}

\begin{sol}
\[
\begin{gathered}
x^2  + y^2  = r^2  \hfill \\
\left. \begin{gathered}
x\left( t \right) = r\cos t \hfill \\
y\left( t \right) = r\sin t \hfill \\
\end{gathered}  \right\}\gamma \left( t \right) = \left( {r\cos t,r\sin t} \right),t \in \left[ {0,2\pi } \right] \hfill \\
  L\left( \gamma  \right) = \int\limits_0^{2\pi } {\left\| {\gamma '\left( t \right)} \right\|dt}  \hfill \\
\gamma '\left( t \right) = \left( { - r\sin t,r\cos t} \right) \hfill \\
\left\| {\gamma '\left( t \right)} \right\| = r \hfill \\
  L\left( \gamma  \right) = \int\limits_0^{2\pi } {rdt}  = \left. {rt} \right|_0^{2\pi }  = 2\pi r \hfill \\
\end{gathered}
\]

\end{sol}

\begin{ex}
Calcule o comprimento de uma elipse.

\end{ex}

\begin{sol}
\[
\begin{gathered}
\frac{{x^2 }}
{{a^2 }} + \frac{{y^2 }}
{{b^2 }} = 1 \hfill \\
x\left( t \right) = a\cos t \hfill \\
y\left( t \right) = b\sin t \hfill \\
\end{gathered}
\]

N\~ao \'e poss\'ivel calcular o comprimento de uma elipse apenas com t\'ecnicas de integra\c c\~ao.

\end{sol}

\begin{ex}
Calcule o comprimento:

\begin{enumerate}[a)]
      \item $\gamma \left( t \right) = \left( {t\cos t,t\sin t} \right),0 \leqslant t \leqslant 2\pi$
      \item $\gamma \left( t \right) = \left( {\cos 2t,\sin 2t} \right),0 \leqslant t \leqslant 2\pi$
\end{enumerate}
\end{ex}

\newpage 

\section{Fun\c c\~oes de V\'arias Vari\'aveis a Valores Reais} \label{sec09}

\begin{defn}
    Seja $A \subset \R^n$ uma fun\c c\~ao $f:A \to \R$ \'e uma tripla $\left( {f,A,\R} \right)$ onde $f$ \'e uma regra que associa a cada ponto de $A$ um \'unico ponto em $\R$.

Nota\c c\~ao:
\begin{eqnarray*}
f: &&A \to \R \hfill \\
&&x \mapsto f\left( x \right) \hfill \\
\end{eqnarray*}

    A imagem de $f$ \'e $\operatorname{Im} f = \left\{ {f\left( x \right):x \in A} \right\} = f\left( A \right)$

    O gr\'afico de $f$ \'e ${\text{graf}}f = \left\{ {\left( {x,f\left( x \right)} \right):x \in A} \right\} \subset \R^{n + 1}$.
\end{defn}

\begin{defn}
    Seja $f:A \subset \R^n  \to \R$ a \textit{hipersuperf\'icie}\index{Hipersuperf\'icie} de n\'ivel de $c$ de $f$ \'e $f^{ - 1} \left( c \right) = \left\{ {x \in A:f\left( x \right) = c} \right\}$.
\end{defn}

\textbf{Observa\c c\~ao:}

Se $n=2$ $f^{ - 1} \left( c \right)$ \'e uma \textit{curva} em $\R^2$.
Se $n=3$ $f^{ - 1} \left( c \right)$ \'e uma \textit{superf\'icie} em $\R^3$.

\begin{ex}
Seja $f\left( {x,y} \right) = 2x^2  + y^2$

Determine o dom\'inio e a imagem, desenhe as curvas de n\'ivel e esboce o gr\'afico.
\end{ex}

\begin{sol}
    Verifiquemos o valor de $f\left( {x,y} \right)$ quando comparamos com um valor $c$.

\[
\begin{gathered}
c < 0:f^{ - 1} \left( c \right) = \emptyset  \hfill \\
  c = 0:f^{ - 1} \left( c \right) = \left\{ {\left( {0,0} \right)} \right\} \hfill \\
  c > 0:f^{ - 1} \left( c \right) = \left\{ {\left( {x,y} \right) \in \R^2 :f\left( {x,y} \right) = c} \right\} \Rightarrow 2x^2  + y^2  = c \hfill \\
\end{gathered}
\]

Figuras (\ref{fig18}) e (\ref{fig19}).

\begin{figure}[!h]
  \centering
  \subfloat[Curva de n\'ivel: elipse]{
    \includegraphics{figuras/fig18}
    \label{fig18}
  }
  \quad
  \subfloat[Gr\'afico: elips\'oide]{
    \includegraphics{figuras/fig19}
    \label{fig19}
  }
  \caption{Elipse e elips\'oide}
\end{figure}

\end{sol}

\begin{ex}
Seja $f\left( {x,y} \right) = x^2 - y^2$

Determine o dom\'inio e a imagem, desenhe as curvas de n\'ivel e esboce o gr\'afico.
\end{ex}

\newpage 

\begin{sol}
\[
\begin{gathered}
  c < 0:f^{ - 1} \left( c \right) = \left\{ {\left( {x,y} \right) \in \R^2 :f\left( {x,y} \right) = c} \right\} \hfill \\
\Rightarrow x^2  - y^2  = c < 0 \hfill \\
\Rightarrow y^2  - x^2  =  - c > 0 \hfill \\
c = 0:f\left( {x,y} \right) = 0 \hfill \\
\Rightarrow x^2  - y^2  = 0 \hfill \\
\Rightarrow x^2  = y^2  \hfill \\
\Rightarrow \left| x \right| = \left| y \right| \hfill \\
\Rightarrow y =  \pm x \hfill \\
c > 0:x^2  - y^2  = c > 0 \hfill \\
\end{gathered}
\]

Figuras (\ref{fig20}) e (\ref{fig21}).

\begin{figure}[!h]
  \centering
  \subfloat[Curva de n\'ivel: hiperbol\'oide]{
    \includegraphics{figuras/fig20}
    \label{fig20}
  }
  \quad
  \subfloat[Gr\'afico: parabol\'oide hiperb\'olico]{
    \includegraphics{figuras/fig21}
    \label{fig21}
  }
  \caption{Hiperbol\'oide e parabol\'oide hiperb\'olico}
\end{figure}

\end{sol}

\begin{ex}
Seja $\displaystyle f\left( {x,y} \right) = \frac{y}{x - 1}$

Determine o dom\'inio e a imagem, desenhe as curvas de n\'ivel e esboce o gr\'afico.
\end{ex}

\newpage 

\begin{sol}
\[
D_f  = \left\{ {x \in \R:x - 1 \ne 0} \right\}
\]

Fazendo

\[
\begin{gathered}
\frac{y}
{{x - 1}} = c \hfill \\
y = c\left( {x - 1} \right) \hfill \\
y = cx - c \hfill \\
\end{gathered}
\]

% figura 22 Curva de n\'ivel: feixe de retas
% figura 23 Gr\'afico

% \begin{figure}[!h]
%   \centering
%   \subfloat[Curva de n\'ivel: feixe de retas]{
%     \includegraphics{figuras/fig22}
%     \label{fig22}
%   }
%   \quad
%   \subfloat[Gr\'afico]{
%     \includegraphics{figuras/fig23}
%     \label{fig23}
%   }
% \end{figure}

\end{sol}

\begin{ex}
Seja $f\left( {x,y} \right) = \ln \left( {x - y} \right)$

Determine o dom\'inio e a imagem, desenhe as curvas de n\'ivel e esboce o gr\'afico.
\end{ex}

\begin{sol}
\[
D_f  = \left\{ {\left( {x,y} \right) \in \R^2 :x - y > 0} \right\}
\]

% figura 24 
\myfig[height=3cm]{fig24}{Dom\'inio de $f(x,y)$}

\[
\begin{gathered}
\ln \left( {x - y} \right) = c \hfill \\
\Rightarrow x - y = e^c  \hfill \\
\Rightarrow y = x - e^c  \hfill \\
\end{gathered}
\]

% figura 25 Curva de n\'ivel: feixe de retas paralelas
% figura 26 Gr\'afico

% % % % \begin{figure}[!h]
% % % %   \centering
% % % %   \subfloat[Curva de n\'ivel: feixe de retas paralelas]{
% % % %     \includegraphics{figuras/fig25}
% % % %     \label{fig25}
% % % %   }
% % % %   \quad
% % % %   \subfloat[Gr\'afico]{
% % % %     \includegraphics{figuras/fig26}
% % % %     \label{fig26}
% % % %   }
% % % % \end{figure}

\end{sol}

\newpage

\begin{ex}
$f\left( {x,y,z} \right) = x^2  + y^2  + z^2$
\end{ex}

\begin{sol}
\[
\begin{gathered}
c < 0:f^{ - 1} \left( c \right) = \emptyset  \hfill \\
  c = 0:f^{ - 1} \left( 0 \right) = \left\{ {\left( {0,0,0} \right)} \right\} \hfill \\
c > 0:x^2  + y^2  + z^2  = c > 0 \hfill \\
\end{gathered}
\]

% \begin{figure}[!htb]
%   \centering
%   \includegraphics[width=3.5cm]{fig27}
%   \caption{Curva de n\'ivel: conjunto de esferas.}\label{fig27a}
% \end{figure}

\myfig[width=3.5cm]{fig27bola}{Curva de n\'ivel: conjunto de esferas.}

\textbf{Obs:} Imposs\'ivel desenhar o gr\'afico de $f(x,y)$, pois est\'a em $\R^4$.
\end{sol}

\begin{ex}
$f\left( {x,y,z} \right) = x^2  + 4y^2  + z^2$
\end{ex}

\begin{sol}
\[
\begin{gathered}
c < 0:f^{ - 1} \left( c \right) = \emptyset  \hfill \\
  c = 0:f^{ - 1} \left( 0 \right) = \left\{ {\left( {0,0,0} \right)} \right\} \hfill \\
c > 0:x^2  + 4y^2  + z^2  = c > 0 \hfill \\
\end{gathered}
\]

% figura 28 Curva de n\'ivel: conjunto de elips\'oides

% \myfig{fig28}{Curva de n\'ivel: conjunto de elips\'oides}
\end{sol}

\begin{ex}
$f\left( {x,y,z} \right) = x^2  - y^2  - z^2$
\end{ex}

\section{Limite} \label{sec10}

\begin{defn}[Ponto de acumula\c c\~ao]
Seja $A \subset \R^n$ e $x_0 \in \R^n$. O ponto $x_0 \in \R^n$ \'e de \textit{acumula\c c\~ao}\index{Ponto!de acumula\c c\~ao} se $B_\varepsilon  \left( {x_0 } \right):\left\{ {x_0 } \right\} \cap A \ne \emptyset ,\forall \varepsilon  > 0$.
\end{defn}

\begin{ex}
Seja $A = \left[ {0,1} \right]$.

$1$ \'e ponto de acumula\c c\~ao.

\tkzfig{fig30}{Limite}
\end{ex}

\begin{ex}
Seja $A = \left[ {0,1} \right[ \cup \left\{ 2 \right\}$.

$2$ n\~ao \'e ponto de acumula\c c\~ao.
\end{ex}

\begin{defn}[Limite]
    Seja $f:A \subset \R^n  \to \R$ uma fun\c c\~ao e $x_0$ ponto de acumula\c c\~ao de $A$. Dizemos que $f$ tem \textit{limite}\index{Limite} $L$ em $x_0$ se dada uma bola centrada em $L$, $B_\varepsilon  \left( L \right)$, existe bola centrada em $x_0$, $B_\delta  \left( {x_0 } \right)$, tal que $f\left( {B_\delta  \left( {x_0 } \right) \cap A} \right) \subset B_\varepsilon  \left( L \right)$.

\

Nota\c c\~ao: $\displaystyle \mathop {\lim }\limits_{x \to x_0 } f\left( x \right) = L$

\

Em outras palavras,

    Dado $\varepsilon  > 0$, existe $\delta  > 0$ tal que $0 < \left\| {x - x_0 } \right\| < \delta  \Rightarrow \left| {f\left( x \right) - L} \right| < \varepsilon$.
\end{defn}

\begin{ex}
    Seja $f:\R^2  - \left\{ {\left( {0,0} \right)} \right\} \to \R$ dado por $\displaystyle f\left( {x,y} \right) = \frac{{ - x^2  + y^2 }}{{x^2  + y^2 }}$. Calcule o limite da fun\c c\~ao quando $(x,y)$ tende a $(0,0)$.
\end{ex}

\begin{sol}
Tomando duas retas que passam no ponto $(0,0)$, temos:

$f\left( {x,0} \right) =  - 1$ e $f\left( {0,y} \right) = 1$

Portanto, o limite n\~ao existe em $(0,0)$.
\end{sol}

\begin{teo}[Fun\c c\~ao Composta]\index{Fun\c c\~ao!composta}
    Seja $A \subset \R^n$ e $x_0$ ponto de acumula\c c\~ao de $A$ $f:A \subset \R^n  \to \R$. Seja ainda $\gamma$ uma curva cont\'inua tal que $\gamma \left( t \right) \in A,\forall t \ne t_0$ e $\gamma \left( {t_0 } \right) = x_0$.

    Se $\mathop {\lim }\limits_{x \to x_0 } f\left( x \right) = L$, ent\~ao $\mathop {\lim }\limits_{t \to t_0 } f\left( {\gamma \left( t \right)} \right) = L$.
\end{teo}

\begin{dem}
Sejam

\begin{eqnarray*}
\gamma &:&\R \to \R^n  \hfill \\
f&:&\R^n  \to \R \hfill \\
f \circ \gamma &:&\R \to \R \hfill \\
&& t \to f\left( {\gamma \left( t \right)} \right) \hfill \\
\end{eqnarray*}

    Hip\'otese, $\mathop {\lim }\limits_{x \to x_0 } f\left( x \right) = L$, ent\~ao, dada $B_\varepsilon  \left( L \right) \Rightarrow \exists B_\delta  \left( {x_0 } \right):f\left( {B_\delta  \left( {x_0 } \right) \cap A} \right) \subset B_\varepsilon  \left( L \right)$.

% figura 32

    $\gamma$ \'e cont\'inua, ent\~ao, dada $B_\delta  \underbrace {\left( {x_0 } \right)}_{\gamma \left( {t_0 } \right)}$, existe $B_r \left( {t_0 } \right)$ tal que $\gamma \left( {B_r \left( {t_0 } \right)} \right) \subset B_\delta  \left( {x_0 } \right)$

\[
\begin{gathered}
  f\left( {\gamma \left( {B_r \left( {t_0 } \right)} \right)} \right) \subset f\left( {B_\delta  \left( {x_0 } \right) \cap A} \right) \subset B_\varepsilon  \left( L \right) \hfill \\
   \Rightarrow \mathop {\lim }\limits_{t \to t_0 } f\left( {\gamma \left( t \right)} \right) = L \hfill \\
\end{gathered}
\]

\end{dem}

\textbf{Obs:} Se $\gamma_1$ e $\gamma_2$ s\~ao curvas como no teorema e $\mathop {\lim }\limits_{t \to t_0 } f\left( {\gamma _1 \left( t \right)} \right) \ne \mathop {\lim }\limits_{t \to t_0 } f\left( {\gamma _2 \left( t \right)} \right)$, ent\~ao, $\nexists \mathop {\lim }\limits_{x \to x_0 } f\left( x \right)$.

\begin{ex}
No exemplo anterior usamos:
\end{ex}

\begin{sol}
    $\gamma _1 \left( t \right) = \left( {t,0} \right)$ e $\gamma _2 \left( t \right) = \left( {0,t} \right)$,

     ent\~ao, $f\left( {\gamma _1 \left( t \right)} \right) = f\left( {t,0} \right) =  - 1$ e $f\left( {\gamma _2 \left( t \right)} \right) = f\left( {0,t} \right) = 1$
\end{sol}

\begin{teo}[do confronto]\index{Teorema!do confronto}
    Sejam $A \subset \R^n$, $x_0$ ponto de acumula\c c\~ao de $A$ e $f,g,h$ fun\c c\~oes de $A$ em $\R$ tais que $f\left( x \right) \leqslant g\left( x \right) \leqslant h\left( x \right)$ para $x \in B_r \left( {x_0 } \right) \cap A$

    se $\mathop {\lim }\limits_{x \to x_0 } f\left( x \right) = \mathop {\lim }\limits_{x \to x_0 } h\left( x \right) = L$ ent\~ao, $\mathop {\lim }\limits_{x \to x_0 } h\left( x \right) = L$.
\end{teo}

\begin{dem}
    $\mathop {\lim }\limits_{x \to x_0 } f\left( x \right) = L \Rightarrow$ dada $B_\varepsilon  \left( L \right)$ existe $B_{\delta _1 } \left( {x_0 } \right)$ tal que $f\left( {B_{\delta _1 } \left( {x_0 } \right) \cap A} \right) \subset B_\varepsilon  \left( L \right)$

    $\mathop {\lim }\limits_{x \to x_0 } h\left( x \right) = L \Rightarrow$ dada $B_\varepsilon  \left( L \right)$ existe $B_{\delta _2 } \left( {x_0 } \right)$ tal que $h\left( {B_{\delta _2 } \left( {x_0 } \right) \cap A} \right) \subset B_\varepsilon  \left( L \right)$

    Seja $z \in B_{\delta _1 } \left( {x_0 } \right) \cap B_{\delta _2 } \left( {x_0 } \right) \cap A$, ent\~ao

\[
\begin{gathered}
      f\left( z \right) \leqslant g\left( z \right) \leqslant h\left( z \right) \hfill \\
      \left| {f\left( z \right) - L} \right| < \varepsilon  \Rightarrow \underbrace { - \varepsilon  < f\left( z \right) - L}_{} < \varepsilon  \hfill \\
      \left| {h\left( z \right) - L} \right| < \varepsilon  \Rightarrow  - \varepsilon  < \underbrace {h\left( z \right) - L < \varepsilon }_{} \hfill \\
       - \varepsilon  < f\left( z \right) - L \leqslant g\left( x \right) - L \leqslant h\left( z \right) - L < \varepsilon  \hfill \\
\left| {g\left( x \right) - L} \right| < \varepsilon  \hfill \\
       \Rightarrow \mathop {\lim }\limits_{x \to x_0 } g\left( x \right) = L \hfill \\
\end{gathered}
\]

\end{dem}

\begin{cor}
    Sejam $A \subset \R^n$, $x_0$ ponto de acumula\c c\~ao de $A$ e $f,g$ fun\c c\~oes de $A$ em $\R$ com $\left| {g\left( x \right)} \right| < M$ (limitada), para $x \in B_r \left( {x_0 } \right) \cap A$ e $\mathop {\lim }\limits_{x \to x_0 } f\left( x \right) = 0$. Ent\~ao, $\mathop {\lim }\limits_{x \to x_0 } f\left( x \right)g\left( x \right) = 0$.
\end{cor}

\begin{dem}
\[
\begin{gathered}
  \left| {f\left( x \right)g\left( x \right)} \right| < \left| {f\left( x \right)} \right|M \hfill \\
   - M\left| {f\left( x \right)} \right| < f\left( x \right)g\left( x \right) < M\left| {f\left( x \right)} \right| \hfill \\
  \mathop {\lim }\limits_{x \to x_0 }  \pm M\left| {f\left( x \right)} \right| = 0 \hfill \\
   \Rightarrow \mathop {\lim }\limits_{x \to x_0 } f\left( x \right)g\left( x \right) = 0 \hfill \\
\end{gathered}
\]

\end{dem}

\textbf{Exerc\'icios}

Prove que:

\begin{exerc}
    $\mathop {\lim }\limits_{x \to x_0 } \left| {f\left( x \right)} \right| = 0 \Leftrightarrow \mathop {\lim }\limits_{x \to x_0 } f\left( x \right) = 0$
\end{exerc}

\begin{exerc}
    $\mathop {\lim }\limits_{x \to x_0 } f\left( x \right) - L = 0 \Leftrightarrow \mathop {\lim }\limits_{x \to x_0 } f\left( x \right) = L$
\end{exerc}

\begin{exerc}
    $\mathop {\lim }\limits_{x \to x_0 } f\left( x \right) = L \Leftrightarrow \mathop {\lim }\limits_{h \to 0} f\left( {x_0  + h} \right) = L,\left( {h \in \R^n } \right)$
\end{exerc}

\begin{exerc}
    Se $\mathop {\lim }\limits_{x \to x_0 } f\left( x \right) > 0$, ent\~ao, $\exists B_r \left( {x_0 } \right)$ tal que $f\left( x \right) > 0,\forall x \in B_r \left( {x_0 } \right)$.
\end{exerc}

\textbf{Nota:} As propriedades de limite s\~ao as mesmas das fun\c c\~oes de uma vari\'avel.

\begin{ex}
    Calcule $\mathop {\lim }\limits_{\left( {x,y} \right) \to \left( {0,0} \right)} \frac{{x^3 }}{{x^2  + y^2 }}$
\end{ex}

\begin{sol}
\[
    \mathop {\lim }\limits_{\left( {x,y} \right) \to \left( {0,0} \right)} \frac{{x^3 }}{{x^2  + y^2 }} = \mathop {\lim }\limits_{\left( {x,y} \right) \to \left( {0,0} \right)} x\overbrace {\frac{{x^2 }}
{{x^2  + y^2 }}}^{{\text{limitada}}} = 0
\]

    Pois $0 \leqslant x^2  \leqslant x^2  + y^2  \Rightarrow 0 \leqslant \frac{{x^2 }}
{{x^2  + y^2 }} \leqslant 1$ \'e limitada.
\end{sol}

\begin{ex}
    Calcule $\mathop {\lim }\limits_{\left( {x,y} \right) \to \left( {0,0} \right)} \frac{{x^2 }}{{x^2  + y^2 }}$
\end{ex}

\begin{sol}
Tomemos duas curvas:

\[
\begin{gathered}
  \gamma _1 \left( t \right) = \left( {t,0} \right) \Rightarrow f\left( {\gamma _1 \left( t \right)} \right) = f\left( {t,0} \right) = \frac{{t^2 }}
{{t^2  + 0}} = 1 \Rightarrow \mathop {\lim }\limits_{t \to 0} f\left( {\gamma _1 \left( t \right)} \right) = 1 \hfill \\
  \gamma _2 \left( t \right) = \left( {0,t} \right) \Rightarrow f\left( {\gamma _2 \left( t \right)} \right) = f\left( {0,t} \right) = \frac{0}
{{0 + t^2 }} = 0 \Rightarrow \mathop {\lim }\limits_{t \to 0} f\left( {\gamma _2 \left( t \right)} \right) = 0 \hfill \\
  \therefore \mathop {\lim }\limits_{t \to 0} f\left( {\gamma _1 \left( t \right)} \right) \ne \mathop {\lim }\limits_{t \to 0} f\left( {\gamma _2 \left( t \right)} \right) \hfill \\
\end{gathered}
\]

Portanto, o limite n\~ao existe em $(0,0)$.
\end{sol}

\begin{ex}
    Calcule $\mathop {\lim }\limits_{\left( {x,y} \right) \to \left( {0,0} \right)} \frac{{x^4 \sin \left( {x^2  + y^2 } \right)}}{{x^4  + y^2 }}$
\end{ex}

\begin{sol}
\[
\mathop {\lim }\limits_{\left( {x,y} \right) \to \left( {0,0} \right)} \frac{{x^4 \sin \left( {x^2  + y^2 } \right)}}
{{x^4  + y^2 }} = \mathop {\lim }\limits_{\left( {x,y} \right) \to \left( {0,0} \right)} \overbrace {\frac{{x^4 }}
{{x^4  + y^2 }}}^{{\text{limitada}}}\sin \left( {x^2  + y^2 } \right) = 0
\]

\end{sol}

\begin{ex}
    Calcule $\mathop {\lim }\limits_{\left( {x,y} \right) \to \left( {0,0} \right)} \frac{{xy}}
{{2x - y^7 }}$
\end{ex}

\begin{sol}
    Se tomarmos $\gamma _1 \left( t \right) = \left( {t,0} \right)$ e $\gamma _2 \left( t \right) = \left( {0,t} \right)$

\[
     \Rightarrow \mathop {\lim }\limits_{t \to 0} f\left( {\gamma _1 \left( t \right)} \right) = \mathop {\lim }\limits_{t \to 0} f\left( {\gamma _2 \left( t \right)} \right) = 0
\]

    Mas n\~ao \'e suficiente, pois pode haver infinitas curvas que passem no ponto, ent\~ao, vamos usar as curvas de n\'ivel para ver o comportamento da fun\c c\~ao.

Curva de n\'ivel de $f: f(x,y)=c$

\[
\begin{gathered}
\frac{{xy}}
{{2x - y^7 }} = c \Rightarrow x = \frac{{y^7 c}}
{{2c - y}} \hfill \\
\gamma \left( t \right) = \left( {\frac{{t^7 c}}
{{2c - t}},t} \right) \hfill \\
t = 0 \Rightarrow \gamma \left( 0 \right) = \left( {0,0} \right) \hfill \\
  f\left( {\gamma \left( t \right)} \right) = c \Rightarrow \mathop {\lim }\limits_{t \to 0} f\left( {\gamma \left( t \right)} \right) = c \hfill \\
\end{gathered}
\]

    O limite de $f\left( {\gamma \left( t \right)} \right)$ pode assumir qualquer valor. Portanto, o limite de $f(x,y)$ n\~ao existe em $(0,0)$.
\end{sol}

\section{Continuidade} \label{sec11}

\begin{defn}
    Sejam $A \subset \R^n$, $x_0 \in A$ ponto de acumula\c c\~ao de $A$. Dizemos que $f$ \'e \textit{cont\'inua}\index{Fun\c c\~ao!cont\'inua} em $x_0$ se

\[\boxed{
    \mathop {\lim }\limits_{x \to x_0 } f\left( x \right) = f\left( {x_0 } \right)}
\]

\end{defn}

\begin{ex}
\begin{equation*}
f(x,y)=\left\{ \begin{array}{cl}\displaystyle
        \frac{{x^2  - y^2 }}{{x^2  + y^2 }} & \textrm{se }\left( {x,y} \right) \ne \left( {0,0} \right)\\
        0 & \textrm{se }\left( {x,y} \right) = \left( {0,0} \right)\end{array}\right.
\end{equation*}
\end{ex}

\begin{sol}
    $\displaystyle \mathop {\lim }\limits_{\left( {x,y} \right) \to \left( {0,0} \right)} f\left( {x,y} \right) = \mathop {\lim }\limits_{\left( {x,y} \right) \to \left( {0,0} \right)} \frac{{x^2  - y^2 }}{{x^2  + y^2 }}$ n\~ao existe, pois, tomando $(t,0)$ e $(0,t)$ verificamos limites diferentes. Ent\~ao $f$ n\~ao \'e cont\'inua em $(0,0)$ mas nos outros pontos ela \'e.
\end{sol}

\begin{ex}
\begin{equation*}
f(x,y)=\left\{ \begin{array}{cl}\displaystyle
\frac{{x^4 \sin \left( {x^2  + y^2 } \right)}}
        {{x^4  + y^2 }} & \textrm{se }\left( {x,y} \right) \ne \left( {0,0} \right)\\
        0 & \textrm{se }\left( {x,y} \right) = \left( {0,0} \right)\end{array}\right.
\end{equation*}

\end{ex}

\begin{sol}
    A fun\c c\~ao \'e cont\'inua em $(0,0)$, $\mathop {\lim }\limits_{\left( {x,y} \right) \to \left( {0,0} \right)} f\left( {x,y} \right) = 0 = f\left( 0 \right)$.
\end{sol}

\begin{ex}
\begin{equation*}
f(x,y)=\left\{ \begin{array}{cl}\displaystyle
\frac{{x^2 \sin \left( \displaystyle {\frac{{x^2 y^8 }}
        {{y^8  + x^4 }}} \right)}}{{x^2  + y^2 }} & \textrm{se }\left( {x,y} \right) \ne \left( {0,0} \right)\\
        0 & \textrm{se }\left( {x,y} \right) = \left( {0,0} \right)\end{array}\right.
\end{equation*}

\end{ex}

\begin{sol}
    $\displaystyle \frac{{x^2 }}{{x^2  + y^2 }}$ \'e limitada e $\displaystyle \sin \left( {\frac{{x^2 y^8 }}{{y^8  + x^4 }}} \right)$ tamb\'em \'e limitada, mas n\~ao resolve a equa\c c\~ao, mas $\displaystyle \frac{{y^8 }}{{y^8  + x^4 }}$ \'e limitada, ent\~ao, $\mathop {\lim }\limits_{\left( {x,y} \right) \to \left( {0,0} \right)} \sin x^2  = 0$.

\[
    \mathop {\lim }\limits_{\left( {x,y} \right) \to \left( {0,0} \right)} f\left( {x,y} \right) = 0 \ne f\left( 0 \right) = 1
\]

Portanto, $f(x,y)$ n\~ao \'e cont\'inua em $(0,0)$.

\end{sol}

\begin{teo}
    Sejam $A \subset \R^n ,B \subset \R$ e fun\c c\~oes $f:A \to \R$ e $g:B \to \R$ tal que $f\left( A \right) \subset B$. Se $f$ \'e cont\'inua em $x_0$ e $g$ \'e cont\'inua em $f\left( {x_0 } \right)$ ent\~ao, $g \circ f$ \'e cont\'inua em $x_0$.
\end{teo}

\begin{dem}
$g$ \'e cont\'inua em $f\left( {x_0 } \right)$
    $\Rightarrow$ dado $B_\varepsilon  \left( {g\left( {f\left( {x_0 } \right)} \right)} \right)$ existe $B_{\delta _1 } \left( {f\left( {x_0 } \right)} \right)$ tal que

\[
    g\left( {B_{\delta _1 } \left( {f\left( {x_0 } \right)} \right) \cap A} \right) \subset B_\varepsilon  \left( {g\left( {f\left( {x_0 } \right)} \right)} \right)
\]

    $f$ \'e cont\'inua em $x_0$ $\Rightarrow \mathop {\lim }\limits_{x \to x_0 } f\left( x \right) = f\left( {x_0 } \right)$

    $\Rightarrow$ para $B_{\delta _1 } \left( {f\left( {x_0 } \right)} \right)$ existe $B_{\delta _2 } \left( {x_0 } \right)$ tal que

\[
    f\left( {B_{\delta _2 } \left( {x_0 } \right) \cap A} \right) \subset B_{\delta _1 } \left( {f\left( {x_0 } \right)} \right)
\]

\[
    g\left( {f\left( {B_{\delta _2 } \left( {x_0 } \right) \cap A} \right)} \right) \subset g\left( {B_{\delta _1 } \left( {f\left( {x_0 } \right)} \right)} \right) \subset B_\varepsilon  \left( {g\left( {f\left( {x_0 } \right)} \right)} \right)
\]

    $\Rightarrow \mathop {\lim }\limits_{x \to x_0 } \left( {g \circ f} \right)\left( x \right) = g\left( {f\left( {x_0 } \right)} \right) \Rightarrow g \circ f$ \'e cont\'inua em $x_0$.
\end{dem}

\begin{teo}
    Sejam $A \subset \R^n ,I \subset \R,f:A \to \R,\gamma :I \to \R^n ,\gamma \left( I \right) \subset A$. Se $\gamma$ \'e cont\'inua em $t_0 \in I$ e $f$ \'e cont\'inua em $\gamma \left( {t_0 } \right) \in A$ ent\~ao, $f \circ \gamma$ \'e cont\'inua em $t_0$.

\end{teo}

\chapter{Derivadas} \label{chap03}

\section{Derivadas Parciais} \label{sec12}

\begin{defn}
    Seja $A \subset \R^n$ e $x_0$ ponto interior de $A$. Definimos a \textit{derivada parcial}\index{Derivada!parcial} de $f$ em $x_0$ na dire\c c\~ao $e_i$ (base can\^onica) por

\[\boxed{
    \frac{{\partial f}}{{\partial x_i }}\left( {x_0 } \right) = \mathop {\lim }\limits_{h \to 0} \frac{{f\left( {x_0  + he_i } \right) - f\left( {x_0 } \right)}}{h}}
\]

\end{defn}

Na pr\'atica, basta derivar $f$ em rela\c c\~ao \`a var\'i\'avel $x_i$, considerando as outras como constantes.

\begin{ex}
$f\left( {x,y} \right) = \arctan \left( {x^2  + y^2 } \right)$
\end{ex}

\begin{sol}
\[
\begin{gathered}
\frac{{\partial f}}
{{\partial x}}\left( {x,y} \right) = \frac{1}
{{1 + \left( {x^2  + y^2 } \right)}}2x \hfill \\
\frac{{\partial f}}
{{\partial y}}\left( {x,y} \right) = \frac{1}
{{1 + \left( {x^2  + y^2 } \right)}}2y \hfill \\
\end{gathered}
\]
\end{sol}

\begin{ex}
$z = f\left( {x,y} \right)$ \'e solu\c c\~ao da equa\c c\~ao $x^2  + y^2  + z^2  = 1$.
\end{ex}

\begin{sol}
Note que $z^2 \left( {x,y} \right)$. Derivando em $x$, temos:

\[
\begin{gathered}
2x + 2z\frac{{\partial f}}{{\partial x}} = 0 \hfill \\
      \frac{{\partial f}}{{\partial x}} =  - \frac{x}{{f\left( {x,y} \right)}} \hfill \\
\frac{{\partial f}}{{\partial y}} =  - \frac{y}
{{f\left( {x,y} \right)}} \hfill \\
\end{gathered}
\]

\end{sol}

\begin{ex}
    Suponha $f$ cont\'inua e $\frac{{\partial f}}{{\partial x}}$ existe. Ent\~ao, $\frac{{\partial f}}{{\partial x}} = 0$.
\end{ex}

\begin{sol}
$f$ n\~ao varia na dire\c c\~ao do eixo $x$.
\end{sol}

\begin{ex}
Seja
\begin{equation*}
f(x,y)=\left\{ \begin{array}{cl}\displaystyle
        \frac{{x^3  - y^2 }}{{x^2  + y^2 }} & \textrm{se }\left( {x,y} \right) \ne \left( {0,0} \right)\\
        0 & \textrm{se }\left( {x,y} \right) = \left( {0,0} \right)\end{array}\right.
\end{equation*}
\end{ex}

\begin{sol}
    Nos pontos $\left( {x,y} \right) \ne \left( {0,0} \right)$ podemos aplicar a regra do quociente

\[
\begin{gathered}
\frac{{\partial f}}
{{\partial x}}\left( {x,y} \right) = \frac{{3x^2 \left( {x^2  + y^2 } \right) - \left( {x^3  - y^2 } \right)2x}}
{{\left( {x^2  + y^2 } \right)^2 }} \hfill \\
\frac{{\partial f}}
{{\partial x}}\left( {x,y} \right) = \frac{{x^4  + 3x^2 y^2  + 2xy^2 }}
{{\left( {x^2  + y^2 } \right)^2 }} \hfill \\
\frac{{\partial f}}
{{\partial y}}\left( {x,y} \right) =  - \frac{{2x^2 y\left( {1 + x} \right)}}
{{\left( {x^2  + y^2 } \right)^2 }} \hfill \\
\end{gathered}
\]

Em $(0,0)$, temos:

\begin{eqnarray*}
  \frac{{\partial f}}{{\partial x}}\left( {0,0} \right) &=& \mathop {\lim }\limits_{h \to 0} \frac{{f\left( {\left( {0,0} \right) + h\left( {1,0} \right)} \right) - f\left( {0,0} \right)}}{h} \hfill \\
   &=& \mathop {\lim }\limits_{h \to 0} \frac{{f\left( {h,0} \right) - f\left( {0,0} \right)}}{h} \hfill \\
&=& \mathop {\lim }\limits_{h \to 0} \frac{{h - 0}}
{h} = 1 \hfill \\
  \frac{{\partial f}}{{\partial y}}\left( {0,0} \right) &=& \mathop {\lim }\limits_{h \to 0} \frac{{f\left( {0,h} \right) - f\left( {0,0} \right)}}{h} \hfill \\
&=& \mathop {\lim }\limits_{h \to 0} \frac{{ - 1 - 0}}{h} \hfill \\
\end{eqnarray*}

que n\~ao existe.
\end{sol}

\textbf{Interpreta\c c\~ao geom\'etrica das derivadas parciais}

% figura 33

\section{Derivadas Direcionais} \label{sec13}

\begin{defn}
    Seja $f:A \subset \R^n \to \R$ e $x_0$ ponto interior de $A$ e $v$ um vetor unit\'ario de $\R^n$. Definimos a \textit{derivada direcional}\index{Derivada!direcional} de $f$ em $x_0$ na dire\c c\~ao de $v$ por

\[\boxed{
    \frac{{\partial f}}{{\partial v }}\left( {x_0 } \right) = \mathop {\lim }\limits_{h \to 0} \frac{{f\left( {x_0  + hv } \right) - f\left( {x_0 } \right)}}{h}}
\]

\end{defn}

\begin{ex}
Seja
\begin{equation*}
f(x,y)=\left\{ \begin{array}{cl}\displaystyle
        \frac{{xy^3}}{{x^2  + y^6 }} & \textrm{se }\left( {x,y} \right) \ne \left( {0,0} \right)\\
        0 & \textrm{se }\left( {x,y} \right) = \left( {0,0} \right)\end{array}\right.
\end{equation*}

Calcule $\frac{{\partial f}}{{\partial v}}\left( {0,0} \right)$.
\end{ex}

\begin{sol}
Seja $v = \left( {a,b} \right) \in \R^2$ com $a^2 + b^2 = 1$.

\begin{eqnarray*}\displaystyle
  \frac{{\partial f}}{{\partial v}}\left( {0,0} \right) &=& \mathop {\lim }\limits_{h \to 0} \frac{{f\overbrace {\left( {\left( {0,0} \right) + h\left( {a,b} \right)} \right)}^{\left( {ha,hb} \right)} - f\left( {0,0} \right)}}{h} \hfill \\
   &=& \mathop {\lim }\limits_{h \to 0} \frac{{\displaystyle  \frac{{h^4 ab^3 }}{{h^2 \left( {a^2  + h^4 b^6 } \right)}} - 0}}
{h} = 0 \hfill \\
\end{eqnarray*}

Ent\~ao, existe a derivada direcional e \'e igual a $0$ em todas as dire\c c\~oes.

Vamos ver agora se $f$ \'e cont\'inua.

Seja $\gamma \left( t \right) = \left( {t^3 ,t} \right)$

\[
    \mathop {\lim }\limits_{t \to 0} f\left( {\gamma \left( t \right)} \right) = \mathop {\lim }\limits_{t \to 0} \frac{{t^6 }}{{t^6  + t^6 }} = \frac{1}{2}\left( { \ne 0 = f\left( {0,0} \right)} \right)
\]

Portanto, $f$ n\~ao \'e cont\'inua em $(0,0)$.
\end{sol}

\section{Diferenciabilidade de Fun\c c\~oes de $\R^n$ em $\R$} \label{sec14}

Relembrando C\'alculo I, temos:

$f$ \'e deriv\'avel em $x_0$ se

\[
\begin{gathered}
  \mathop {\lim }\limits_{h \to 0} \frac{{f\left( {x_0  + h} \right) - f\left( {x_0 } \right)}}
{h} = f'\left( {x_0 } \right) \hfill \\
   \Rightarrow \mathop {\lim }\limits_{h \to 0} \frac{{f\left( {x_0  + h} \right) - f\left( {x_0 } \right) - hf'\left( {x_0 } \right)}}
{h} = 0 \hfill \\
   \Leftrightarrow \mathop {\lim }\limits_{h \to 0} \frac{{f\left( {x_0  + h} \right) - f\left( {x_0 } \right) - f'\left( {x_0 } \right)h}}
{{\left| h \right|}} = 0 \hfill \\
\end{gathered}
\]

Onde $f'\left( {x_0 } \right)$ \'e um n\'umero real que deve existir e $h$ \'e um vetor.

\begin{defn}[Transforma\c c\~ao linear]
\begin{sloppypar}
Uma \textit{transforma\c c\~ao linear}\index{Transforma\c c\~ao!linear} ${T:\R^n  \to \R^m}$ \'e uma fun\c c\~ao tal que
\end{sloppypar}


\begin{enumerate}[i)]
      \item $T\left( {x + y} \right) = T\left( x \right) + T\left( y \right),\forall x,y \in \R^n$
      \item $T\left( {\alpha x} \right) = \alpha T\left( x \right),\forall x \in \R^n {\text{ e }}\forall \alpha  \in \R$
\end{enumerate}
\end{defn}

\begin{defn}
    A matriz de $T$ nas bases can\^onicas de $\R^n$ e $\R^m$ \'e $\left[ T \right] = \left( {a_{ij} } \right)_{m \times n}$ onde $a_{ij}  = T\left( {e_j } \right)_i$ (i-\'esima coordenada de $T\left( {e_j } \right)$, $e_j$ da base can\^onica de $\R^n$).

    Usando isso, temos, $T\left( x \right) = \left[ T \right]_{m \times n} X_{n + 1}$.

    Quando $n=1$, $f$ ser diferenci\'avel em $x_0$ \'e dizer que existe $T_{x_0 } :\R \to \R$ tal que

\[
    \mathop {\lim }\limits_{h \to 0} \frac{{f\left( {x_0  + h} \right) - f\left( {x_0 } \right) - T_{x_0 } \left( h \right)}}{{\left| h \right|}} = 0
\]

e nesse caso, $\left[ T \right] = f'\left( {x_0 } \right)$.
\end{defn}

\begin{defn}[Fun\c c\~ao diferenci\'avel]
    Seja $A \subset \R^n$, $x_0$ ponto interior de $A$ e $f:A \to \R$. Dizemos que $f$ \'e diferenci\'avel em $x_0$ se existir $T_{x_0 } :\R^n  \to \R$ linear tal que

\[\boxed{
    \mathop {\lim }\limits_{h \to 0} \frac{{f\left( {x_0  + h} \right) - f\left( {x_0 } \right) - T_{x_0 } \left( h \right)}}{{\left\| h \right\|}} = 0} \left( 1 \right)
\]

    Onde: $\left\| h \right\|$ \'e a norma do vetor devido ao $\R^n$ e $h \in \R^n$.
\end{defn}

\newpage 

\begin{teo}
Se existe ${T_{x_0 } }$ nas condi\c c\~oes acima ent\~ao ela \'e \'unica.
\end{teo}

\begin{dem}
    Sejam ${T_{x_0 } }$ e ${L_{x_0 } }$ transforma\c c\~oes lineares de $\R^n$ em $\R$ satisfazendo a equa\c c\~ao $(1)$, ent\~ao

\[
    \mathop {\lim }\limits_{h \to 0} \frac{{f\left( {x_0  + h} \right) - f\left( {x_0 } \right) - T_{x_0 } \left( h \right)}}{{\left\| h \right\|}} = 0
\]

e

\[
\begin{gathered}
      \mathop {\lim }\limits_{h \to 0} \frac{{f\left( {x_0  + h} \right) - f\left( {x_0 } \right) - L_{x_0 } \left( h \right)}}{{\left\| h \right\|}} = 0 \hfill \\
       \Rightarrow \mathop {\lim }\limits_{h \to 0} \frac{{T_{x_0 } \left( h \right) - L_{x_0 } \left( h \right)}}{{\left\| h \right\|}} = 0 \hfill \\
\end{gathered}
\]

Escolha $h = h_i e_i$, $e_i$ \'e da base can\^onica de $\R^n$.

\[
\begin{gathered}
      \mathop {\lim }\limits_{h \to 0} \frac{{T_{x_0 } \left( {h_i e_i } \right) - L_{x_0 } \left( {h_i e_i } \right)}}{{\left\| {h_i e_i } \right\|}} = 0 \hfill \\
       \Rightarrow \mathop {\lim }\limits_{h \to 0} \frac{{h_i \left( {T_{x_0 } \left( {e_i } \right) - L_{x_0 } \left( {e_i } \right)} \right)}}{{\left| {h_i } \right|}} = 0 \hfill \\
\end{gathered}
\]

    Como, $\mathop {\lim }\limits_{h_i  \to 0} \frac{{h_i }}{{\left| {h_i } \right|}} = \nexists$ n\~ao existe

\[
\begin{gathered}
   \Rightarrow T_{x_0 } \left( {e_i } \right) - L_{x_0 } \left( {e_i } \right) = 0 \hfill \\
   \Rightarrow T_{x_0 } \left( {e_i } \right) = L_{x_0 } \left( {e_i } \right),\forall i = 1,2,...,n \hfill \\
\Rightarrow L_{x_0 }  = T_{x_0 }  \hfill \\
\end{gathered}
\]

\end{dem}

\begin{teo}
    Seja $A \subset \R^n$, $f:A \to \R$, diferenci\'avel em $x_0$, ent\~ao

\[
\left[ {T_{x_0 } } \right] = \left[ {\begin{array}{*{20}c}\displaystyle
    {\frac{{\partial f}}{{\partial x_1 }}\left( {x_0 } \right)} & \displaystyle {\frac{{\partial f}}{{\partial x_2 }}\left( {x_0 } \right)} &  \cdots  & \displaystyle {\frac{{\partial f}}{{\partial x_n }}\left( {x_0 } \right)}  \\
\end{array} } \right]_{1 \times n}
\]

\end{teo}

\begin{dem}
$f$ \'e diferenci\'avel em $x_0$, ent\~ao

\[
    \mathop {\lim }\limits_{h \to 0} \frac{{f\left( {x_0  + h} \right) - f\left( {x_0 } \right) - T_{x_0 } \left( h \right)}}{{\left\| h \right\|}} = 0
\]

Com $h = h_i e_i$, com $h_i \to 0$, temos

\[
\begin{gathered}
  \mathop {\lim }\limits_{h_i  \to 0} \frac{{f\left( {x_0  + h_i e_i } \right) - f\left( {x_0 } \right) - h_i T_{x_0 } \left( {e_i } \right)}}
{{h_i }} = 0 \hfill \\
   \Rightarrow \mathop {\lim }\limits_{h_i  \to 0} \frac{{f\left( {x_0  + h_i e_i } \right) - f\left( {x_0 } \right)}}
{{h_i }} - \mathop {\lim }\limits_{h_i  \to 0} \frac{{\bcancel{h_i} T_{x_0 } \left( {e_i } \right)}}{{\bcancel{h_i} }} = 0 \hfill \\
\Rightarrow \frac{{\partial f}}
{{\partial x_i }}\left( {x_0 } \right) - T_{x_0 } \left( {e_i } \right) = 0 \hfill \\
\Rightarrow T_{x_0 } \left( {e_i } \right) = \frac{{\partial f}}
{{\partial x_i }}\left( {x_0 } \right),i = 1,2,...,n \hfill \\
\end{gathered}
\]

\end{dem}

\textbf{Nota:} $T_{x_0 }$ \'e a diferencial de $f$ em $x_0$.

\begin{teo}
    Seja $f:A \subset \R^n \to \R$ diferenci\'avel em $x_0$. Ent\~ao, $f$ \'e cont\'inua em $x_0$.
\end{teo}

\begin{dem}
    Seja $E\left( h \right) = f\left( {x_0  + h} \right) - f\left( {x_0 } \right) - T_{x_0} \left( h \right)$, $f$ \'e diferenci\'avel em $x_0$, ent\~ao

\[
    \mathop {\lim }\limits_{h \to 0} \frac{{E\left( h \right)}}{{\left\| h \right\|}} = 0
\]

$T_{x_0}$ \'e cont\'inua

\[
\begin{gathered}
   \Rightarrow \mathop {\lim }\limits_{h \to 0} T_{x_0 } \left( h \right) = T\left( 0 \right) = 0 \hfill \\
  \mathop {\lim }\limits_{h \to 0} E\left( h \right) = \mathop {\lim }\limits_{h \to 0} \overbrace {\left\| h \right\|}^0\overbrace {\frac{{E\left( h \right)}}
{{\left\| h \right\|}}}^0 = 0 \hfill \\
  0 = \mathop {\lim }\limits_{h \to 0} E\left( h \right) = \mathop {\lim }\limits_{h \to 0} \left( {f\left( {x_0  + h} \right) - f\left( {x_0 } \right) - T_{x_0 } \left( h \right)} \right) \hfill \\
  0 = \mathop {\lim }\limits_{h \to 0} \left( {f\left( {x_0  + h} \right) - f\left( {x_0 } \right)} \right) - \mathop {\lim }\limits_{h \to 0} \overbrace {T_{x_0 } \left( h \right)}^0 \hfill \\
  \mathop {\lim }\limits_{h \to 0} f\left( {x_0  + h} \right) = f\left( {x_0 } \right) \hfill \\
\end{gathered}
\]

Portanto, $f$ \'e cont\'inua em $x_0$.
\end{dem}

\begin{ex}
Seja
\begin{equation*}
f(x,y)=\left\{ \begin{array}{cl}\displaystyle
        \frac{{x^3}}{{x^2 + y^2 }} & \textrm{se }\left( {x,y} \right) \ne \left( {0,0} \right)\\
        0 & \textrm{se }\left( {x,y} \right) = \left( {0,0} \right)\end{array}\right.
\end{equation*}

Esta fun\c c\~ao \'e diferenci\'avel em $(0,0)$?
\end{ex}

\begin{sol}
    $f$ \'e cont\'inua em $(0,0)$ (tem chance de ser diferenci\'avel, mas o teorema diz o contr\'ario.)

Ent\~ao, $f$ ser diferenci\'avel em $(0,0)$

$\Rightarrow \exists T_{x_0 } :\R^2  \to \R$ tal que

\[
    \mathop {\lim }\limits_{h \to 0} \frac{{f\left( {x_0  + h} \right) - f\left( {x_0 } \right) - T_{x_0 } \left( h \right)}}{{\left\| h \right\|}} = 0
\]

Melhorando, para $\R^2  \to \R$ temos

\[
    \mathop {\lim }\limits_{\left( {h,k} \right) \to \left( {0,0} \right)} \frac{{f\left( {\left( {0,0} \right) + \left( {h,k} \right)} \right) - f\left( {0,0} \right) - T_{\left( {0,0} \right)} \left( {h,k} \right)}}{{\left\| {\left( {h,k} \right)} \right\|}} = 0
\]

Se existe $T_{\left( {0,0} \right)}$, ela \'e \'unica e

\[
\left[ {T_{\left( {0,0} \right)} } \right] = \left[ {\begin{array}{*{20}c}
    {\frac{{\partial f}}{{\partial x}}\left( {0,0} \right)} & {\frac{{\partial f}}{{\partial y}}\left( {0,0} \right)}  \\
\end{array} } \right]
\]

Ent\~ao,

\[
\begin{gathered}
\frac{{\partial f}}
{{\partial x}}\left( {0,0} \right) = \mathop {\lim }\limits_{h \to 0} \frac{{\overbrace {f\left( {h,0} \right)}^h - f\left( {0,0} \right)}}
{h} = 1 \hfill \\
\frac{{\partial f}}
{{\partial y}}\left( {0,0} \right) = \mathop {\lim }\limits_{h \to 0} \frac{{\overbrace {f\left( {0,h} \right)}^0 - f\left( {0,0} \right)}}
{h} = 0 \hfill \\
   \Rightarrow T_{\left( {0,0} \right)} \left( {h,k} \right) = \left[ {T_{\left( {0,0} \right)} } \right]\left[ {\begin{array}{*{20}c}
h  \\
k  \\

\end{array} } \right] = \left[ {\begin{array}{*{20}c}
1 & 0  \\

\end{array} } \right]\left[ {\begin{array}{*{20}c}
h  \\
k  \\

\end{array} } \right] = h \hfill \\
\end{gathered}
\]

\[
\begin{gathered}
  \mathop {\lim }\limits_{\left( {h,k} \right) \to \left( {0,0} \right)} \frac{{f\left( {h,k} \right) - f\left( {0,0} \right) - h}}
{{\sqrt {h^2  + k^2 } }} =  \hfill \\
  \mathop {\lim }\limits_{\left( {h,k} \right) \to \left( {0,0} \right)} \frac{{\frac{{h^3 }}
{{h^2  + k^2 }} - 0 - h}}
{{\sqrt {h^2  + k^2 } }} = \mathop {\lim }\limits_{\left( {h,k} \right) \to \left( {0,0} \right)} \frac{{h^3  - h\left( {h^2  + k^2 } \right)}}
{{\left( {h^2  + k^2 } \right)^{{\raise0.5ex\hbox{$\scriptstyle 3$}
\kern-0.1em/\kern-0.15em
\lower0.25ex\hbox{$\scriptstyle 2$}}} }} = \mathop {\lim }\limits_{\left( {h,k} \right) \to \left( {0,0} \right)} \frac{{ - hk^2 }}
{{\left( {h^2  + k^2 } \right)^{{\raise0.5ex\hbox{$\scriptstyle 3$}
\kern-0.1em/\kern-0.15em
\lower0.25ex\hbox{$\scriptstyle 2$}}} }} \hfill \\
\end{gathered}
\]

Vamos chamar $\displaystyle \frac{{ - hk^2 }}
{{\left( {h^2  + k^2 } \right)^{{\raise0.5ex\hbox{$\scriptstyle 3$}
\kern-0.1em/\kern-0.15em
\lower0.25ex\hbox{$\scriptstyle 2$}}} }} = g\left( {h,k} \right)$

Escolhendo $\gamma \left( t \right) = \left( {t,t} \right)$, temos:

\[
\mathop {\lim }\limits_{t \to 0} g\left( {\gamma \left( t \right)} \right) = \mathop {\lim }\limits_{t \to 0} \frac{{ - t^3 }}
{{\left( {2t^2 } \right)^{{\raise0.5ex\hbox{$\scriptstyle 3$}
\kern-0.1em/\kern-0.15em
\lower0.25ex\hbox{$\scriptstyle 2$}}} }} = \mathop {\lim }\limits_{t \to 0} \frac{{ - 1}}
{{2\sqrt 2 }}\frac{{t^3 }}
{{\left| t \right|^3 }} = \nexists
\]

Logo, $f$ n\~ao \'e diferenci\'avel em $(0,0)$.
\end{sol}

\begin{teo}
    Seja $A \subset \R^n$, $x_0$ interior a $A$ e $f:A \to \R$. Se $\frac{{\partial f}}{{\partial x_i }}$ s\~ao cont\'inuas em $x_0$, ent\~ao $f$ \'e diferenci\'avel em $x_0$.
\end{teo}

\begin{dem}
    Como $A$ \'e aberto, existe uma bola aberta $B$ de centro $\left( {x_0 ,y_0 } \right)$, contida em $A$. Sejam $h$ e $k$ tais que $\left( {x_0  + h,y_0  + k} \right) \in B$. Temos

\[
\scriptstyle{f\left( {x_0  + h,y_0  + k} \right) - f\left( {x_0 ,y_0 } \right) = \underbrace {f\left( {x_0  + h,y_0  + k} \right) - f\left( {x_0 ,y_0  + k} \right)}_{\left( I \right)} + \underbrace {f\left( {x_0 ,y_0  + k} \right) - f\left( {x_0 ,y_0 } \right)}_{\left( {II} \right)}}
\]

    Fazendo $G\left( x \right) = f\left( {x,y_0  + k} \right)$, pelo TVM \ref{tvm} existe $\overline x$, entre $x_0$ e $x_0 + h$ tal que

\[
\left( I \right) = G\left( {x_0  + h} \right) - G\left( {x_0 } \right) = G'\left( {\overline x } \right)h = \frac{{\partial f}}
{{\partial x}}\left( {\overline x ,y_0  + k} \right)h
\]

Do mesmo modo, existe $\overline y$, entre $y_0$ e $y_0 + k$ tal que

\[
    \left( {II} \right) = \frac{{\partial f}}{{\partial y}}\left( {x_0 ,\overline y } \right)k
\]

Assim,

\[
f\left( {x_0  + h,y_0  + k} \right) - f\left( {x_0 ,y_0 } \right) = \frac{{\partial f}}
{{\partial x}}\left( {\overline x ,y_0  + k} \right)h + \frac{{\partial f}}
{{\partial y}}\left( {x_0 ,\overline y } \right)k
\]

    Subtraindo a ambos os membros da igualdade acima $\frac{{\partial f}}{{\partial x}}\left( {x_0 ,y_0 } \right)h + \frac{{\partial f}}{{\partial y}}\left( {x_0 ,y_0 } \right)k$ obtemos:

\[
\begin{gathered}
  f\left( {x_0  + h,y_0  + k} \right) - f\left( {x_0 ,y_0 } \right) - \frac{{\partial f}}
{{\partial x}}\left( {x_0 ,y_0 } \right)h - \frac{{\partial f}}
{{\partial y}}\left( {x_0 ,y_0 } \right)k =  \hfill \\
= \left[ {\frac{{\partial f}}
{{\partial x}}\left( {\overline x ,y_0  + k} \right) - \frac{{\partial f}}
{{\partial x}}\left( {x_0 ,y_0 } \right)} \right]h + \left[ {\frac{{\partial f}}
{{\partial y}}\left( {x_0 ,\overline y } \right) - \frac{{\partial f}}
{{\partial y}}\left( {x_0 ,y_0 } \right)} \right]k \hfill \\
\end{gathered}
\]

Segue que

\[
\begin{gathered}
  \left| {\frac{{f\left( {x_0  + h,y_0  + k} \right) - f\left( {x_0 ,y_0 } \right) - \frac{{\partial f}}
{{\partial x}}\left( {x_0 ,y_0 } \right)h - \frac{{\partial f}}
{{\partial y}}\left( {x_0 ,y_0 } \right)k}}
{{\left\| {\left( {h,k} \right)} \right\|}}} \right| \leqslant  \hfill \\
\underbrace { \leqslant \left| {\frac{{\partial f}}
{{\partial x}}\left( {\overline x ,y_0  + k} \right) - \frac{{\partial f}}
{{\partial x}}\left( {x_0 ,y_0 } \right)} \right|}_{\left( {III} \right)}\overbrace {\frac{{\left| h \right|}}
{{\sqrt {h^2  + k^2 } }}}^{{\text{limitada}}} +  \hfill \\
+ \underbrace {\left| {\frac{{\partial f}}
{{\partial y}}\left( {x_0 ,\overline y } \right) - \frac{{\partial f}}
{{\partial y}}\left( {x_0 ,y_0 } \right)} \right|}_{\left( {IV} \right)}\frac{{\left| h \right|}}
{{\sqrt {h^2  + k^2 } }} \hfill \\
\end{gathered}
\]

    Pela continuidade de $\frac{{\partial f}}{{\partial x}}$ e $\frac{{\partial f}}{{\partial y}}$ em $\left( {x_0 ,y_0 } \right)$, as express\~oes $(III)$ e $(IV)$ tendem a zero, quando $\left( {h,k} \right) \to \left( {0,0} \right)$, e, portanto,

\[
\mathop {\lim }\limits_{\left( {h,k} \right) \to \left( {0,0} \right)} \frac{{f\left( {x_0  + h,y_0  + k} \right) - f\left( {x_0 ,y_0 } \right) - \frac{{\partial f}}
{{\partial x}}\left( {x_0 ,y_0 } \right)h - \frac{{\partial f}}
{{\partial y}}\left( {x_0 ,y_0 } \right)k}}
{{\left\| {\left( {h,k} \right)} \right\|}} = 0
\]

logo, $f$ \'e diferenci\'avel em $\left( {x_0 ,y_0 } \right)$.

\end{dem}

\section{Espa\c co Tangente} \label{sec15}

\begin{defn}
Seja $f:A \subset \R^n \to \R$ diferenci\'avel em $x_0$:

\[
\begin{gathered}
  \mathop {\lim }\limits_{h \to 0} \frac{{f\left( {x_0  + h} \right) - f\left( {x_0 } \right) - T_{x_0 } \left( h \right)}}
{{\left\| h \right\|}} = 0 \hfill \\
   \Rightarrow \mathop {\lim }\limits_{x \to x_0 } \frac{{f\left( x \right) - f\left( {x_0 } \right) - T_{x_0 } \left( {x - x_0 } \right)}}
{{\left\| {x - x_0 } \right\|}} = 0 \hfill \\
   \Rightarrow \mathop {\lim }\limits_{x \to x_0 } \frac{{f\left( x \right) - f\left( {x_0 } \right) - \sum\limits_{i = 1}^n {\frac{{\partial f}}
{{\partial x_i }}\left( {x_0 } \right)\left( {x_i  - x_{0_i } } \right)} }}
{{\left\| {x - x_0 } \right\|}} = 0 \hfill \\
\end{gathered}
\]

\textbf{Nota:} Pode-se escrever, $\frac{{\partial f}}{{\partial x}} = f_x$.

    Ent\~ao, seja, $T\left( x \right) = f\left( {x_0 } \right) + \sum\limits_{i = 1}^n {f_{x_i } \left( {x_0 } \right)\left( {x_i  - x_{0_i } } \right)}$

Temos,  $E\left( x \right) = f\left( x \right) - T\left( x \right)$

    Portanto, $\displaystyle \mathop {\lim }\limits_{x \to x_0 } \frac{{E\left( x \right)}}{{\left\| {x - x_0 } \right\|}} = 0$

    Ent\~ao, $T:A \to \R$ \'e a "melhor" aproxima\c c\~ao afim de $f$ em torno de $x_0$.
\end{defn}

\begin{defn}
    Seja $A \subset \R^n$ e $f:A \to \R$ diferenci\'avel em $x_0$. O subespa\c co afim de $\R^{n + 1}$ dado por

\[
\begin{gathered}
  x_{n + 1}  - f\left( {x_0 } \right) = \sum\limits_{i = 1}^n {f_{x_i } \left( {x_0 } \right)\left( {x_i  - x_{0_i } } \right)}  \hfill \\
\boxed{x_{n + 1}  - f\left( {x_0 } \right) = \frac{{\partial f}}
{{\partial x_1 }}\left( {x_0 } \right)\left( {x_1  - x_{0_1 } } \right) + ... + \frac{{\partial f}}
{{\partial x_n }}\left( {x_0 } \right)\left( {x_n  - x_{0_n } } \right) \hfill }\\
\end{gathered}
\]

    \'e chamado \textit{espa\c co tangente} ao gr\'afico de $f$ em $\left( {x_0 ,f\left( {x_0 } \right)} \right),T_{\left( {x_0 ,f\left( {x_0 } \right)} \right)} {\text{graf }}f$.

    \textbf{Obs:} $\dim T_{\left( {x_0 ,f\left( {x_0 } \right)} \right)} {\text{graf }}f = n$.

A equa\c c\~ao para o plano tangente de $\R^2$ em $\R$ \'e

\[\boxed{
z - f\left( {x_0 ,y_0 } \right) = \frac{{\partial f}}
    {{\partial x}}\left( {x_0 ,y_0 } \right)\left( {x - x_0 } \right) + \frac{{\partial f}}
{{\partial y}}\left( {x_0 ,y_0 } \right)\left( {y - y_0 } \right)}
\]

    Existe uma dire\c c\~ao em $\R^{n + 1}$ ortogonal a $T_{\left( {x_0 ,f\left( {x_0 } \right)} \right)} {\text{graf }}f$, tamb\'em chamado de \textit{vetor normal}\index{Vetor normal}. Sua dire\c c\~ao \'e dada por

\[
    \overrightarrow n  = \left( {f_{x_1 } \left( {x_0 } \right),f_{x_2 } \left( {x_0 } \right),...,f_{x_n } \left( {x_0 } \right), - 1} \right) = \left( {\nabla f\left( {x_0 } \right), - 1} \right)
\]

Em $\R^2$ o vetor normal \'e

\[
\overrightarrow n  = \left( {\frac{{\partial f}}
{{\partial x}},\frac{{\partial f}}
    {{\partial y}}, - 1} \right) = \left( {\nabla f\left( {x_0 ,y_0 } \right), - 1} \right)
\]

    A \textit{reta normal}\index{Reta normal} ao gr\'afico de $f$ em $\left( {x_0 ,f\left( {x_0 } \right)} \right)$ ($\R^2$) \'e

\[
\begin{gathered}
      v = v_0  + t\left( {\nabla f\left( {x_0 ,y_0 } \right), - 1} \right) \hfill \\
      \boxed{r: \left( {x,y,z} \right) = \left( {x_0 ,y_0 ,f\left( {x_0 ,y_0 } \right)} \right) + t\left( {\frac{{\partial f}}
{{\partial x}}\left( {x_0 ,y_0 } \right),\frac{{\partial f}}
{{\partial y}}\left( {x_0 ,y_0 } \right), - 1} \right) \hfill} \\
\end{gathered}
\]

\end{defn}

\begin{ex}
Seja $f\left( {x,y} \right) = 3xy^2  - y$. Determine as equa\c c\~oes do plano tangente e da reta normal no ponto $(2,1)$.
\end{ex}

\newpage 

\begin{sol}
Plano tangente

\[
\begin{gathered}
z - f\left( {2,1} \right) = \frac{{\partial f}}
{{\partial x}}\left( {2,1} \right)\left( {x - 2} \right) + \frac{{\partial f}}
{{\partial y}}\left( {2,1} \right)\left( {y - 1} \right) \hfill \\
  \frac{{\partial f}}{{\partial x}}\left( {x,y} \right) = 3y^2 \Rightarrow \frac{{\partial f}}
{{\partial x}}\left( {2,1} \right) = 3 \hfill \\
\frac{{\partial f}}
{{\partial y}}\left( {x,y} \right) = 6xy - 1   \Rightarrow \frac{{\partial f}}
{{\partial y}}\left( {2,1} \right) = 11 \hfill \\
f\left( {2,1} \right) = 5 \hfill \\
\end{gathered}
\]

A equa\c c\~ao do plano tangente \'e

\[
z - 5 = 3\left( {x - 2} \right) + 11\left( {y - 1} \right)
\]

Reta normal

\[
r: \left( {x,y,z} \right) = \left( {2,1,f\left( {2,1} \right)} \right) + t\left( {\frac{{\partial f}}
{{\partial x}}\left( {2,1} \right),\frac{{\partial f}}
{{\partial y}}\left( {2,1} \right), - 1} \right),t \in \R
\]

ou seja

\[
    r:\left( {x,y,z} \right) = \left( {2,1,5} \right) + t\left( {3,11, - 1} \right)
\]

\end{sol}

\begin{ex}
Seja
\begin{equation*}
f(x,y)=\left\{ \begin{array}{cl}\displaystyle
        \frac{{xy^2}}{{x^2 + y^2 }} & \textrm{se }\left( {x,y} \right) \ne \left( {0,0} \right)\\
        0 & \textrm{se }\left( {x,y} \right) = \left( {0,0} \right)\end{array}\right.
\end{equation*}

    Mostre que o gr\'afico de $f$ n\~ao admite plano tangente em $\left( {0,0,f\left( {0,0} \right)} \right)$.
\end{ex}

\begin{sol}
    $f$ n\~ao \'e cont\'inua, pois $f$ n\~ao \'e diferenci\'avel em $\left( {0,0,f\left( {0,0} \right)} \right)$.

\[
\begin{gathered}
\frac{{\partial f}}
{{\partial x}}\left( {0,0} \right) = \mathop {\lim }\limits_{h \to 0} \frac{{f\left( {x,0} \right)h - f\left( {0,0} \right)}}
{h} = 0 \hfill \\
\frac{{\partial f}}
{{\partial y}}\left( {0,0} \right) = 0 \hfill \\
\end{gathered}
\]

Equa\c c\~ao do plano tangente

\[
\begin{gathered}
z - 0 = 0\left( {x - 0} \right) + 0\left( {y - 0} \right) \hfill \\
z = 0 \hfill \\
\end{gathered}
\]

    A curva $\gamma \left( t \right) = \left( {t,t,f\left( {t,t} \right)} \right)$ tem imagem em ${\text{graf}}\left( f \right)$ e

\[
\gamma '\left( t \right) = \left( {1,1,\frac{d}
{{dt}}f\left( {t,t} \right)} \right) = \left( {1,1,\frac{d}
{{dt}}\left( {\frac{{t^3 }}
{{2t^2 }}} \right)} \right) = \left( {1,1,{\raise0.5ex\hbox{$\scriptstyle 1$}
\kern-0.1em/\kern-0.15em
\lower0.25ex\hbox{$\scriptstyle 2$}}} \right)
\]

    em particular, $\gamma '\left( 0 \right) = \left( {1,1,{\raise0.5ex\hbox{$\scriptstyle 1$}
\kern-0.1em/\kern-0.15em
\lower0.25ex\hbox{$\scriptstyle 2$}}} \right) \notin \pi :z = 0$
\end{sol}

\section{Regra da Cadeia}
\label{sec16}
\index{Regra da cadeia}

Dois casos:

\begin{enumerate}
\item $\left. \begin{gathered}
f:A \subset \R^n  \to \R \hfill \\
\gamma :I \subset \R \to A \hfill \\
\end{gathered}  \right\}\left( {f \circ \gamma } \right):I \to \R$
\item $\left. \begin{gathered}
f:A \subset \R^n  \to \R \hfill \\
g:B \subset \R^n  \to A \hfill \\
\end{gathered}  \right\}\left( {f \circ g} \right):B \to \R$
\end{enumerate}

\textit{\textbf{Caso 1}}

\begin{lem} \label{lem01}
    Seja $f:A \subset \R^n  \to \R$ e $f$ diferenci\'avel em $x_0$. Existe $\varphi :A \to \R$ cont\'inua em $x_0$ tal que

\[
f\left( x \right) - f\left( {x_0 } \right) = \left\langle {\nabla f\left( {x_0 } \right),\left( {x - x_0 } \right)} \right\rangle  + \varphi \left( x \right).\left\| {x - x_0 } \right\|
\]
\end{lem}

\begin{dem}
$f$ \'e diferenci\'avel em $x_0$, ent\~ao

\[
f\left( x \right) - f\left( {x_0 } \right) = T_{x_0 } \left( {x - x_0 } \right) + E\left( x \right) = \left\langle {\nabla f\left( {x_0 } \right),\left( {x - x_0 } \right)} \right\rangle  + E\left( x \right)
\]

Defina

\begin{equation*}
\varphi(x)=\left\{ \begin{array}{cl}\displaystyle
\frac{{E(x)}}{{\left\| {x - x_0 } \right\|}} & \textrm{se }x \ne x_0\\
0 & \textrm{se }x=x_0\end{array}\right.
\end{equation*}

$\varphi$ \'e cont\'inua, pois

\[
\mathop {\lim }\limits_{x \to x_0 } \varphi \left( x \right) = \mathop {\lim }\limits_{x \to x_0 } \frac{{E\left( x \right)}}
{{\left\| {x - x_0 } \right\|}} = 0
\]

\end{dem}

\begin{teo}
    Sejam $f:A \subset \R^n \to \R$, $A$ aberto, $x_0 \in A$ e $\gamma: I \subset \R \to \R^n$ tais que $\gamma \left( I \right) \subset A$ e $\gamma \left( {t_0 } \right) = x_0$. Se $\gamma$ \'e diferenci\'avel em $t_0$ e $f$ \'e diferenci\'avel em $x_0  = \gamma \left( {t_0 } \right)$, temos:

\[
\left. {\frac{d}
{{dt}}f \circ \gamma } \right|_{t = t_0 }  = \left\langle {\nabla f\left( {x_0 } \right),\gamma '\left( {t_0 } \right)} \right\rangle
\]

\end{teo}

\begin{dem}
\[
f\left( x \right) - f\left( {x_0 } \right) = \left\langle {\nabla f\left( {x_0 } \right),x - x_0 } \right\rangle  + \varphi \left( x \right)\left\| {x - x_0 } \right\|
\]

Fazendo $x = \gamma \left( t \right)$, temos

\[
f\left( {\gamma \left( t \right)} \right) - f\left( {\gamma \left( {t_0 } \right)} \right) = \left\langle {\nabla f\left( {\gamma \left( {t_0 } \right)} \right),\gamma \left( t \right) - \gamma \left( {t_0 } \right)} \right\rangle  + \varphi \left( {\gamma \left( {t} \right)} \right)\left\| {\gamma \left( t \right) - \gamma \left( {t_0 } \right)} \right\|
\]

dividindo ambos os membros por $t - t_0$ e derivando, temos:

\[
\scriptstyle{
\left. {\mathop {\lim }\limits_{t \to t_0 } \frac{d}
{{dt}}f \circ \gamma } \right|_{t = t_0 }  = \left\langle {\nabla f\left( {x_0 } \right),\mathop {\lim }\limits_{t \to t_0 } \frac{{\gamma \left( t \right) - \gamma \left( {t_0 } \right)}}
{{t - t_0 }}} \right\rangle  + \mathop {\lim }\limits_{t \to t_0 } \varphi \left( {\gamma \left( t \right)} \right)\overbrace {\frac{{\left\| {\gamma \left( t \right) - \gamma \left( {t_0 } \right)} \right\|}}
{{t - t_0 }}}^0 = \left\langle {\nabla f\left( {x_0 } \right),\gamma '\left( {t_0 } \right)} \right\rangle
}
\]

\end{dem}

\begin{itemize}
\item Se $\gamma$ \'e curva de n\'ivel $c$ de $f$

\[
\begin{gathered}
\Rightarrow f\left( {\gamma \left( t \right)} \right) = c \hfill \\
\Rightarrow \frac{d}
{{dt}}f \circ \gamma  = 0 \hfill \\
   \Rightarrow \left\langle {\nabla f\left( {\gamma \left( t \right)} \right),\gamma '\left( t \right)} \right\rangle  = 0 \hfill \\
\end{gathered}
\]

\item Seja $u$ unit\'ario e $f$ diferenci\'avel

\[
\gamma \left( t \right) = x_0  + tu
\]

\[
\begin{gathered}
\left. {\frac{d}
{{dt}}f\left( {\gamma \left( t \right)} \right)} \right|_{t = 0}  = \left\langle {\nabla f\left( {x_0 } \right),u} \right\rangle  \hfill \\
\frac{{\partial f}}
{{\partial u}}\left( {x_0 } \right) = \left\langle {\nabla f\left( {x_0 } \right),u} \right\rangle  = \left\| {\nabla f\left( {x_0 } \right)} \right\|\cos \theta  \hfill \\
\end{gathered}
\]

    Lembrando a defini\c c\~ao de derivada direcional na se\c c\~ao \ref{sec13} p\'ag. \pageref{sec13}.

\end{itemize}

$f:A \subset \R^2  \to \R$ diferenci\'avel
$\gamma :I \subset \R \to A$ curva de n\'ivel (diferenci\'avel)

Defina

\[
\begin{gathered}
\left( {f \circ \gamma } \right)\left( t \right) = c \hfill \\
\frac{d}
{{dt}}\left( {f \circ \gamma } \right)\left( t \right) = \frac{d}
{{dt}}c \hfill \\
  \left\langle {\nabla f\left( {\gamma \left( t \right)} \right),\gamma '\left( t \right)} \right\rangle  = 0 \hfill \\
   \Rightarrow \nabla f\left( {\gamma \left( t \right)} \right) \bot \gamma '\left( t \right) \hfill \\
\end{gathered}
\]


\textit{\textbf{Caso 2}}

Seja $f: \R^n \to \R$ e $g: \R^n \to \R^n$ se $\operatorname{Im} g \subset D_f$, temos, $f \circ g:\R^n  \to \R$

\[
C^k \left( A \right) = \left\{ {f:A \to \R^n :\frac{{\partial ^{\left| \alpha  \right|} f}}
{{\partial x^\alpha  }}{\text{ \'e cont\'inua}}} \right\}
\]

$\alpha$ \'e multi-\'indice de ordem $k$.

$\alpha  \in \mathbb{Z}_ + ^n ,\alpha  = \left( {\alpha _1 ,\alpha _2 ,...,\alpha _n } \right)$ \'e multi-\'indice de ordem $k$ se $\sum\limits_{i = 1}^n {\alpha _i }  = k$

\begin{ex}
$\alpha  \in \mathbb{Z}_ + ^3$ e $f:\R^3 \to \R$

\[
\begin{gathered}
\alpha  = \left( {1,0,2} \right) \hfill \\
\frac{{\partial ^{\left| \alpha  \right|} f}}
{{\partial x^\alpha  }} = \frac{{\partial ^3 f}}
{{\partial x\partial z^2 }} \hfill \\
\end{gathered}
\]

\end{ex}

\begin{ex}
$c^0  = \left\{ {f:A \to \R^n :f{\text{ \'e  cont\'inua}}} \right\}$

\end{ex}

\begin{teo}
    Sejam $f:A \subset \R^2 \to \R$ e $g:B \subset \R^2 \to \R^2$ fun\c c\~oes com $A$ e $B$ abertos e $g(B) \subset A$.

    Se $g\left( {u,v} \right) = \left( {g_1 \left( {u,v} \right),g_2 \left( {u,v} \right)} \right)$ e $f(x,y)$ s\~ao de classe $C^1$, ent\~ao, $\left( {f \circ g} \right)\left( {u,v} \right)$ \'e de classe $C^1$.

\[
\begin{gathered}
\frac{\partial }
{{\partial u}}\left( {f \circ g} \right)\left( {u,v} \right) = \frac{{\partial f}}
{{\partial x}}\left( {g\left( {u,v} \right)} \right)\frac{{\partial g_1 }}
{{\partial u}}\left( {u,v} \right) + \frac{{\partial f}}
{{\partial y}}\left( {g\left( {u,v} \right)} \right)\frac{{\partial g_2 }}
{{\partial u}}\left( {u,v} \right) \hfill \\
\frac{\partial }
{{\partial v}}\left( {f \circ g} \right)\left( {u,v} \right) = \frac{{\partial f}}
{{\partial x}}\left( {g\left( {u,v} \right)} \right)\frac{{\partial g_1 }}
{{\partial v}}\left( {u,v} \right) + \frac{{\partial f}}
{{\partial y}}\left( {g\left( {u,v} \right)} \right)\frac{{\partial g_2 }}
{{\partial v}}\left( {u,v} \right) \hfill \\
\end{gathered}
\]

Lembrando que,

\[
\left[ {\begin{array}{*{20}c}
{f_u } & {f_v }  \\

 \end{array} } \right] = \nabla f.dg = \left( {f_x ,f_y } \right)\left[ {\begin{array}{*{20}c}
{g_1 u} & {g_1 v}  \\
{g_2 u} & {g_2 v}  \\

\end{array} } \right]
\]

\end{teo}

\begin{dem}

% figura 34

$\frac{{\partial f}}{{\partial u}}$ basta "congelar" $v$.

Seja $v$ constante, ent\~ao, $g\left( {u,v_0 } \right)$ d\'a uma curva em $A$.

\begin{eqnarray*}
\frac{\partial }
{{\partial u}}\left( {f \circ g} \right)\left( {u,v_0 } \right) &=& \left\langle {\nabla f\left( {g\left( {u,v_0 } \right)} \right),\left( {\frac{{\partial g_1 }}
{{\partial u}},\frac{{\partial g_2 }}
{{\partial u}}} \right)} \right\rangle  \hfill \\
&=& \frac{{\partial f}}
{{\partial x}}\left( {g\left( {u,v_0 } \right)} \right)\frac{{\partial g_1 }}
{{\partial u}}\left( {u,v_0 } \right) + \frac{{\partial f}}
{{\partial y}}\left( {g\left( {u,v_0 } \right)} \right)\frac{{\partial g_2 }}
{{\partial u}}\left( {u,v_0 } \right) \hfill \\
\frac{\partial }
{{\partial v}}\left( {f \circ g} \right)\left( {u_0 ,v} \right) &=& \left\langle {\nabla f\left( {g\left( {u_0 ,v} \right)} \right),\left( {\frac{{\partial g_1 }}
{{\partial v}},\frac{{\partial g_2 }}
{{\partial v}}} \right)} \right\rangle  \hfill \\
&=& \frac{{\partial f}}
{{\partial x}}\left( {g\left( {u_0 ,v} \right)} \right)\frac{{\partial g_1 }}
{{\partial v}}\left( {u_0 ,v} \right) + \frac{{\partial f}}
{{\partial y}}\left( {g\left( {u_0 ,v} \right)} \right)\frac{{\partial g_2 }}
{{\partial v}}\left( {u_0 ,v} \right) \hfill \\
\end{eqnarray*}

\end{dem}

\textbf{Exerc\'icio}

\begin{exerc}
Escreva essas f\'ormulas para $A,B \subset \R^n$.
\end{exerc}

\begin{ex}
Seja $f$ de classe $C^1$. Defina $z\left( {u,v} \right) = f\left( {\underbrace {u^2  + v^2 }_x,\underbrace {uv}_y} \right)$.

Calcule $\frac{{\partial z}}{{\partial u}}$ e $\frac{{\partial z}}{{\partial v}}$.
\end{ex}

\begin{sol}
    Note que $g\left( {u,v} \right) = \left( {u^2  + v^2 ,uv} \right)$ e $z = \left( {f \circ g} \right)\left( {u,v} \right)$.

\[
\begin{gathered}
\frac{{\partial z}}
{{\partial u}} = \frac{\partial }
{{\partial u}}f \circ g = f_x \left( {u^2  + v^2 ,uv} \right)2u + f_y \left( {u^2  + v^2 ,uv} \right)v \hfill \\
\frac{{\partial z}}
{{\partial v}} = \frac{\partial }
{{\partial v}}f \circ g = f_x \left( {u^2  + v^2 ,uv} \right)2v + f_y \left( {u^2  + v^2 ,uv} \right)u \hfill \\
\end{gathered}
\]

\end{sol}

\begin{ex}
    Sejam $f:\R^3  \to \R$ e $g:\R^3  \to \R$ fun\c c\~oes diferenci\'aveis. Como achar o vetor tangente a intersecc\c c\~ao de duas superf\'icies de n\'ivel de $f$ e $g$?
\end{ex}

\begin{sol}
% figura 35

    $\gamma ' \bot \nabla f$ e $\gamma ' \bot \nabla g \Rightarrow \gamma '\parallel \nabla f \times \nabla g$

$\gamma '$ \'e paralelo ao produto vetorial entre $\nabla f$ e $\nabla g$.

Por exemplo, seja

$f\left( {x,y,z} \right) = x^2  + y^2  + z^2$ (n\'ivel 2)
$g\left( {x,y,z} \right) = x^2  + y^2  - z^2$ (n\'ivel 0)

% figura 36

\[
\begin{gathered}
\gamma \left( t \right) = \left( {\cos t,\sin t, \pm 1} \right) \hfill \\
\gamma '\left( t \right) = \left( { - \sin t,\cos t,0} \right) \hfill \\
\gamma ' = \nabla f \times \nabla g \hfill \\
  \gamma ' = \left( {2x,2y,2z} \right) \times \left( {2x,2y, - 2z} \right) \hfill \\
\gamma ' = \left| {\begin{array}{*{20}c}
i & j & k  \\
{2x} & {2y} & {2z}  \\
{2x} & {2y} & { - 2z}  \\

\end{array} } \right| = \left( { - 8yz,8xz,0} \right) \hfill \\
\Rightarrow \gamma '\parallel \left( { - yz,xz,0} \right) \hfill \\
\end{gathered}
\]

\end{sol}

\begin{teo}[de Schwarz] \label{t5}
    Seja $F: A \to \R,A \subset \R^n$ aberto. Se $f$ \'e de classe $C^2$, ent\~ao,

\[
\frac{{\partial ^2 f}}
{{\partial x_i \partial x_j }} = \frac{{\partial ^2 f}}
{{\partial x_j \partial x_i }}
\]

\end{teo}

\section{Teorema da Fun\c c\~ao Impl\'icita} \label{sec17}
\index{Teorema!da fun\c c\~ao impl\'icita}

\begin{defn}
Seja $F:\R^{n + 1}  \to \R$ uma fun\c c\~ao. Dizemos que $g:\R^n  \to \R$ \'e dada implicitamente por $F$ se

\[
F\left( {x,g\left( x \right)} \right) = 0,\forall x \in D_g
\]
\end{defn}

Suponha $g:\R^n  \to \R$ dada implicitamente por $F:\R^{n + 1}  \to \R$ e $g$ e $F$ s\~ao diferenci\'aveis.

\[
\begin{gathered}
   \Rightarrow F\underbrace {\left( {x,g\left( x \right)} \right)}_{\left( {x_1 ,...,x_{n + 1} } \right)} = 0 \hfill \\
x = \left( {x_1 ,...,x_n } \right) \hfill \\
\end{gathered}
\]

Obtemos $\frac{{\partial g}}{{\partial x_i }}$ da seguinte forma:

\[
F\left( {x_1 ,x_2 ,...,x_n ,g\left( {x_1 ,...,x_n } \right)} \right) = 0
\]

Aplicando $\frac{\partial}{{\partial x_i }}$ dos dois lados.

\[
\begin{gathered}
\frac{{\partial F}}
{{\partial x_i }}\left( {x,g\left( x \right)} \right) = \frac{\partial }
{{\partial x_i }}.0 = 0 \hfill \\
\sum\limits_{j = 1}^n {\frac{{\partial F}}
{{\partial x_j }}\overbrace {\frac{{\partial x_j }}
{{\partial x_i }}}^0 + \frac{{\partial F}}
{{\partial x_n }}\frac{{\partial g}}
{{\partial x_i }}}  = 0 \hfill \\
\frac{{\partial F}}
{{\partial x_i }} + \frac{{\partial F}}
{{\partial x_n }}\frac{{\partial g}}
{{\partial x_i }} = 0 \hfill \\
\boxed{\displaystyle \frac{{\partial g}}
{{\partial x_i }} = \displaystyle \frac{{\displaystyle - \frac{{\partial F}}
{{\partial x_i }}}}
{{\displaystyle \frac{{\partial F}}
{{\partial x_n }}}} \hfill} \\
\end{gathered}
\]

\begin{ex}
Seja

\[
\begin{gathered}
F:\R^2  \to \R \hfill \\
F\left( {x,y} \right) = y^3  + xy + x^3  - 3 \hfill \\
\end{gathered}
\]

    sup\~oe que $g(x)$ tal que $F\left( {x,\underbrace {g\left( x \right)}_y} \right) = 0$ \'e diferenci\'avel. Calcule $g'\left( x \right)$.
\end{ex}

\begin{sol}
\[
\begin{gathered}
g'\left( x \right) = \frac{{ - F_x }}
{{F_y }} \hfill \\
\left\{ \begin{gathered}
F_x  = y + 3x^2  \hfill \\
F_y  = 3y^2  + x \hfill \\
\end{gathered}  \right. \hfill \\
\Rightarrow g'\left( x \right) = \frac{{ - \left( {y + 3x^2 } \right)}}
{{3y^2  + x}} = \frac{{ - \left( {g\left( x \right) + 3x^2 } \right)}}
{{3g^2 \left( x \right) + x}} \hfill \\
\end{gathered}
\]

\end{sol}

\begin{ex}
    Seja $F:\R^3  \to \R$ e $z:\R^2  \to \R$ tal que $F\left( {x,y,z\left( {x,y} \right)} \right) = 0$. Calcule $\frac{{\partial z}}{{\partial x}}$ e $\frac{{\partial z}}{{\partial y}}$.
\end{ex}

\begin{sol}
\[
\begin{array}{*{20}c}
{\frac{{\partial z}}
{{\partial x}} = \frac{{ - F_x }}
{{F_z }}} & {\frac{{\partial z}}
{{\partial y}} = \frac{{ - F_y }}
{{F_z }}}  \\

\end{array}
\]

\end{sol}

\begin{ex}
    Sejam $F,G:\R^3  \to \R$ e $y,z:\R \to \R$ dadas implicitamente por:

\[
\left\{ \begin{gathered}
F\left( {x,y\left( x \right),z\left( x \right)} \right) = 0 \hfill \\
F\left( {x,y\left( x \right),z\left( x \right)} \right) = 0 \hfill \\
\end{gathered}  \right.
\]

Calcule $y'$ e $z'$.
\end{ex}

\begin{sol}
\[
\begin{gathered}
\left\{ \begin{gathered}
F_x  + F_y .y' + F_z .z' = 0 \hfill \\
G_x  + G_y .y' + G_z .z' = 0 \hfill \\
\end{gathered}  \right. \hfill \\
\left\{ \begin{gathered}
F_y .y' + F_z .z' =  - F_x  \hfill \\
G_y .y' + G_z .z' =  - G_x  \hfill \\
\end{gathered}  \right. \hfill \\
\end{gathered}
\]

tem solu\c c\~ao \'unica se, e somente se,

\[
\det \left( {\begin{array}{*{20}c}
{F_y } & {F_z }  \\
{G_y } & {G_z }  \\

\end{array} } \right) \ne 0
\]

se vale isso, ent\~ao, por Cramer, temos:

\[
\begin{array}{*{20}c}
{y' = \frac{{\left| {\begin{array}{*{20}c}
{ - F_x } & {F_z }  \\
{ - G_x } & {G_z }  \\

\end{array} } \right|}}
{{\left| {\begin{array}{*{20}c}
{F_y } & {F_z }  \\
{G_y } & {G_z }  \\

\end{array} } \right|}}} & {z' = \frac{{\left| {\begin{array}{*{20}c}
{F_y } & { - F_x }  \\
{G_y } & { - G_x }  \\

\end{array} } \right|}}
{{\left| {\begin{array}{*{20}c}
{F_y } & {F_z }  \\
{G_y } & {G_z }  \\

\end{array} } \right|}}}  \\

\end{array}
\]

\end{sol}

\begin{teo}
    Seja $F:\R^{n - 1}  \times \R \to \R$ de classe $C^1$ e seja $x_0  = \left( {x_{0_1 } ,x_{0_2 } ,...,x_{0_{n - 1} } } \right) \in \R^{n - 1}$ e $y_0 \in \R$ tal que $F\left( {x_0 ,y_0 } \right) = 0$. Se $\frac{{\partial F}}{{\partial x_n }}\left( {x_0 ,y_0 } \right) \ne 0$ ent\~ao existe abertos $A \subset \R^{n - 1}$ e $B \subset \R$ com $x_0 \in A$ e $y_0 \in B$ tais que para cada $x \in A$ existe um \'unico $y = g(x)$ tal que $F\left( {x,y} \right) = 0$ em $A \times B$.

A fun\c c\~ao $g(x)$ \'e diferenci\'avel em $A$ e

\[\displaystyle
\frac{{\partial g}}{{\partial x_i }}\left( x \right) = \displaystyle \frac{{\displaystyle - \frac{{\partial F}}{{\partial x_i }}\left( {x,y} \right)}}
{{\displaystyle \frac{{\partial F}}{{\partial x_n }}\left( {x,y} \right)}}
\]

onde, $y = g\left( x \right)$.
\end{teo}

\begin{dem}
Sem perda de generalidade, fa\c camos $n = 2$.

Suponhamos $\frac{{\partial F}}{{\partial y}}\left( {x_0 ,y_0 } \right) > 0$.

    $F$ de classe $C^1$: $\frac{{\partial F}}{{\partial x}}$ e $\frac{{\partial F}}{{\partial y}}$ s\~ao cont\'inuas, ent\~ao, $F$ \'e diferenci\'avel, portanto, $F$ \'e cont\'inua.

$\frac{{\partial F}}
{{\partial y}}\left( {x_0 ,y_0 } \right) > 0 \Rightarrow \exists D = B_\varepsilon  \left( {x_0 ,y_0 } \right) \subset \R^2$ tal que $\frac{{\partial F}}
{{\partial y}}\left( {x,y} \right) > 0,\forall \left( {x,y} \right) \in D$.

Para $y_1$ e $y_2$ tal que $y_1 < y_0 < y_2$ temos,

\[
F\left( {x_0 ,y_1 } \right) < 0{\text{ e }}F\left( {x_0 ,y_2 } \right) > 0\left( 1 \right)
\]

onde $\left( {x_0 ,y_1 } \right){\text{e}}\left( {x_0 ,y_2 } \right) \in D$

$F\left( {x_0 ,y} \right)$ \'e crescente em $\left[ {y_1 ,y_2 } \right]$.

Seja $B = \left] {y_1 ,y_2 } \right[$. Note que $y_0$ \'e o \'unico ponto onde $F\left( {x_0 ,y} \right)$ se anula.

De $(1)$ e da continuidade de $F$ temos que, existe $A$ (aberto) $\subset \R$, $x_0 \in A$ tal que para $x \in A$ e $\left( {x,y_1 } \right),\left( {x,y_2 } \right) \in D$ temos $F\left( {x,y_1 } \right) < 0$ e $F\left( {x,y_2 } \right) > 0$.

$F$ cont\'inua em $D$ implica $\exists y \in B$ tal que $F\left( {x,y} \right) = 0$, tal $y$ \'e \'unico, pois $\frac{{\partial f}}
{{\partial y}} > 0$ em $D$, implica que, $F\left( {x,y} \right)$ \'e crescente para cada $x \in A$ fixado.

Logo, $x \mapsto y$. Defina $y = g\left( x \right),g:A \to B$.

\

Continuidade de $g$.

\

Para cada par $\left( {x,g\left( x \right)} \right)$ em $A \times B$, temos $F\left( {x,g\left( x \right)} \right) = 0$ e $\frac{{\partial f}}
{{\partial y}}\left( {x,g\left( x \right)} \right) > 0$, ent\~ao dados $\overline {y_1}$ e $\overline {y_2}$ com $y_1  < \overline {y_1 }  < g\left( x \right) < \overline {y_2 }  < y_2$ temos, repetindo o argumento, fazendo $x = x_0$ e $g(x) = y_0$, temos que existe $A_1 \subset A, x \in A$, tal que $\overline x  \in A$, temos, $g\left( {\overline x } \right) \in \left] {\overline {y_1 } ,\overline {y_2 } } \right[ \Rightarrow g\left( {A_1 } \right) \subset \left] {\overline {y_1 } ,\overline {y_2 } } \right[$, isto implica que $g$ \'e cont\'inua $\forall x \in A$.

\

Por hip\'otese, $F$ \'e diferenci\'avel, pelo Lema \ref{lem01} p\'ag. \pageref{lem01}, temos,

\[
F\left( {x,y} \right) = F\left( {x_0 ,y_0 } \right) + \left\langle {\nabla F\left( {x_0 ,y_0 } \right),\left( {x - x_0 ,y - y_0 } \right)} \right\rangle  + \varphi \left( {x,y} \right)\left\| {\left( {x,y} \right) - \left( {x_0 ,y_0 } \right)} \right\|
\]

$\varphi$ cont\'inua em $\left( {x_0 ,y_0 } \right) = \left( {x_0 ,g\left( {x_0 } \right)} \right)$

multiplicando $\varphi \left( {x,y} \right)\left\| {\left( {x,y} \right) - \left( {x_0 ,y_0 } \right)} \right\|$ por $\frac{{\left\| {\left( {x,y} \right) - \left( {x_0 ,y_0 } \right)} \right\|}}
{{\left\| {\left( {x,y} \right) - \left( {x_0 ,y_0 } \right)} \right\|}}$, obtemos

\[
\scriptstyle{
\varphi \left( {x,y} \right)\left\| {\left( {x,y} \right) - \left( {x_0 ,y_0 } \right)} \right\| =
\scriptstyle{\underbrace {\scriptstyle{\varphi \left( {x,y} \right)\frac{{\left( {x - x_0 } \right)}}
{{\left\| {\left( {x,y} \right) - \left( {x_0 ,y_0 } \right)} \right\|}}\left( {x - x_0 } \right)}}_{\varphi _1 \left( {x,y} \right)}} + 
\scriptstyle{\underbrace {\scriptstyle{\varphi \left( {x,y} \right)\frac{{\left( {y - y_0 } \right)}}
{{\left\| {\left( {x,y} \right) - \left( {x_0 ,y_0 } \right)} \right\|}}\left( {y - y_0 } \right)}}_{\varphi _2 \left( {x,y} \right)}}
}
\]

Ent\~ao,

\[
\scriptstyle{
F\left( {x,y} \right) = F\left( {x_0 ,y_0 } \right) + F_x \left( {x_0 ,y_0 } \right)\left( {x - x_0 } \right) + F_y \left( {x_0 ,y_0 } \right)\left( {y - y_0 } \right) + \varphi _1 \left( {x,y} \right)\left( {x - x_0 } \right) + \varphi _2 \left( {x,y} \right)\left( {y - y_0 } \right)
}
\]

onde, $y = g\left( {x} \right)$, $y_0  = g\left( {x_0 } \right)$, $F_x  = \frac{{\partial F}}{{\partial x}}$ e $F_y  = \frac{{\partial F}}{{\partial y}}$.

Ent\~ao,

\[
\scriptstyle{
0 = 0 + F_x \left( {x_0 ,g\left( {x_0 } \right)} \right)\left( {x - x_0 } \right) + F_y \left( {x_0 ,g\left( {x_0 } \right)} \right)\left( {g\left( x \right) - g\left( {x_0 } \right)} \right) + \varphi _1 \left( {x,g\left( x \right)} \right)\left( {x - x_0 } \right) + \varphi _2 \left( {x,g\left( x \right)} \right)\left( {g\left( x \right) - g\left( {x_0 } \right)} \right)
}
\]

Fazendo $x \to x_0 ,\varphi _1  \to 0,\varphi _2  \to 0$

\[
\frac{{dg}}
{{dx}}\left( {x_0 } \right) = g'\left( {x_0 } \right) = \frac{{ - \frac{{\partial F}}
{{\partial x}}\left( {x_0 ,g\left( {x_0 } \right)} \right)}}
{{\frac{{\partial F}}
{{\partial y}}\left( {x_0 ,g\left( {x_0 } \right)} \right)}},\forall x_0  \in A
\]

\end{dem}

\begin{ex}
    A equa\c c\~ao $y^3  + xy + x^3  = 4$ define uma fun\c c\~ao diferenci\'avel $y(x)$? Se sim, quem \'e $y'(x)$?
\end{ex}

\begin{sol}
\[
\begin{gathered}
F\left( {x,y} \right) = y^3  + xy + x^3  - 4 = 0 \hfill \\
x_0  = 0 \Rightarrow F\left( {0,y_0 } \right) = y_0^3  - 4 = 0 \hfill \\
\Rightarrow y_0  = \sqrt[3]{4} \hfill \\
\end{gathered}
\]

$\left( {x_0 ,y_0 } \right) = \left( {0,\sqrt[3]{4}} \right)$ \'e solu\c c\~ao de $F\left( {x,y} \right) = 0$.

\[
\begin{gathered}
\frac{{\partial F}}
{{\partial y}}\left( {x,y} \right) = 3y^2  + x \hfill \\
\frac{{\partial F}}
{{\partial y}}\left( {0,\sqrt[3]{4}} \right) \ne 0 \Rightarrow \exists A \subset \R \hfill \\
\end{gathered}
\]

$0 \in A$ e $B \subset \R,\sqrt[3]{4} \in B$ e $g:A \to B$ diferenci\'avel tais que $g\left( 0 \right) = \sqrt[3]{4}$ e

\[
g'\left( x \right) = \frac{{ - F_x \left( {x,g\left( x \right)} \right)}}
{{F_y \left( {x,g\left( x \right)} \right)}} = \frac{{ - \left( {3x^2  + g\left( x \right)} \right)}}
{{x + 3g^2 \left( x \right)}}
\]

\end{sol}

\begin{ex}
Seja $x^2  + y^2  = 1$
\end{ex}

\begin{sol}
O ponto $(1,0)$ resolve

\[
\begin{gathered}
F\left( {x,y} \right) = x^2  + y^2  - 1 \hfill \\
F_x  = 2x \Rightarrow F_x \left( {1,0} \right) = 2 \ne 0 \hfill \\
F_y  = 2y \Rightarrow F_y \left( {1,0} \right) = 0 \hfill \\
\end{gathered}
\]

% figura 37

\[
\begin{gathered}
x\left( y \right) = \sqrt {1 - y^2 }  \hfill \\
\frac{{dx}}
{{dy}} = \frac{{ - y}}
{{\sqrt {1 - y^2 } }} \hfill \\
\frac{{dx}}
{{dy}} = \frac{{ - F_y }}
{{F_x }} = \frac{{ - 2y}}
{{2x}} = \frac{{ - y}}
{{\sqrt {1 - y^2 } }} \hfill \\
\end{gathered}
\]

\end{sol}

\begin{defn}[Jacobiano]
Sejam $f_1 ,f_2 ,...,f_n :\R^m  \to \R$ de classe $C^1$. Definimos o \textit{Jacobiano}\index{Jacobiano} de $f_1 ,f_2 ,...,f_n$ em rela\c c\~ao \`as vari\'aveis $\left( {x_{i1 } ,x_{i2 } ,...,x_{in } } \right),$\\
$i = 1,2,...,n$ por

\[
\frac{{\partial \left( {f_1 ,f_2 ,...,f_n } \right)}}
{{\partial \left( {x_{i_1 } ,x_{i_2 } ,...,x_{i_n } } \right)}} = \det \left( {\begin{array}{*{20}c}
   \displaystyle {\frac{{\partial f_1 }}{{\partial x_{i1} }}} & \displaystyle {\frac{{\partial f_1 }}{{\partial x_{i2} }}} &  \cdots  & \displaystyle {\frac{{\partial f_1 }}{{\partial x_{in} }}}  \\
   \displaystyle {\frac{{\partial f_2 }}{{\partial x_{i1} }}} & \displaystyle {\frac{{\partial f_2 }}{{\partial x_{i2} }}} &  \cdots  & \displaystyle {\frac{{\partial f_2 }}{{\partial x_{in} }}}  \\
\vdots  &  \vdots  &  \ddots  &  \vdots   \\
   \displaystyle {\frac{{\partial f_n }}{{\partial x_{i1} }}} & \displaystyle {\frac{{\partial f_n }}{{\partial x_{i2} }}} &  \cdots  & \displaystyle {\frac{{\partial f_n }}{{\partial x_{in} }}}  \\

\end{array} } \right)
\]

\end{defn}

\begin{teo} \label{t4}
    Sejam $F,G:\R^3 \to \R$ de classe $C^1$ e $\left( {x_0 ,y_0 ,z_0 } \right) \in \R^3$ tal que $F\left( {x_0 ,y_0 ,z_0 } \right) = G\left( {x_0 ,y_0 ,z_0 } \right) = 0$. Nessas condi\c c\~oes se $\frac{{\partial \left( {F,G} \right)}}
{{\partial \left( {y,z} \right)}}\left( {x_0 ,y_0 ,z_0 } \right) \ne 0$ existe $I \subset \R$, intervalo aberto com $x_0 \in I$ e $y,z: I \to \R$ de classe $C^1$ tais que $F\left( {x,y\left( x \right),z\left( x \right)} \right) = 0 = G\left( {x,y\left( x \right),z\left( x \right)} \right)$ com $y\left( {x_0 } \right) = y_0$ e $z\left( {x_0 } \right) = z_0$. Al\'em disso

\[
\begin{array}{*{20}c}
   {y'\left( x \right) = \frac{{\displaystyle - \frac{{\partial \left( {F,G} \right)}}
{{\partial \left( {x,z} \right)}}}}
{{\displaystyle \frac{{\partial \left( {F,G} \right)}}
{{\partial \left( {y,z} \right)}}}}} & {z'\left( x \right) = \frac{{\displaystyle - \frac{{\partial \left( {F,G} \right)}}
{{\partial \left( {x,y} \right)}}}}
{{\displaystyle \frac{{\partial \left( {F,G} \right)}}
{{\partial \left( {y,z} \right)}}}}}  \\

\end{array}
\]

\end{teo}

\begin{dem}
Como $F,G$ s\~ao de classe $C^1$, ent\~ao,

\[
\frac{{\partial \left( {F,G} \right)}}
{{\partial \left( {y,z} \right)}} = \left| {\begin{array}{*{20}c}
{F_y } & {F_z }  \\
{G_y } & {G_z }  \\

\end{array} } \right|
\]

\'e cont\'inua.

Como $\frac{{\partial \left( {F,G} \right)}}
{{\partial \left( {y,z} \right)}}\left( {x_0 ,y_0 ,z_0 } \right) \ne 0$ existe $A \subset \R^3$ aberto, $\left( {x_0 ,y_0 ,z_0 } \right) \in A$ tal que

\[
\begin{gathered}
\frac{{\partial \left( {F,G} \right)}}
{{\partial \left( {y,z} \right)}}\left( {x,y,z} \right) \ne 0,\forall \left( {x,y,z} \right) \in A \hfill \\
\frac{{\partial \left( {F,G} \right)}}
{{\partial \left( {y,z} \right)}}\left( {x_0 ,y_0 ,z_0 } \right) \ne 0 \hfill \\
   \Rightarrow F_y \left( {x_0 ,y_0 ,z_0 } \right) \ne 0{\text{ ou }}F_z \left( {x_0 ,y_0 ,z_0 } \right) \ne 0 \hfill \\
\end{gathered}
\]

    Suponhamos $F_z \left( {x_0 ,y_0 ,z_0 } \right) \ne 0$, pelo Teorema \ref{t4}, existem $B \subset \R^2$ aberto e $g:B \to \R$ de classe $C^1$ tal que $F\left( {x,y,g\left( {x,y} \right)} \right) = 0,\forall \left( {x,y} \right) \in B$.

    Seja $H\left( {x,y} \right) = G\left( {x,y,g\left( {x,y} \right)} \right)$, $H$ \'e de classe $C^1$.

\[
\begin{gathered}
  H\left( {x_0 ,y_0 } \right) = G\left( {x_0 ,y_0 ,g\left( {x_0 ,y_0 } \right)} \right) = 0 \hfill \\
H_y  = G_y  + G_z \frac{{\partial g}}
{{\partial y}} \hfill \\
  H_y \left( {x_0 ,y_0 } \right) = G_y \left( {x_0 ,y_0 ,z_0 } \right) + G_z \left( {x_0 ,y_0 ,z_0 } \right)\frac{{\partial g}}
{{\partial y}}\left( {x_0 ,y_0 } \right) \ne 0 \hfill \\
  H_y \left( {x_0 ,y_0 } \right) = G_y \left( {x_0 ,y_0 ,z_0 } \right) + G_z \left( {x_0 ,y_0 ,z_0 } \right)\left( {\frac{{ - \frac{{\partial F}}
{{\partial y}}\left( {x_0 ,y_0 ,z_0 } \right)}}
{{\frac{{\partial F}}
{{\partial z}}\left( {x_0 ,y_0 ,z_0 } \right)}}} \right) \hfill \\
\end{gathered}
\]

    O Teorema da Fun\c c\~ao Impl\'icita para $H$ diz, $\exists I \subset \R, x_0 \in I$ e $k: I \to \R$ de classe $C^1$ tal que $H\left( {x,h\left( x \right)} \right) = 0,\forall x \in I$.

    Ent\~ao, $y\left( x \right) = h\left( x \right)$ e $z\left( x \right) = g\left( {x,h\left( x \right)} \right)$ s\~ao definidas em $I$ implicitamente por

\[
\left\{ \begin{gathered}
F\left( {x,y,z} \right) = 0 \hfill \\
G\left( {x,y,z} \right) = 0 \hfill \\
\end{gathered}  \right.
\]

As express\~oes das derivadas s\~ao

\[
\begin{gathered}
\left\{ \begin{gathered}
F_x  + F_y .y' + F_z .z' = 0 \hfill \\
G_x  + G_y .y' + G_z .z' = 0 \hfill \\
\end{gathered}  \right. \hfill \\
\left\{ \begin{gathered}
F_y .y' + F_z .z' =  - F_x  \hfill \\
G_y .y' + G_z .z' =  - G_x  \hfill \\
\end{gathered}  \right. \hfill \\
\begin{array}{*{20}c}
{y' = \frac{{\left| {\begin{array}{*{20}c}
{ - F_x } & {F_z }  \\
{ - G_x } & {G_z }  \\

\end{array} } \right|}}
{{\left| {\begin{array}{*{20}c}
{F_y } & {F_z }  \\
{G_y } & {G_z }  \\

\end{array} } \right|}}} & {z' = \frac{{\left| {\begin{array}{*{20}c}
{F_y } & { - F_x }  \\
{G_y } & { - G_x }  \\

\end{array} } \right|}}
{{\left| {\begin{array}{*{20}c}
{F_y } & {F_z }  \\
{G_y } & {G_z }  \\

\end{array} } \right|}}}  \\

\end{array}  \hfill \\
\end{gathered}
\]

\end{dem}

\section{Teorema do Valor M\'edio} \label{sec18}

\begin{lem}[TVM de Cauchy] \label{lem02}
\index{Teorema!do valor m\'edio!de Cauchy}
Sejam $f,g$ deriv\'aveis e cont\'inuas em $\left] {a,b} \right[$, onde $g$ n\~ao constante e $g\left( b \right) \ne g\left( a \right)$. Ent\~ao, existe $c \in \left] {a,b} \right[$ tal que

\[
\frac{{f\left( b \right) - f\left( a \right)}}
{{g\left( b \right) - g\left( a \right)}} = \frac{{f'\left( c \right)}}
{{g'\left( c \right)}}
\]

\end{lem}

\begin{dem}
Pelo Teorema de Rolle \ref{tr}, temos:

\[
\begin{gathered}
h\left( x \right) = f\left( x \right) - mg\left( x \right) \hfill \\
h\left( a \right) = h\left( b \right) \hfill \\
\Rightarrow m = \frac{{f\left( b \right) - f\left( a \right)}}
{{g\left( b \right) - g\left( a \right)}} \hfill \\
\end{gathered}
\]

Novamente, pelo Teorema de Rolle, $\exists c \in \left] {a,b} \right[$ tal que $h'\left( c \right) = 0$

\[
\begin{gathered}
  h'\left( x \right) = f'\left( x \right) - \left( {\frac{{f\left( b \right) - f\left( a \right)}}
{{g\left( b \right) - g\left( a \right)}}} \right)g'\left( x \right) \hfill \\
  0 = h'\left( c \right) = f'\left( c \right) - \left( {\frac{{f\left( b \right) - f\left( a \right)}}
{{g\left( b \right) - g\left( a \right)}}} \right)g'\left( c \right) \hfill \\
\end{gathered}
\]

\end{dem}

\begin{prop}
Se $f:I \to \R$ duas vezes deriv\'avel e $x,x_0 \in I$. Ent\~ao, existe $\overline x$ entre $x$ e $x_0$ tal que

\[
f\left( x \right) = f\left( {x_0 } \right) + f'\left( {x_0 } \right)\left( {x - x_0 } \right) + \frac{{f''\left( {\overline x } \right)}}
{2}\left( {x - x_0 } \right)^2
\]

\end{prop}

\begin{dem}
Seja $E\left( x \right) = f\left( x \right) - f\left( {x_0 } \right) - f'\left( {x_0 } \right)\left( {x - x_0 } \right)$

Ent\~ao, $E\left( {x_0 } \right) = 0$ e $E'\left( {x_0 } \right) = 0$.

Seja, $h\left( x \right) = \left( {x - x_0 } \right)^2$

Ent\~ao, $h\left( {x_0 } \right) = 0$ e $h'\left( {x_0 } \right) = 0$.

\[
\frac{{E\left( x \right)}}
{{h\left( x \right)}} = \frac{{E\left( x \right) - E\left( {x_0 } \right)}}
{{h\left( x \right) - h\left( {x_0 } \right)}}
\]

Usando o Lema anterior \ref{lem02}, temos:

\[
\frac{{E\left( x \right) - E\left( {x_0 } \right)}}
{{h\left( x \right) - h\left( {x_0 } \right)}} = \frac{{E'\left( {x_1 } \right)}}
{{h'\left( {x_1 } \right)}} = \frac{{E'\left( {x_1 } \right) - E'\left( {x_0 } \right)}}
{{h'\left( {x_1 } \right) - h\left( {x_0 } \right)}}
\]

Novamente,

\[
\frac{{E'\left( {x_1 } \right) - E'\left( {x_0 } \right)}}
{{h'\left( {x_1 } \right) - h\left( {x_0 } \right)}} = \frac{{E''\left( {\overline x } \right)}}
{{h''\left( {\overline x } \right)}}
\]

Portanto, $E''\left( {\overline x } \right) = f''\left( {\overline x } \right)$ e $h''\left( {\overline x } \right) = 2$

Ent\~ao,

\[
\begin{gathered}
\frac{{E''\left( {\overline x } \right)}}
{{h''\left( {\overline x } \right)}} = \frac{{f''\left( {\overline x } \right)}}
{2} \hfill \\
\Rightarrow \frac{{E\left( x \right)}}
{{h\left( x \right)}} = \frac{{f''\left( {\overline x } \right)}}
{2} \hfill \\
\Rightarrow E\left( x \right) = \frac{{f''\left( {\overline x } \right)}}
{2}\left( {x - x_0 } \right)^2  \hfill \\
   \Rightarrow f\left( x \right) = f\left( {x_0 } \right) + f'\left( {x_0 } \right)\left( {x - x_0 } \right) + \frac{{f''\left( {\overline x } \right)}}
{2}\left( {x - x_0 } \right)^2  \hfill \\
\end{gathered}
\]

\end{dem}

Em geral se $f$ \'e de classe $C^{n+1}$ e $x,x_0 \in I$. Ent\~ao, existe $\overline x$ entre $x$ e $x_0$ tal que

\[
\begin{gathered}
  f\left( x \right) = \underbrace {f\left( {x_0 } \right) + f'\left( {x_0 } \right)\left( {x - x_0 } \right) + \frac{{f''\left( {x_0 } \right)}}
{2}\left( {x - x_0 } \right)^2  + \frac{{f'''\left( {x_0 } \right)}}
{{3!}}\left( {x - x_0 } \right)^3 }_{\left( I \right)} + \ldots + \hfill \\
  + \underbrace {\frac{{f^{n + 1} \left( {\overline x } \right)}}{{\left( {n + 1} \right)!}}\left( {x - x_0 } \right)^{n + 1} }_{\left( {II} \right)} \hfill \\ 
\end{gathered} 
\]

Onde: $(I)$ \'e o Polin\^omio de Taylor de ordem $n$ em volta de $x_0$.

$(II)$ \'e o resto $E(x)$ de ordem $n+1$.

\begin{teo}[Teorema do Valor M\'edio] \label{tvm2}
\index{Teorema!do valor m\'edio}
\begin{sloppypar}
Seja $A \subset \R^n$ aberto convexo e ${F:A \to \R}$ uma fun\c c\~ao diferenci\'avel. Sejam $x,x_0 \in A$. Ent\~ao, existe $\overline x$ no segmento $\overline {xx_0}$ tal que
\end{sloppypar}

\[
f\left( x \right) - f\left( {x_0 } \right) = \left\langle {\nabla f\left( {\overline x } \right),x - x_0 } \right\rangle
\]

\end{teo}

\begin{dem}
    Seja $g:\left[ {0,1} \right] \to \R$ dada por $g\left( t \right) = f\left( {x_0  + t\left( {x - x_0 } \right)} \right)$

    $g\left( 0 \right) = f\left( {x_0 } \right)$ e $g\left( 1 \right) = f\left( x \right)$

$g$ \'e diferenci\'avel em $\left] {0,1} \right[$, aplicando TVM para $g$, temos

\[
\begin{gathered}
\frac{{g\left( 1 \right) - g\left( 0 \right)}}
{{1 - 0}} = g'\left( {\overline t } \right),\overline t  \in \left] {0,1} \right[ \hfill \\
   \Rightarrow f\left( x \right) - f\left( {x_0 } \right) = g'\left( {\overline t } \right) = \left\langle {\nabla f\left( {x_0  + \overline t \left( {x - x_0 } \right)} \right),x - x_0 } \right\rangle  \hfill \\
\end{gathered}
\]

    Onde, $x_0  + \overline t \left( {x - x_0 } \right) = \overline x$ entre $x$ e $x_0$.
\end{dem}

\begin{cor}
Nas mesmas condi\c c\~oes com $u = \frac{{x - x_0 }}
{{\left\| {x - x_0 } \right\|}}$, temos

\[
\frac{{f\left( x \right) - f\left( {x_0 } \right)}}
{{\left\| {x - x_0 } \right\|}} = \frac{{\partial f}}
{{\partial u}}\left( {\overline x } \right)
\]

\end{cor}

\section{F\'ormula de Taylor com Resto de Lagrange}\label{sec18a}

\subsection{Polin\^omio de Taylor de Ordem 1}

Seja $f:A \subset \R^2 \to \R$, $A$ aberto convexo e $f$ de classe $C^2$. Sejam ainda $\left( {x_0 ,y_0 } \right) \in A$ e $\left( {h,k} \right) \in \R^2 ,\left( {h,k} \right) \ne \left( {0,0} \right)$ tal que $\left( {x_0  + h,y_0  + k} \right) \in A$.

Considere $g:\left[ {0,1} \right] \to \R,g\left( t \right) = f\left( {\left( {x_0 ,y_0 } \right) + t\left( {h,k} \right)} \right),t \in \left[ {0,1} \right]$

O \textit{Polin\^omio de Taylor}\index{Polin\^omio de Taylor} de ordem 1 para $g$ em volta de $0$:

\[
\begin{gathered}
  P_1 \left( t \right) = g\left( 0 \right) + g'\left( 0 \right)\left( {t - 0} \right) \hfill \\
E\left( t \right) = \frac{{g''\left( {\overline t } \right)}}
{2}\left( {t - 0} \right)^2 ,\overline t  \in \left] {0,t} \right[ \hfill \\
   \Rightarrow g\left( 1 \right) = g\left( 0 \right) + g'\left( 0 \right)t + E\left( 1 \right) \hfill \\
  g\left( 1 \right) = g\left( 0 \right) + g'\left( 0 \right)t + \frac{{g''\left( {\overline t } \right)}}
{2}1^2  \hfill \\
   \Rightarrow f\left( {x_0  + h,y_0  + k} \right) = f\left( {x_0 ,y_0 } \right) + \left\langle {\nabla f\left( {x_0 ,y_0 } \right),\left( {h,k} \right)} \right\rangle  + \frac{{g''\left( {\overline t } \right)}}
{2} \hfill \\
\end{gathered}
\]

Onde, $\left\langle {\nabla f\left( {x_0 ,y_0 } \right),\left( {h,k} \right)} \right\rangle  = \frac{{\partial f}}
{{\partial x}}\left( {x_0 ,y_0 } \right)h + \frac{{\partial f}}
{{\partial y}}\left( {x_0 ,y_0 } \right)k$

\[
\begin{gathered}
g''\left( {\overline t } \right) = \frac{d}
{{dt}}g'\left( {\overline t } \right) \hfill \\
= \frac{d}
{{dt}}\left( {\frac{{\partial f}}
{{\partial x}}\left( {x_0  + \overline t h,y_0  + \overline t k} \right)h + \frac{{\partial f}}
{{\partial y}}\left( {x_0  + \overline t h,y_0  + \overline t k} \right)k} \right) \hfill \\
   = \scriptstyle{f_{xx} \left( {x_0  + \overline t h,y_0  + \overline t k} \right)h^2  + f_{xy} \left( {x_0  + \overline t h,y_0  + \overline t k} \right)hk + f_{yx} \left( {x_0  + \overline t h,y_0  + \overline t k} \right)hk + f_{yy} \left( {x_0  + \overline t h,y_0  + \overline t k} \right)k^2}  \hfill \\
= f_{xx} h^2  + 2f_{xy} hk + f_{yy} k^2  \hfill \\
\end{gathered}
\]

Onde, $f_{xx}  = \frac{{\partial ^2 f}}{{\partial x^2 }}$ e $f_{xy}  = \frac{{\partial ^2 f}}{{\partial x\partial y}}$.

Fazendo $\left( {x,y} \right) = \left( {x_0  + h,y_0  + k} \right)$ e $\left( {\overline x ,\overline y } \right) = \left( {x_0  + \overline t h,y_0  + \overline t k} \right)$, temos

\[\boxed{
\begin{gathered}
  f\left( {x,y} \right) = \underbrace {f\left( {x_0 ,y_0 } \right) + f_x \left( {x_0 ,y_0 } \right)\left( {x - x_0 } \right) + f_y \left( {x_0 ,y_0 } \right)\left( {y - y_0 } \right)}_{P_1 {\text{ de Taylor}}} +  \hfill \\
+ \frac{1}
{2}\underbrace {\left[ {f_{xx} \left( {\overline x ,\overline y } \right)\left( {x - x_0 } \right)^2  + 2f_{xy} \left( {\overline x ,\overline y } \right)\left( {x - x_0 } \right)\left( {y - y_0 } \right) + f_{yy} \left( {\overline x ,\overline y } \right)\left( {y - y_0 } \right)^2 } \right]}_{E\left( {x,y} \right)} \hfill \\
\end{gathered}
}\]

\begin{ex}
    Seja $f\left( {x,y} \right) = \ln \left( {x + y} \right)$. Determine o polin\^omio de Taylor de ordem 1 de $f$ em volta de $\left( {{\raise0.5ex\hbox{$\scriptstyle 1$}
\kern-0.1em/\kern-0.15em
\lower0.25ex\hbox{$\scriptstyle 2$}},{\raise0.5ex\hbox{$\scriptstyle 1$}
\kern-0.1em/\kern-0.15em
\lower0.25ex\hbox{$\scriptstyle 2$}}} \right)$.
\end{ex}

\begin{sol}
\[
\begin{gathered}
P_1 \left( {x,y} \right) = f\left( {{\raise0.5ex\hbox{$\scriptstyle 1$}
\kern-0.1em/\kern-0.15em
\lower0.25ex\hbox{$\scriptstyle 2$}},{\raise0.5ex\hbox{$\scriptstyle 1$}
\kern-0.1em/\kern-0.15em
\lower0.25ex\hbox{$\scriptstyle 2$}}} \right) + f_x \left( {{\raise0.5ex\hbox{$\scriptstyle 1$}
\kern-0.1em/\kern-0.15em
\lower0.25ex\hbox{$\scriptstyle 2$}},{\raise0.5ex\hbox{$\scriptstyle 1$}
\kern-0.1em/\kern-0.15em
\lower0.25ex\hbox{$\scriptstyle 2$}}} \right)\left( {x - {\raise0.5ex\hbox{$\scriptstyle 1$}
\kern-0.1em/\kern-0.15em
\lower0.25ex\hbox{$\scriptstyle 2$}}} \right) + f_y \left( {{\raise0.5ex\hbox{$\scriptstyle 1$}
\kern-0.1em/\kern-0.15em
\lower0.25ex\hbox{$\scriptstyle 2$}},{\raise0.5ex\hbox{$\scriptstyle 1$}
\kern-0.1em/\kern-0.15em
\lower0.25ex\hbox{$\scriptstyle 2$}}} \right)\left( {y - {\raise0.5ex\hbox{$\scriptstyle 1$}
\kern-0.1em/\kern-0.15em
\lower0.25ex\hbox{$\scriptstyle 2$}}} \right) \hfill \\
= 0 + 1\left( {x - {\raise0.5ex\hbox{$\scriptstyle 1$}
\kern-0.1em/\kern-0.15em
\lower0.25ex\hbox{$\scriptstyle 2$}}} \right) + 1\left( {y - {\raise0.5ex\hbox{$\scriptstyle 1$}
\kern-0.1em/\kern-0.15em
\lower0.25ex\hbox{$\scriptstyle 2$}}} \right) \hfill \\
= x + y - 1 \hfill \\
\end{gathered}
\]

\[
\begin{gathered}
f_x  = \frac{1}
{{x + y}} = f_y  \hfill \\
f_{xx}  = \frac{{ - 1}}
{{\left( {x + y} \right)^2 }} = f_{yy}  = f_{xy}  \hfill \\
E\left( {x,y} \right) = \frac{{ - 1}}
{{2\left( {x + y} \right)^2 }}\left( {\left( {x - x_0 } \right)^2  + 2\left( {x - x_0 } \right)\left( {y - y_0 } \right) + \left( {y - y_0 } \right)^2 } \right) \hfill \\
\end{gathered}
\]

Se $x + y > 1$, ent\~ao

\[
\left| {E\left( {x,y} \right)} \right| \leqslant \frac{1}
{2}\left( {\left( {x - {\raise0.5ex\hbox{$\scriptstyle 1$}
\kern-0.1em/\kern-0.15em
\lower0.25ex\hbox{$\scriptstyle 2$}}} \right)^2  + 2\left( {x - {\raise0.5ex\hbox{$\scriptstyle 1$}
\kern-0.1em/\kern-0.15em
\lower0.25ex\hbox{$\scriptstyle 2$}}} \right)\left( {y - {\raise0.5ex\hbox{$\scriptstyle 1$}
\kern-0.1em/\kern-0.15em
\lower0.25ex\hbox{$\scriptstyle 2$}}} \right) + \left( {y - {\raise0.5ex\hbox{$\scriptstyle 1$}
\kern-0.1em/\kern-0.15em
\lower0.25ex\hbox{$\scriptstyle 2$}}} \right)^2 } \right)
\]

Repetindo o argumento acima e calculando o Polin\^omio de Taylor de ordem 2 para $g(t)$ em volta de $t=0$ no ponto $t=1$. ($f$ de classe $C^3$)

\[
\begin{gathered}
  g\left( t \right) = g\left( 0 \right) + g'\left( 0 \right) + \frac{{g''\left( 0 \right)t^2 }}
{2} + \frac{{g'''\left( {\overline t } \right)t^3 }}
{{3!}} \hfill \\
  g\left( 1 \right) = g\left( 0 \right) + g'\left( 0 \right) + \frac{{g''\left( 0 \right)}}
{2} + \frac{{g'''\left( {\overline t } \right)}}
{{3!}} \hfill \\
  f\left( {x,y} \right) = f\left( {x_0 ,y_0 } \right) + f_x \left( {x_0 ,y_0 } \right)\left( {x - x_0 } \right) + f_y \left( {x_0 ,y_0 } \right)\left( {y - y_0 } \right) +  \hfill \\
 + \frac{1}{2}\left[ {f_{xx} \left( {x_0 ,y_0 } \right)\left( {x - x_0 } \right)^2  + 2f_{xy} \left( {x_0 ,y_0 } \right)\left( {x - x_0 } \right)\left( {y - y_0 } \right) + f_{yy} \left( {x_0 ,y_0 } \right)\left( {y - y_0 } \right)^2 } \right] \hfill \\
\end{gathered}
\]

Denomina-se Polin\^omio de Taylor de ordem 2.

\[
E\left( {x,y} \right) = \frac{1}
{{3!}}\left[ {\sum\limits_{k = 0}^3 {\left( {\begin{array}{*{20}c}
3  \\
k  \\

\end{array} } \right)\frac{{\partial ^3 f}}
{{\partial x^{3 - k} \partial y^k }}\left( {\overline x ,\overline y } \right)\left( {x - x_0 } \right)^{3 - k} \left( {y - y_0 } \right)^k } } \right]
\]

\end{sol}

Em geral se $f$ \'e de classe $C^{n+1}$, $A$ aberto convexo de $\R^2, \left( {x_0 ,y_0 } \right) \in A$ e $\left( {\overline x ,\overline y } \right)$ entre $\left( {x,y} \right) \in A$ e $\left( {x_0 ,y_0 } \right)$, temos:

\[\boxed{
\begin{gathered}
  f\left( {x,y} \right) = f\left( {x_0 ,y_0 } \right) + \sum\limits_{j = 1}^n {\frac{1}
{{j!}}\left( {\sum\limits_{k = 0}^j {\left( {\begin{array}{*{20}c}
j  \\
k  \\

\end{array} } \right)\frac{{\partial ^j f}}
{{\partial x^{j - k} \partial y^k }}\left( {x_0 ,y_0 } \right)\left( {x - x_0 } \right)^{j - k} \left( {y - y_0 } \right)^k } } \right)}  +  \hfill \\
+ \frac{1}
{{\left( {n + 1} \right)!}}\sum\limits_{k = 0}^{n + 1} {\left( {\left( {\begin{array}{*{20}c}
{n + 1}  \\
k  \\

\end{array} } \right)\frac{{\partial ^{n + 1} f}}
{{\partial x^{n + 1 - k} \partial y^k }}\left( {\overline x ,\overline y } \right)\left( {x - x_0 } \right)^{n + 1 - k} \left( {y - y_0 } \right)^k } \right)}  \hfill \\
\end{gathered}
}\]


\chapter{M\'aximos e M\'inimos} \label{chap04}

\section{Pontos de M\'aximo e Pontos de M\'inimo} \label{sec20}

Seja $f:A \subset \R^n \to \R$, $A$ aberto. $f$ \'e constante, ent\~ao, $\nabla f = \left( {0,...,0} \right)$

Se $\nabla f(x) = 0, \forall x \in A$, ent\~ao, $f$ \'e constante?

Falso: $f:A \to \R, A = \left\{ {\left( {x,y} \right) \in \R^2 :x \ne 0} \right\}$

\begin{equation*}
f(x,y)=\left\{ \begin{array}{cl}\displaystyle
1 & \textrm{se } x > 0\\
0 & \textrm{se } x< 0 \end{array}\right.
\end{equation*}

$\nabla f\left( {x,y} \right) = \left( {0,0} \right)$ e $f$ n\~ao \'e constante.

\begin{teo}
    Seja $A \subset \R^n$ aberto conexo por caminhos e $f:A \to \R$ tal que $\left. {\nabla f} \right|_A  \equiv 0$. Ent\~ao, $f$ \'e constante em $A$.
\end{teo}

\begin{dem}
    $A$ conexo por caminhos, implica que dados $x,y \in A$ existe uma poligonal $\gamma$ que une $x$ a $y$, $\operatorname{Im} \gamma  \subset A$. Tal poligonal pode ser escrita como uni\~ao de segmentos $\overline {x_{i - 1} x_i } ,i = 1,2,...,n$ com $x_0 = x$ e $x_n = y$. Em cada segmento $\overline {x_{i - 1} x_i }$ temos o TVM:

\[
\begin{gathered}
  f\left( {x_i } \right) - f\left( {x_{i - 1} } \right) = \left\langle {\nabla f\left( {\overline x } \right),x_i  - x_{i - 1} } \right\rangle  = 0 \hfill \\
   \Rightarrow f\left( {x_i } \right) = f\left( {x_{i - 1} } \right),i = 1,2,...,n \hfill \\
  f\left( x \right) = f\left( {x_0 } \right) = f\left( {x_1 } \right) = f\left( {x_2 } \right) = ... = f\left( {x_n } \right) = f\left( y \right) \hfill \\
\Rightarrow f\left( x \right) = f\left( y \right),\forall x,y \in A \hfill \\
\end{gathered}
\]

$\Rightarrow f$ \'e constante em $A$.

% figura 38
\end{dem}

\begin{cor}
    Sejam $A \subset \R^n$ aberto, conexo por caminho e $f,g:A \to \R$ tal que $\nabla f\left( x \right) = \nabla g\left( x \right),\forall x \in A$. Ent\~ao, existe $k \in \R$ tal que $f(x) = g(x) + k$.
\end{cor}

\begin{dem}
Considere a fun\c c\~ao

\[
\begin{gathered}
h\left( x \right) = f\left( x \right) - g\left( x \right) \hfill \\
   \Rightarrow \nabla h\left( x \right) = \nabla \left( {f - g} \right)\left( x \right) \hfill \\
   \Rightarrow \nabla h\left( x \right) = \nabla f\left( x \right) - \nabla g\left( x \right) = 0 \hfill \\
\Rightarrow h\left( x \right) = k \hfill \\
\Rightarrow f\left( x \right) = g\left( x \right) + k \hfill \\
\end{gathered}
\]

\end{dem}

\begin{ex}
    Ache $f:\R^2 \to \R$ tal que $\nabla f\left( {x,y} \right) = \left( {3x^2 y^2  + 4,2x^3 y + y^2 } \right)$.
\end{ex}

\begin{sol}
\[
\begin{gathered}
\frac{{\partial f}}
{{\partial x}} = 3x^2 y^2  + 4 \Rightarrow f\left( {x,y} \right) = x^3 y^2  + 4x + g\left( y \right)\left( 1 \right) \hfill \\
\frac{{\partial f}}
{{\partial y}} = 2x^3 y + y^2  \hfill \\
\end{gathered}
\]

    Derivando $\overbrace {f\left( {x,y} \right)}^{\left( 1 \right)}$ em rela\c c\~ao a $y$, temos:

\[
\begin{gathered}
\frac{{\partial f}}
{{\partial y}} = 2x^3 y + g'\left( y \right) \hfill \\
\Rightarrow 2x^3 y + y^2  = 2x^3 y + g'\left( y \right) \hfill \\
\Rightarrow g'\left( y \right) = y^2  \hfill \\
\Rightarrow g\left( y \right) = \frac{{y^3 }}
{3} + k \hfill \\
\Rightarrow f\left( {x,y} \right) = x^3 y^2  + 4x + \frac{{y^3 }}
{3} + k \hfill \\
\end{gathered}
\]

\end{sol}

Ser\'a que sempre existe $f$ dado tal que $\nabla f = \left( {P\left( {x,y} \right),Q\left( {x,y} \right)} \right)$?

\begin{prop}
    Sejam $A \subset \R^2$ aberto e $P,Q:A \to \R$ de classe $C^1$. Para que exista $f:A \to \R$ tal que $\nabla f = \left( {P,Q} \right)$ \'e necess\'ario que $\frac{{\partial P}}{{\partial y}} = \frac{{\partial Q}}{{\partial x}}$.
\end{prop}

\begin{dem}
Se existe $f$ tal que $\nabla f = \left( {P,Q} \right)$, ent\~ao,

\[
\begin{array}{*{20}c}
{\frac{{\partial f}}
{{\partial x}} = P} & {\frac{{\partial f}}
{{\partial y}} = Q}  \\

\end{array}
\]

$P,Q$ s\~ao de classe $C^1$, ent\~ao,

\[
\begin{array}{*{20}c}
{\frac{{\partial P}}
{{\partial y}} = \frac{{\partial ^2 f}}
{{\partial y\partial x}}} & {\frac{{\partial ^2 f}}
{{\partial x\partial y}} = \frac{{\partial Q}}
{{\partial x}}}  \\

\end{array}
\]

    ambas s\~ao cont\'inuas, ent\~ao, como $f$ \'e de classe $C^2$, pelo Teorema de Schwarz \ref{t5}, temos,

\[
\frac{{\partial ^2 f}}
{{\partial x\partial y}} = \frac{{\partial ^2 f}}
{{\partial y\partial x}}
\]

\end{dem}

\begin{ex}
    Existe $f:\R^2 \to \R$ tal que $\nabla f = \left( {\underbrace {xy}_P,\underbrace y_Q} \right)$?
\end{ex}

\begin{sol}
\[
\frac{{\partial P}}
{{\partial y}} = x \ne 0 = \frac{{\partial Q}}
{{\partial x}}
\]

    Ent\~ao, n\~ao existe $f:\R^2 \to \R$ tal que $\nabla f = \left( {xy,y} \right)$.
\end{sol}

\begin{ex}
    Existe $f:\R^2 \backslash \left\{ {\left( {0,0} \right)} \right\}$ tal que $\nabla f = \left( {\underbrace {\frac{x}
{{x^2  + y^2 }}}_P,\underbrace {\frac{y}
{{x^2  + y^2 }} - e^{ - y} }_Q} \right)$?
\end{ex}

\begin{sol}
\[
\left\{ \begin{gathered}
\frac{{\partial P}}
{{\partial y}} = \frac{{ - 2xy}}
{{\left( {x^2  + y^2 } \right)^2 }} \hfill \\
\frac{{\partial Q}}
{{\partial x}} = \frac{{ - 2xy}}
{{\left( {x^2  + y^2 } \right)^2 }} \hfill \\
\end{gathered}  \right. \Rightarrow P_y  = Q_x
\]

tem chance de existir $f$.

\[
\begin{gathered}
f_x  = \frac{x}
{{x^2  + y^2 }} \Rightarrow f\left( {x,y} \right) = \frac{1}
{2}\ln \left( {x^2  + y^2 } \right) + g\left( y \right) \hfill \\
\Rightarrow \frac{y}
{{x^2  + y^2 }} - e^{ - y}  = f_y  = \frac{y}
{{x^2  + y^2 }} + g'\left( y \right) \hfill \\
\Rightarrow g'\left( y \right) =  - e^{ - y}  \hfill \\
\Rightarrow g\left( y \right) = e^{ - y}  + k \hfill \\
\Rightarrow f\left( {x,y} \right) = \frac{1}
{2}\ln \left( {x^2  + y^2 } \right) + e^{ - y}  + k \hfill \\
\end{gathered}
\]

\end{sol}

Ser\'a que a condi\c c\~ao da proposi\c c\~ao \'e suficiente?

Isso depende "mais do dom\'inio" do que das express\~oes de $P$ e $Q$.

\begin{ex}
Considere o campo $\left( {P,Q} \right) = \left( {\frac{{ - y}}
{{x^2  + y^2 }},\frac{x}
{{x^2  + y^2 }}} \right)$ em $\R^2 \backslash \left\{ {\left( {0,0} \right)} \right\}$.
\end{ex}

\begin{sol}
\[
\frac{{\partial P}}
{{\partial y}} = \frac{{ - \left( {x^2  + y^2 } \right) + 2y^2 }}
{{\left( {x^2  + y^2 } \right)^2 }} = \frac{{\partial Q}}
{{\partial x}}
\]

    Por\'em, n\~ao existe $f:\R^2 \backslash \left\{ {\left( {0,0} \right)} \right\} \to \R$ tal que $\nabla f = \left( {P,Q} \right)$.

\

Uma justificativa para tal fato \'e:

Se $f$ \'e potencial de $(P,Q)$, ent\~ao,

\[
    \int\limits_a^b {\left\langle {\left( {P,Q} \right),\gamma '\left( t \right)} \right\rangle dt}  = 0
\]

    para toda curva fechada $\gamma$ de classe $C^1$. Ser fechada significa que a curva tem $\gamma \left( a \right) = \gamma \left( b \right)$.

\[
\begin{gathered}
  \int\limits_a^b {\left\langle {\left( {P,Q} \right),\gamma '\left( t \right)} \right\rangle dt}  = \int\limits_a^b {\left\langle {\nabla f\left( {\gamma \left( t \right)} \right),\gamma '\left( t \right)} \right\rangle dt}  =  \hfill \\
\int\limits_a^b {\frac{d}
{{dt}}\left( {f \circ \gamma } \right)\left( t \right)dt}  = \left. {\left( {f \circ \gamma } \right)} \right|_a^b  = f\left( {\gamma \left( b \right)} \right) - f\left( {\gamma \left( a \right)} \right) = 0 \hfill \\
\end{gathered}
\]

Ent\~ao, continuando a resolu\c c\~ao do exemplo, temos:

    Escolhendo $\gamma \left( t \right) = \left( {\cos t,\sin t} \right),t = \left[ {0,2\pi } \right]$

\[
\int\limits_0^{2\pi } {\left\langle {\left( {\frac{{ - \sin t}}
{{\cos ^2 t + \sin ^2 t}},\frac{{\cos t}}
{{\cos ^2 t + \sin ^2 t}}} \right),\left( { - \sin t,\cos t} \right)} \right\rangle dt}  = \int\limits_0^{2\pi } {1dt}  = 2\pi  \ne 0
\]

Portanto, n\~ao existe potencial para $(P,Q)$.

\end{sol}

\begin{defn} [Ponto de M\'aximo e de M\'inimo]
  \begin{sloppypar}
    Seja $A \subset \R^n$ e $f:A \to \R$ fun\c c\~ao. $x_0 \in A$ \'e \textit{ponto de m\'aximo local}\index{Ponto!de m\'aximo local} de $f$ se existe $B_\delta  \left( {x_0 } \right)$ tal que ${f\left( x \right) \leqslant f\left( {x_0 } \right),\forall x \in B_\delta  \left( {x_0 } \right)}$ e \textit{m\'aximo global} se $f\left( x \right) \leqslant f\left( {x_0 } \right),\forall x \in A$. Analogamente definimos m\'inimo local e m\'inimo global.
  \end{sloppypar}
\end{defn}

\begin{ex}
Seja $f\left( {x,y} \right) = x^2  + y^2$
\end{ex}

\begin{sol}
$f\left( {x,y} \right) \geqslant 0 = f\left( {0,0} \right)$, ent\~ao, $\left( {0,0} \right)$ \'e m\'inimo global de $f$.
\end{sol}

\begin{ex}
Seja $f\left( {x,y} \right) = 2x - y$ definida em $A \subset \R^2$ dada por \\
$x \geqslant 0,y \geqslant 0,x + y \leqslant 3$.
\end{ex}

\begin{sol}
Temos que, $y \geqslant x$ e $y \leqslant 3 - x$

Fazendo as curvas de n\'ivel de $f$, temos:

\[
\begin{gathered}
2x - y = c \hfill \\
y = 2x - c \hfill \\
\end{gathered}
\]

% figura 39

\end{sol}

\begin{ex}
\begin{equation*}
f(x,y)=\left\{ \begin{array}{cl}\displaystyle
x^2 + y^2 & \textrm{se }x^2 + y^2 \leqslant 4\\
        1 - \left( {x - 3} \right)^2  - y^2& \textrm{se } x^2 + y^2 > 4 \end{array}\right.
\end{equation*}

\end{ex}

\begin{sol}
$(0,0)$ \'e m\'inimo local (n\~ao global)

$3,0)$ \'e m\'aximo local (n\~ao global)

Os pontos onde $x^2 + y^2 = 4$ s\~ao m\'aximo global.

% figura 40

\end{sol}

\newpage 

\begin{teo}
Seja $x_0 \in A, A \subset \R^n$ aberto. Se $x_0$ \'e \textit{ponto extremo} de $f:A \to \R$, $f$ diferenci\'avel, ent\~ao, $\nabla f\left( {x_0 } \right) = 0$.
\end{teo}

\begin{dem}
    Suponha que $x_0$ \'e m\'aximo local de $f$, isto \'e, existe $B_\delta  \left( {x_0 } \right)$ tal que $f\left( x \right) \leqslant f\left( {x_0 } \right),\forall x \in B_\delta  \left( {x_0 } \right)$.

    Considere $x = \left( {x_1 ,...,x_n } \right)$ e $x_0  = \left( {x_{01} ,...,x_{0n} } \right)$

\[
    g\left( {x_i } \right) = f\left( {x_{01} ,x_{02} ,...,x_i ,...,x_{0n} } \right)
\]

$g$ \'e uma fun\c c\~ao real diferenci\'avel, com m\'aximo em $x_{0i}$, logo

\[
\begin{gathered}
0 = g'\left( {x_{0i} } \right) = \frac{{\partial f}}
{{\partial x_i }}\left( {x_0 } \right),i = 1,2,...,n \hfill \\
\Rightarrow \nabla f\left( {x_0 } \right) = 0 \hfill \\
\end{gathered}
\]

\end{dem}

\begin{defn}
    Sejam $f:A \subset \R^n \to \R$, $A$ aberto e $x_0 \in A$. Dizemos que $x_0$ \'e \textit{ponto cr\'itico}\index{Ponto!cr\'itico} de $f$ se $\nabla f\left( {x_0 } \right) = 0$.

Os candidatos a m\'aximo e m\'inimo de $f$ s\~ao pontos cr\'iticos.
\end{defn}

\begin{ex}
Seja $f\left( {x,y} \right) = x^2  + y^2$
\end{ex}

\begin{sol}
    $\nabla f = \left( {2x,2y} \right) = \left( {0,0} \right) \Leftrightarrow \left( {x,y} \right) = \left( {0,0} \right)$

e $\left( {0,0} \right)$ \'e m\'inimo global.
\end{sol}

\begin{ex}
Seja $f\left( {x,y} \right) = x^2  - y^2$
\end{ex}

\begin{sol}
    $\nabla f = \left( {2x, - 2y} \right) = \left( {0,0} \right) \Leftrightarrow \left( {x,y} \right) = \left( {0,0} \right)$

    $\left( {0,0} \right)$ n\~ao \'e m\'aximo, pois $f\left( {x,0} \right) > f\left( {0,0} \right),x \ne 0$.

    $\left( {0,0} \right)$ n\~ao \'e m\'inimo, pois $f\left( {0,y} \right) < f\left( {0,0} \right),y \ne 0$.

$\left( {0,0} \right)$ \'e chamado ponto de "sela".
\end{sol}

\begin{ex}
    Seja $f\left( {x,y} \right) = x^2  + y^2$ em $A = \left\{ {x \in \R^2 :\left\| x \right\| < 2} \right\}$
\end{ex}

\begin{sol}
    $\nabla f = \left( {0,0} \right) \Leftrightarrow \left( {x,y} \right) = \left( {0,0} \right)$

$\left( {0,0} \right)$ \'e m\'inimo global de $f$.

$\nabla f\left( {2,0} \right) = \left( {4,0} \right) \ne \left( {0,0} \right)$

$\left( {2,0} \right)$ \'e ponto de m\'aximo de $f$.

    Todos os pontos de m\'aximo s\~ao $\max  = \left\{ {\left( {x,y} \right) \in \R^2 :x^2  + y^2  = 4} \right\}$.
\end{sol}

\begin{teo}
    Seja $f:A \subset \R^n \to \R$ de classe $C^2$, $A$ aberto e $x_0 \in A$. Se $f$ tem m\'aximo em $x_0$, ent\~ao, $\nabla f\left( {x_0 } \right) = 0$ e $\displaystyle \frac{{\partial ^2 f}}{{\partial x_i^2 }}\left( {x_0 } \right) \leqslant 0$.
\end{teo}

\begin{dem}
    Se $x = \left( {x_1 ,...,x_n } \right)$ e $x_0  = \left( {x_{01} ,...,x_{0n} } \right)$ considere

\[
    g\left( {x_i } \right) = f\left( {x_{01} ,x_{02} ,...,x_i ,...,x_{0n} } \right)
\]

    $x_{0i}$ \'e m\'aximo de $g$, ent\~ao, $g'\left( {x_{0i} } \right) = 0$ e $g''\left( {x_{0i} } \right) \leqslant 0$

\[
\begin{gathered}
0 = g'\left( {x_{0i} } \right) = \frac{{\partial f}}
{{\partial x_i }}\left( {x_0 } \right) \hfill \\
g'\left( {x_i } \right) = \frac{{\partial f}}
{{\partial x_i }}\left( {x_{01} ,...,x_i ,...,x_{0n} } \right) \hfill \\
g''\left( {x_i } \right) = \frac{{\partial ^2 f}}
{{\partial x_i^2 }}\left( {x_{01} ,...,x_i ,...,x_{0n} } \right) \hfill \\
0 \leqslant g''\left( {x_{0i} } \right) = \frac{{\partial ^2 f}}
{{\partial x_i^2 }}\left( {x_0 } \right) \hfill \\
\end{gathered}
\]

\end{dem}

\begin{ex}
    Seja $f\left( {x,y} \right) = x^3  + y^3  - 3x - 3y + 4$. Calcule os m\'aximos e m\'inimos locais de $f$ em $\R^2$.
\end{ex}

\begin{sol}
\[
\begin{gathered}
\nabla f = \left( {0,0} \right) \hfill \\
   \Rightarrow \left( {3x^2  - 3,3y^2  - 3} \right) = \left( {0,0} \right) \hfill \\
\Rightarrow \left( {x,y} \right) = \left\{ \begin{gathered}
\left( {1,1} \right) \hfill \\
\left( {1, - 1} \right) \hfill \\
\left( { - 1,1} \right) \hfill \\
\left( { - 1, - 1} \right) \hfill \\
\end{gathered}  \right.{\text{ pontos cr\'iticos}} \hfill \\
\left. \begin{gathered}
\frac{{\partial ^2 f}}
{{\partial x^2 }} = 6x \hfill \\
\frac{{\partial ^2 f}}
{{\partial y^2 }} = 6y \hfill \\
\end{gathered}  \right\}\left\{ \begin{gathered}
\left( {1,1} \right) \to {\text{ min local}} \hfill \\
\left( {1, - 1} \right) \to {\text{ pto de sela}} \hfill \\
\left( { - 1,1} \right) \to {\text{ pto de sela}} \hfill \\
\left( { - 1, - 1} \right) \to {\text{ max local}} \hfill \\
\end{gathered}  \right. \hfill \\
\end{gathered}
\]

\end{sol}

\section{Formas Quadr\'aticas em $\R^2$} \label{sec21}

\begin{defn}
\begin{sloppypar}
Uma \textit{forma quadr\'atica}\index{Formas quadr\'aticas} em $\R^2$ \'e uma fun\c c\~ao do tipo ${Q\left( {x,y} \right) = ax^2  + 2bxy + cy^2 ,a,b,c \in \R}$.
Associamos a $Q$ a matriz sim\'etrica
\end{sloppypar}

\[
\begin{gathered}
A = \left( {\begin{array}{*{20}c}
a & b  \\
b & c  \\

\end{array} } \right) \hfill \\
Q\left( {x,y} \right) = \left( {\begin{array}{*{20}c}
x & y  \\

\end{array} } \right)\left( {\begin{array}{*{20}c}
a & b  \\
b & c  \\

\end{array} } \right)\left( {\begin{array}{*{20}c}
x  \\
y  \\

\end{array} } \right) \hfill \\
= \left( {\begin{array}{*{20}c}
x & y  \\

\end{array} } \right)\left( {\begin{array}{*{20}c}
{ax} & {by}  \\
{bx} & {cy}  \\

\end{array} } \right) \hfill \\
= ax^2  + bxy + bxy + cy^2  \hfill \\
\end{gathered}
\]

\end{defn}

A matriz $A$ sim\'etrica, ent\~ao, $A$ \'e diagonaliz\'avel, ou seja, existe uma base ortonormal $\left\{ {v_1 ,v_2 } \right\}$ de $\R^2$ na qual $A$ \'e diagonal.

\textit{Auto valores} de $A$ s\~ao $\det \left( {A - xI} \right) = 0$

\[
\begin{gathered}
\det \left( {\begin{array}{*{20}c}
{a - x} & b  \\
b & {c - x}  \\

\end{array} } \right) = 0 \hfill \\
\Rightarrow x^2  - \left( {a + c} \right)x - b^2  + ac = 0 \hfill \\
   \Rightarrow x = \frac{{\left( {a + c} \right) \pm \sqrt {\left( {a + c} \right)^2  + 4\left( {b^2  - ac} \right)} }}
{2} \hfill \\
\end{gathered}
\]

\textit{Auto vetores} de $A$ s\~ao $v_i$ tal que $Av_i  = \lambda _i v_i$.

Seja $B = \left\{ {v_1 ,v_2 } \right\}$ a base de auto vetores de $A$.

Nesta base, $A$ \'e diagonal

\[
A = \left( {\begin{array}{*{20}c}
{\lambda _1 } & 0  \\
0 & {\lambda _2 }  \\

\end{array} } \right)
\]

e, portanto,

\[
\begin{gathered}
Q\left( {u,v} \right) = \left( {\begin{array}{*{20}c}
u & v  \\

\end{array} } \right)\left( {\begin{array}{*{20}c}
{\lambda _1 } & 0  \\
0 & {\lambda _2 }  \\

\end{array} } \right)\left( {\begin{array}{*{20}c}
u  \\
v  \\

\end{array} } \right) \hfill \\
Q\left( {u,v} \right) = u^2 \lambda _1  + v^2 \lambda _2  \hfill \\
\end{gathered}
\]

\

\begin{itemize}
  \item Se $\lambda _1$ e $\lambda _2$ s\~ao positivos, ent\~ao, $Q\left( {u,v} \right) > 0,\forall \left( {u,v} \right) \in \R^2$
  \item Se $\lambda _1$ e $\lambda _2$ s\~ao negativos, ent\~ao, $Q\left( {u,v} \right) \leqslant 0,\forall \left( {u,v} \right) \in \R^2$
  \item Se $\lambda _1 > 0$ e $\lambda _2 < 0$, existem dire\c c\~oes no plano ao longo das quais $Q$ \'e positiva ou negativa, respectivamente.

Sabemos o sinal de $Q$ para todo vetor $\left( {x,y} \right) \in \R^2$.
  \item Se $\lambda _1 > \lambda _2$, $Q$ assume valor m\'aximo (restrito ao c\'irculo unit\'ario) em \\
$v_1 ,Q\left( {1,0} \right) = \lambda _1$ e m\'inimo em $v_2 ,Q\left( {0,1} \right) = \lambda _2$.
\end{itemize}

% figura 41

\begin{defn}[Hessiano]
    Se $f:A \subset \R^2 \to \R$ \'e de classe $C^2$ definimos o \textit{Hessiano}\index{Hessiano} de $f$ em $\left( {x_0 ,y_0 } \right)$ por

\[
H_f \left( {x_0 ,y_0 } \right) = \left( {\begin{array}{*{20}c}
{f_{xx} \left( {x_0 ,y_0 } \right)} & {f_{xy} \left( {x_0 ,y_0 } \right)}  \\
{f_{xy} \left( {x_0 ,y_0 } \right)} & {f_{yy} \left( {x_0 ,y_0 } \right)}  \\

\end{array} } \right)
\]

\end{defn}

Ele define uma forma quadr\'atica em $\R^2$

\[
\begin{gathered}
Q_f \left( {x,y} \right) = \left( {\begin{array}{*{20}c}
x & y  \\

\end{array} } \right)H_f \left( {x_0 ,y_0 } \right)\left( {\begin{array}{*{20}c}
x  \\
y  \\

\end{array} } \right) \hfill \\
  Q_f \left( {x,y} \right) = f_{xx} \left( {x_0 ,y_0 } \right)x^2  + 2f_{xy} \left( {x_0 ,y_0 } \right)xy + f_{yy} \left( {x_0 ,y_0 } \right)y^2  \hfill \\
\end{gathered}
\]

Existe base $B = \left\{ {v_1 ,v_2 } \right\}$ de $\R^2$ que diagonaliza $Q_f$.

\[
Q_f \left( {u,v} \right) = \lambda _1 u^2  + \lambda _2 v^2
\]

Nesta base, temos

\[
H_f \left( {x_0 ,y_0 } \right) = \left( {\begin{array}{*{20}c}
{\lambda _1 \left( {x_0 ,y_0 } \right)} & 0  \\
0 & {\lambda _2 \left( {x_0 ,y_0 } \right)}  \\

\end{array} } \right)
\]

$\lambda_1 > \lambda_2$

$Q_f$ \'e m\'axima na dire\c c\~ao $v_1$ e m\'inima na dire\c c\~ao $v_2$.

% figura 42

Se $\left( {x_0 ,y_0 } \right)$ \'e ponto cr\'itico de $f$, o polin\^omio de Taylor de ordem 1 em volta de $\left( {x_0 ,y_0 } \right)$ \'e

\[
\begin{gathered}
  f\left( {x,y} \right) = f\left( {x_0 ,y_0 } \right) + \overbrace {f_x \left( {x_0 ,y_0 } \right)\left( {x - x_0 } \right)}^0 + \overbrace {f_y \left( {x_0 ,y_0 } \right)\left( {y - y_0 } \right)}^0 +  \hfill \\
+ \frac{1}
{2}\left( {f_{xx} \left( {\overline x ,\overline y } \right)\left( {x - x_0 } \right)^2  + 2f_{xy} \left( {\overline x ,\overline y } \right)\left( {x - x_0 } \right)\left( {y - y_0 } \right) + f_{yy} \left( {\overline x ,\overline y } \right)\left( {y - y_0 } \right)^2 } \right) \hfill \\ 
\scriptstyle{ \Rightarrow f\left( {x,y} \right) = f\left( {x_0 ,y_0 } \right) + \frac{1}
{2}\left( {f_{xx} \left( {\overline x ,\overline y } \right)\left( {x - x_0 } \right)^2  + 2f_{xy} \left( {\overline x ,\overline y } \right)\left( {x - x_0 } \right)\left( {y - y_0 } \right) + f_{yy} \left( {\overline x ,\overline y } \right)\left( {y - y_0 } \right)^2 } \right)} \hfill \\
\end{gathered}
\]

Fa\c camos $\left( {x_0  + h,y_0  + k} \right) = \left( {x,y} \right)$, $\left( {x - x_0 } \right)^2  = h^2$ e $\left( {y - y_0 } \right)^2  = k^2$.

Ent\~ao, considere

\[
\begin{gathered}
  Q_f \left( {h,k} \right) = f_{xx} \left( {\overline x ,\overline y } \right)h^2  + 2f_{xy} \left( {\overline x ,\overline y } \right)hk + f_{yy} \left( {\overline x ,\overline y } \right)k^2  \hfill \\
  \widetilde Q_f \left( {h,k} \right) = f_{xx} \left( {x_0 ,y_0 } \right)h^2  + 2f_{xy} \left( {x_0 ,y_0 } \right)hk + f_{yy} \left( {x_0 ,y_0 } \right)k^2  \hfill \\
\end{gathered}
\]

O sinal de $Q_f$ \'e determinado pelos auto valores ou pelos sinais de $\lambda_1$ e $\lambda_1\lambda_2$, que s\~ao o primeiro elemento e o determinante de $H_f \left( {\overline x ,\overline y } \right)$, respectivamente, nesta base.

O sinal de $\widetilde Q_f$ s\'o depende do sinal de $f_{xx} \left( {x_0 ,y_0 } \right)$ e $\det H_f \left( {x_0 ,y_0 } \right)$, e ambas s\~ao cont\'inuas, isto implica que $\exists B_\varepsilon  \left( {x_0 ,y_0 } \right)$ tal que $f_{xx} \left( {x,y} \right)$ e $\det H_f \left( {x,y} \right)$ conservam sinal para $\left( {x,y} \right) \in B_\varepsilon  \left( {x_0 ,y_0 } \right)$.

\

\begin{itemize}
  \item Se $\left( {h,k} \right)$ \'e tal que $\left( {x_0  + h,y_0  + k} \right) \in B_\varepsilon  \left( {x_0 ,y_0 } \right)$ o sinal de $\widetilde Q_f \left( {h,k} \right)$ e $Q_f \left( {h,k} \right)$ s\~ao iguais.

      Logo, se $\widetilde Q_f \left( {h,k} \right) > 0$, ou seja, $f_{xx} \left( {x_0 ,y_0 } \right) > 0$ e $\det H_f \left( {x_0 ,y_0 } \right) > 0$, temos

\[
        f\left( {x_0  + h,y_0  + k} \right) = f\left( {x_0 ,y_0 } \right) + \frac{1}{2}\overbrace {Q_f \left( {h,k} \right)}^{ > 0} \geqslant f\left( {x_0 ,y_0 } \right)
\]

Ent\~ao, $f\left( {x_0 ,y_0 } \right)$ \'e m\'inimo local.

  \item Se $\widetilde Q_f \left( {h,k} \right) < 0$, ou seja, $f_{xx} \left( {x_0 ,y_0 } \right) < 0$ e $\det H_f \left( {x_0 ,y_0 } \right) > 0$

\[
\det H_f \left( {x_0 ,y_0 } \right) > 0
\]

Ent\~ao, $f\left( {x_0 ,y_0 } \right)$ \'e m\'aximo local.

  \item Se $\widetilde Q_f \left( {h,k} \right)$ n\~ao tem sinal definido, $\det H_f \left( {x_0 ,y_0 } \right) < 0$.

Ent\~ao, $\left( {x_0 ,y_0 } \right)$ \'e ponto de sela.
\item Se $\det H_f \left( {x_0 ,y_0 } \right) = 0$, nada a afirmar.
\end{itemize}

\begin{ex}
$f\left( {x,y} \right) = x^3  + y^3  - 3x - 3y + 4$
\end{ex}

\begin{sol}
    Pontos cr\'iticos: $\left( {1,1} \right);\left( {1, - 1} \right);\left( { - 1,1} \right);\left( { - 1, - 1} \right)$

\[
\begin{gathered}
\left. \begin{gathered}
f_{xx}  = 6x \hfill \\
f_{xy}  = 0 \hfill \\
f_{yy}  = 6y \hfill \\
\end{gathered}  \right\} \Rightarrow \det H_f \left( {x,y} \right) = \left| {\begin{array}{*{20}c}
{6x} & 0  \\
0 & {6y}  \\

\end{array} } \right| = 36xy \hfill \\
  \left( {1,1} \right):f_{xx}  > 0;\det H_f  = 36 > 0 \Rightarrow {\text{min local}} \hfill \\
  \left( {1, - 1} \right):f_{xx}  > 0;\det H_f  =  - 36 < 0 \Rightarrow {\text{ponto de sela}} \hfill \\
  \left( { - 1,1} \right):f_{xx}  < 0;\det H_f  =  - 36 > 0 \Rightarrow {\text{ponto de sela}} \hfill \\
  \left( { - 1, - 1} \right):f_{xx}  < 0;\det H_f  = 36 > 0 \Rightarrow {\text{max local}} \hfill \\
\end{gathered}
\]

\end{sol}

\newpage 

\begin{ex}
$f\left( {x,y} \right) = 3x^4  + 2y^4$
\end{ex}

\begin{sol}
\[
\begin{gathered}
  \nabla f = \left( {0,0} \right) \Leftrightarrow \left( {x,y} \right) = \left( {0,0} \right) \hfill \\
H_f \left( {0,0} \right) = \left( {\begin{array}{*{20}c}
0 & 0  \\
0 & 0  \\

\end{array} } \right) \hfill \\
\end{gathered}
\]

    O crit\'erio n\~ao se aplica, mas $f\left( {x,y} \right) \geqslant 0 = f\left( {0,0} \right)$, portanto, $\left( {0,0} \right)$ \'e m\'inimo local.
\end{sol}

\begin{ex}
$f\left( {x,y} \right) = x^5  + 2y^5$
\end{ex}

\begin{ex}
    Construa uma caixa sem tampa com volume $1$, de custo m\'inimo sabendo que o material das paredes custa o triplo do usado no fundo.
\end{ex}

\begin{sol}
% figura 43

Volume:

\[
\begin{gathered}
v = abc = 1 \Rightarrow c = \frac{1}
{{ab}} \hfill \\
f\left( {a,b} \right) = 3\left( {2bc + 2ac} \right) + ab \hfill \\
\end{gathered}
\]

Custo total:

\[
\begin{gathered}
f\left( {a,b} \right) = 3\left( {\frac{2}
a + \frac{2}
{b}} \right) + ab \hfill \\
\nabla f = \left( { - \frac{6}
{{a^2 }} + b, - \frac{6}
{{b^2 }} + a} \right) = 0 \hfill \\
\Rightarrow \left\{ \begin{gathered}
a^2 b = 6 \hfill \\
ab^2  = 6 \hfill \\
\end{gathered}  \right. \Rightarrow a = b = \sqrt[3]{6} \hfill \\
\end{gathered}
\]

$\sqrt[3]{6}$ \'e ponto cr\'itico de $f$.

\[
H_f  = \left( {\begin{array}{*{20}c}
{f_{aa} } & {f_{ab} }  \\
{f_{ab} } & {f_{bb} }  \\

\end{array} } \right) = \left( {\begin{array}{*{20}c}
{\tfrac{{12}}
{{a^3 }}} & 1  \\
1 & {\tfrac{{12}}
{{b^3 }}}  \\

\end{array} } \right)
\]

em $\left( {\sqrt[3]{6},\sqrt[3]{6}} \right)$, temos

\[
\begin{gathered}
H_f \left( {\sqrt[3]{6},\sqrt[3]{6}} \right) = \left( {\begin{array}{*{20}c}
2 & 1  \\
1 & 2  \\

\end{array} } \right) \hfill \\
\det H_f \left( {\sqrt[3]{6},\sqrt[3]{6}} \right) = 3 > 0 \hfill \\
f_{aa} \left( {\sqrt[3]{6},\sqrt[3]{6}} \right) = 2 > 0 \hfill \\
\end{gathered}
\]

Portanto, $\left( {\sqrt[3]{6},\sqrt[3]{6}} \right)$ \'e m\'inimo de $f$.

    Portanto, as dimens\~oes s\~ao $\left( {\sqrt[3]{6},\sqrt[3]{6},6^{{\raise0.5ex\hbox{$\scriptstyle { - 2}$}
\kern-0.1em/\kern-0.15em
\lower0.25ex\hbox{$\scriptstyle 3$}}} } \right)$.
\end{sol}

\section{M\'aximos e M\'inimos sobre Conjunto Compacto} \label{sec22}

\begin{defn}
    $A \subset \R^n$ \'e \textit{limitado}\index{Conjunto!limitado} se existe bola $B_\varepsilon  \left( {x_0 } \right)$ de $\R^n$ tal que $A \subset B_\varepsilon  \left( {x_0 } \right)$.
\end{defn}

\begin{defn}
$A \subset \R^n$ \'e \textit{fechado} se o seu complementar \'e aberto.
\end{defn}

\begin{defn}
$A \subset \R^n$ \'e \textit{compacto}\index{Conjunto!compacto} se for fechado e limitado.
\end{defn}

\begin{teo}
    Sejam $A \subset \R^n$ compacto e $f:A \to \R$ fun\c c\~ao cont\'inua. Ent\~ao existem $x_1$ e $x_2$ em $A$ tais que

\[
    f\left( {x_1 } \right) \leqslant f\left( x \right) \leqslant f\left( {x_2 } \right),\forall x \in A
\]

\end{teo}

\begin{dem}
\begin{enumerate}[(i)]
      \item $A \subset \R^n$ \'e fechado e $x_0$ \'e acumula\c c\~ao de $A$, ent\~ao, $x_0 \in A$;
      \item Se $f$ \'e cont\'inua em $x_0$, existe $B_\varepsilon  \left( {x_0 } \right)$ tal que $f\left( {B_\varepsilon  \left( {x_0 } \right)} \right)$ \'e limitado;
      \item Se $R_i ,i \in \mathbb{N}$ \'e sequ\^encia de ret\^angulos encaixantes em $\R^n$, isto \'e, $R_1  \supset R_2  \supset ...$ e volume $R_i  \to 0$ se $i \to \infty$, ent\~ao, $\bigcap\limits_{i = 1}^\infty  {R_i }  = \left\{ {x_0 } \right\}$, isto \'e, $x_0$ \'e o \'unico ponto em todos $R_i$.
      \item Sejam $A \subset \R^n$ compacto e $f:A \to \R$ cont\'inua. Ent\~ao, $f$ \'e limitada em $A$;

(Dica: Suponha $f$ n\~ao limitada em $A \cap R_i$)
\item Conclua o Teorema.
\end{enumerate}
\end{dem}

\begin{ex}
$f\left( {x,y,z} \right) = x^3  + y^3  - 3x - 3y$ definida em \\
$A = \left\{ {\left( {x,y} \right) \in \R^2 :0 \leqslant x \leqslant 2{\text{ e }}\left| y \right| \leqslant 2} \right\}$.
\end{ex}

\begin{sol}
$A$ compacto e $f$ cont\'inua, implica que $f$ assume m\'aximo e m\'inimo (globais) em $A$.

Pontos cr\'iticos de $f$ no interior:

\[
    \nabla f = \left( {0,0} \right) \Rightarrow x =  \pm 1,y =  \pm 1,x =  - 1 \notin A
\]

Portanto, pontos $\left( {1,1} \right);\left( {1, - 1} \right)$ em $A$.

\[
\begin{gathered}
f\left( {1,1} \right) =  - 4{\text{ min local}} \hfill \\
f\left( {1, - 1} \right) = 0{\text{ ponto de sela}} \hfill \\
\end{gathered}
\]

Na fronteira de $A$:

\[
\begin{gathered}
      A_1  = \left\{ {\left( {x,2} \right):0 \leqslant x \leqslant 2} \right\} \cup  \hfill \\
      A_2  = \left\{ {\left( {x, - 2} \right):0 \leqslant x \leqslant 2} \right\} \cup  \hfill \\
      A_3  = \left\{ {\left( {0,y} \right): - 2 \leqslant y \leqslant 2} \right\} \cup  \hfill \\
      A_4  = \left\{ {\left( {2,y} \right): - 2 \leqslant y \leqslant 2} \right\} \hfill \\
\end{gathered}
\]

Em $A_1$:

\[
\begin{gathered}
      g\left( x \right) = f\left( {x,2} \right),x \in \left[ {0,2} \right] \hfill \\
g\left( x \right) = x^3  - 3x + 2 \hfill \\
      g'\left( x \right) = 3x^2  - 3 = 0 \Leftrightarrow x =  \pm 1 \Rightarrow x = 1{\text{ \'e ponto cr\'itico}} \hfill \\
      g''\left( x \right) = 6x \Rightarrow g''\left( 1 \right) = 6 > 0 \Rightarrow 1{\text{ \'e min local de $g$}} \hfill \\
\end{gathered}
\]

\[
\begin{gathered}
g\left( 1 \right) = 0{\text{ min local}} \hfill \\
g\left( 0 \right) = 2 \hfill \\
g\left( 2 \right) = 4{\text{ max local}} \hfill \\
\end{gathered}
\]

Repetir para $A_2, A_3$ e $A_4$ e comparar os valores de $f$.

\[
\begin{gathered}
      \left( {1,1} \right){\text{ e }}\left( {1, - 2} \right){\text{ m\'inimo }}\left( {f =  - 4} \right) \hfill \\
      \left( {2, - 1} \right){\text{ e }}\left( {2,2} \right){\text{ m\'aximo }}\left( {f = 4} \right) \hfill \\
\end{gathered}
\]

\end{sol}

\begin{ex}
    $f\left( {x,y} \right) = xy$em $A = \left\{ {\left( {x,y} \right) \in \R^2 :x^2  + y^2  \leqslant 1} \right\}$.
\end{ex}

\begin{sol}
No interior:

\[
\begin{gathered}
      \nabla f = \left( {y,x} \right) = \left( {0,0} \right) \Leftrightarrow \left( {x,y} \right) = \left( {0,0} \right) \hfill \\
f\left( {0,0} \right) = 0{\text{ sela}} \hfill \\
\end{gathered}
\]

Na fronteira:

$A_1  = \left\{ {\left( {x,y} \right):x^2  + y^2  = 1} \right\}$

    Tome $\gamma \left( t \right) = \left( {\cos t,\sin t} \right),t \in \left[ {0,2\pi } \right]$

Parametrize $A_1$

\[
\begin{gathered}
  \left( {f \circ \gamma } \right) = f\left( {\gamma \left( t \right)} \right) = f\left( {\cos t,\sin t} \right) = \cos t.\sin t,t \in \left[ {0,2\pi } \right] \hfill \\
  g'\left( t \right) = \cos \left( {2t} \right) = 0 \Leftrightarrow t = {\raise0.5ex\hbox{$\scriptstyle \pi $}
\kern-0.1em/\kern-0.15em
\lower0.25ex\hbox{$\scriptstyle 4$}}{\text{ e }}t = {\raise0.5ex\hbox{$\scriptstyle {3\pi }$}
\kern-0.1em/\kern-0.15em
\lower0.25ex\hbox{$\scriptstyle 4$}} \hfill \\
g''\left( t \right) =  - 2\sin \left( {2t} \right) \hfill \\
g''\left( {{\raise0.5ex\hbox{$\scriptstyle \pi $}
\kern-0.1em/\kern-0.15em
\lower0.25ex\hbox{$\scriptstyle 4$}}} \right) =  - 2 < 0 \Rightarrow {\raise0.5ex\hbox{$\scriptstyle \pi $}
\kern-0.1em/\kern-0.15em
\lower0.25ex\hbox{$\scriptstyle 4$}}{\text{ e max local}} \hfill \\
g''\left( {{\raise0.5ex\hbox{$\scriptstyle {3\pi }$}
\kern-0.1em/\kern-0.15em
\lower0.25ex\hbox{$\scriptstyle 4$}}} \right) = 2 > 0 \Rightarrow {\raise0.5ex\hbox{$\scriptstyle {3\pi }$}
\kern-0.1em/\kern-0.15em
\lower0.25ex\hbox{$\scriptstyle 4$}}\;{\text{e min local}} \hfill \\
g\left( 0 \right) = 0 = g\left( {2\pi } \right) \hfill \\
g\left( {{\raise0.5ex\hbox{$\scriptstyle \pi $}
\kern-0.1em/\kern-0.15em
\lower0.25ex\hbox{$\scriptstyle 4$}}} \right) = {\raise0.5ex\hbox{$\scriptstyle 1$}
\kern-0.1em/\kern-0.15em
\lower0.25ex\hbox{$\scriptstyle 2$}} \hfill \\
g\left( {{\raise0.5ex\hbox{$\scriptstyle {3\pi }$}
\kern-0.1em/\kern-0.15em
\lower0.25ex\hbox{$\scriptstyle 4$}}} \right) =  - {\raise0.5ex\hbox{$\scriptstyle 1$}
\kern-0.1em/\kern-0.15em
\lower0.25ex\hbox{$\scriptstyle 2$}} \hfill \\
\left( {0,0} \right){\text{ sela}} \hfill \\
\left( {\tfrac{{\sqrt 2 }}
{2},\tfrac{{\sqrt 2 }}
{2}} \right){\text{ max global}} \hfill \\
\left( { - \tfrac{{\sqrt 2 }}
{2},\tfrac{{\sqrt 2 }}
{2}} \right){\text{ min global}} \hfill \\
\end{gathered}
\]

\end{sol}

Se a fronteira de $A$ \'e mais "complicada"? (Algo como a curva de n\'ivel de uma fun\c c\~ao de classe $C^1$).

\begin{teo}[Multiplicador de Lagrange]
    Sejam $A \subset \R^n$ aberto e $B = \left\{ {x \in A:g\left( x \right) = 0} \right\}$, onde $g:A \to \R$ \'e de classe $C^1$, com $\nabla g \ne 0,\forall x \in B$. Se $f:A \to \R$ \'e diferenci\'avel e tem extremo em $x_0  \in B$, ent\~ao existe $\lambda_0 \in \R$ tal que

\[
\nabla f\left( {x_0 } \right) = \lambda _0 \nabla g\left( {x_0 } \right)
\]

$\lambda _0$ \'e chamado \textit{multiplicador de Lagrange}\index{Multiplicador de Lagrange}.

    \textbf{Obs:} $B$ \'e fechado, ent\~ao n\~ao necessariamente $\nabla f\left( {x_0 } \right) = 0$ se $x_0$ for extremo de $f$.

\end{teo}

\begin{dem}
Faremos $n = 2$.

    Suponha que $x_0$ \'e m\'aximo de $f$ sobre $B$, ou seja, existe $B_\varepsilon  \left( {x_0 } \right)$ tal que $f\left( x \right) \leqslant f\left( {x_0 } \right)$ para $x \in B_\varepsilon  \left( {x_0 } \right) \cap B$, isto \'e, $x \in B_\varepsilon  \left( {x_0 } \right)$ e $g(x) = 0$.

\[
\nabla g\left( x \right) \ne 0,\forall x \in B
\]

    Pelo Teorema da Fun\c c\~ao Impl\'icita \ref{sec17}, existe $\gamma :B_{\delta _1 } \left( {t_0 } \right) \to \R^2$ com $\gamma \left( {t_0 } \right) = x_0$ e $\gamma '\left( {t_0 } \right) \ne \overrightarrow 0$ e $\gamma \left( {B_{\delta _1 } \left( {t_0 } \right)} \right) \subset B$, ou seja, $g\left( {\gamma \left( {B_{\delta _1 } \left( {t_0 } \right)} \right)} \right) = 0$.

    $\gamma$ \'e cont\'inua, isto implica que, existe $B_\delta  \left( {t_0 } \right)$ tal que $f\left( {\gamma \left( {t_0 } \right)} \right) \geqslant f\left( {\gamma \left( t \right)} \right),\forall t \in B_\delta  \left( {t_0 } \right)$.

\[
\begin{gathered}
\Rightarrow \left. {\frac{d}
{{dt}}f \circ \gamma } \right|_{t = t_0 }  = 0 \hfill \\
   \Rightarrow \left\langle {\nabla f\left( {\underbrace {\gamma \left( {t_0 } \right)}_{x_0 }} \right),\gamma '\left( {t_0 } \right)} \right\rangle  = 0 \hfill \\
   \Rightarrow \nabla f\left( {x_0 } \right) \bot \gamma '\left( {t_0 } \right){\text{ e }}\nabla g\left( {x_0 } \right) \bot \gamma '\left( {t_0 } \right) \hfill \\
   \Rightarrow \nabla f\left( {x_0 } \right)\parallel \nabla g\left( {x_0 } \right) \hfill \\
   \Rightarrow \nabla f\left( {x_0 } \right) = \lambda _0 \nabla g\left( {x_0 } \right) \hfill \\
\end{gathered}
\]

\end{dem}

% figura 45

Ent\~ao, o Teorema diz que os extremos de $f$ sobre $B$ est\~ao entre as solu\c c\~oes do sistema:

\[
\left\{ \begin{gathered}
\nabla f\left( x \right) = \lambda \nabla g\left( x \right) \hfill \\
g\left( x \right) = 0 \hfill \\
\end{gathered}  \right.
\]

\begin{ex}
    Determine m\'aximo e m\'inimo de $f\left( {x,y} \right) = y + x^3$ sujeito a \\ $y - x^3  = 0$.
\end{ex}

\begin{sol}
Seja $g\left( {x,y} \right) = y - x^3$

\[
\begin{gathered}
\nabla f = \left( {3x^2 ,1} \right) \hfill \\
\nabla g = \left( { - 3x^2 ,1} \right) \hfill \\
\left\{ \begin{gathered}
\left( {3x^2 ,1} \right) = \lambda \left( { - 3x^2 ,1} \right) \hfill \\
y - x^3  = 0 \hfill \\
\end{gathered}  \right. \hfill \\
\Rightarrow \left\{ \begin{gathered}
3x^2  =  - \lambda 3x^2  \hfill \\
1 = \lambda  \hfill \\
x^3  = y \hfill \\
\end{gathered}  \right. \Rightarrow x = y = 0 \hfill \\
\end{gathered}
\]

    A solu\c c\~ao do sistema \'e $\left( {x,y} \right) = \left( {0,0} \right)$, que n\~ao \'e m\'aximo nem m\'inimo de $f$ sobre $g\left( {x,y} \right) = 0$, pois $f\left( {0,0} \right) = 0$ e $f\left( {x,x^3 } \right) = 2x^3$ pode ser maior ou menor do que $0$.
\end{sol}

\begin{ex}
Determine a tangente \`a elipse $x^2  + \frac{{y^2 }}
{4} = 1;x \geqslant 0;y \geqslant 0$ que forma um tri\^angulo com os eixos de \'area m\'axima.
\end{ex}

\begin{sol}
Considere a tangente \`a elipse no ponto $(a,b)$

% figura 46

\[
\begin{gathered}
r = \left( {a,b} \right) + t\left( { - g_y ,g_x } \right) \hfill \\
r = \left( {a,b} \right) + t\left( { - \tfrac{b}{2},2a} \right) \hfill \\
g\left( {x,y} \right) = x^2  + \frac{{y^2 }}{4} - 1 \hfill \\
\end{gathered}
\]

\'Area do tri\^angulo: $A\left( {a,b} \right) = \displaystyle \frac{2}{{ab}}$

\[
\begin{gathered}
\left\{ \begin{gathered}
\nabla A = \lambda \nabla g \hfill \\
g\left( {a,b} \right) = 0 \hfill \\
\end{gathered}  \right. \hfill \\
\Rightarrow \left\{ \begin{gathered}
\left( {\frac{{ - 2}}
{{a^2 b}},\frac{{ - 2}}
{{ab^2 }}} \right) = \lambda \left( {2a,\frac{b}
{2}} \right) \hfill \\
a^2  + \frac{{b^2 }}
{4} = 1 \hfill \\
\end{gathered}  \right. \hfill \\
\Rightarrow \left\{ \begin{gathered}
- 2 = 2\lambda a^3 b \hfill \\
- 4 = \lambda ab^3  \hfill \\
a^2  + \frac{{b^2 }}
{4} = 1 \hfill \\
\end{gathered}  \right. \Rightarrow a = \frac{{\sqrt 2 }}
{2}{\text{ e }}b = \sqrt 2  \hfill \\
\end{gathered}
\]

A equa\c c\~ao da reta \'e $\tfrac{{\sqrt 2 }}{2}x + \tfrac{{\sqrt 2 }}{4}y = 1$.

\end{sol}

\begin{ex}
    Determine o ponto do elips\'oide $x^2  + 2y^2  + 3z^2  = 1$ cuja soma das coordenadas \'e m\'axima.
\end{ex}

\begin{sol}
    Queremos maximizar $f\left( {x,y,z} \right) = x + y + z$ com a restri\c c\~ao $x^2  + 2y^2  + 3z^2  = 1$.

\[
\left\{ \begin{gathered}
  \nabla f\left( {x,y,z} \right) = \lambda \nabla g\left( {x,y,z} \right) \hfill \\
g\left( {x,y,z} \right) = 0 \hfill \\
\end{gathered}  \right. \Rightarrow \left\{ \begin{gathered}
\left( {1,1,1} \right) = \lambda \left( {2x,4y,6z} \right) \hfill \\
\underbrace {x^2  + 2y^2  + 3z^2  - 1}_{g\left( {x,y,z} \right)} = 0 \hfill \\
\end{gathered}  \right.
\]

    Como $\lambda$ deve ser diferente de zero, da $1$\textordfeminine\ equa\c c\~ao tiramos: $x = \frac{1}{{2\lambda }},y = \frac{1}{{4\lambda }}$ e $z = \frac{1}{{6\lambda }}$. Substituindo na \'ultima equa\c c\~ao obtemos:

\[
    \frac{1}{{4\lambda ^2 }} + \frac{2}{{16\lambda ^2 }} + \frac{3}{{36\lambda ^2 }} = 1{\text{ ou }}\lambda  =  \pm \sqrt {\tfrac{{11}}{{24}}}
\]

Os candidatos a extremantes s\~ao:

\[
x_1  = \left( {\frac{1}
{2}\sqrt {\frac{{11}}
{{24}}} ,\frac{1}
{4}\sqrt {\frac{{11}}
{{24}}} ,\frac{1}
{6}\sqrt {\frac{{11}}
{{24}}} } \right){\text{ e }}x_2  = \left( { - \frac{1}
{2}\sqrt {\frac{{11}}
{{24}}} , - \frac{1}
{4}\sqrt {\frac{{11}}
{{24}}} , - \frac{1}
{6}\sqrt {\frac{{11}}
{{24}}} } \right)
\]

    Da compacidade de $B$, da continuidade de $f$ e de $f\left( {x_1 } \right) > f\left( {x_2 } \right)$ segue que o ponto procurado \'e

\[
\left( {\frac{1}
{2}\sqrt {\frac{{11}}
{{24}}} ,\frac{1}
{4}\sqrt {\frac{{11}}
{{24}}} ,\frac{1}
{6}\sqrt {\frac{{11}}
{{24}}} } \right)
\]

\end{sol}

\begin{teo}
    Seja $A \subset \R^3$ aberto e $B = \left\{ {\left( {x,y,z} \right) \in A:g\left( {x,y,z} \right) = 0{\text{ e }}h\left( {x,y,z} \right) = 0} \right\}$ onde $g$ e $h$ s\~ao fun\c c\~oes de classe $C^1$ em $A$ com $\left\{ {\nabla g,\nabla h} \right\}$ linearmente independente (L.I.) para todo $x \in B$. Se $x_0 \in B$ \'e ponto extremo de uma fun\c c\~ao diferenci\'avel $f:A \to \R$, restrita a $B$, ent\~ao existem $\lambda _0 ,\mu _0$ reais tais que

\[
    \nabla f\left( {x_0 } \right) = \lambda _0 \nabla g\left( {x_0 } \right) + \mu _0 \nabla h\left( {x_0 } \right)
\]

\end{teo}

\begin{dem}
Suponhamos que $\left( {x_0 ,y_0 ,z_0 } \right)$ seja ponto de m\'aximo local de $f$ em $B$, o que significa que existe uma bola aberta $V$ de centro $\left( {x_0 ,y_0 ,z_0 } \right)$ tal que, para todo $\left( {x,y,z} \right) \in B \cap V$,

\[
f\left( {x,y,z} \right) \leqslant f\left( {x_0 ,y_0 ,z_0 } \right)
\]

(como $A$ \'e aberto, podemos supor $V \subset A$). Consideremos uma curva diferenci\'avel $\gamma :I \to \R^3$, $I$ intervalo aberto, tal que $\gamma \left( {t_0 } \right) = \left( {x_0 ,y_0 ,z_0 } \right),\gamma '\left( {t_0 } \right) \ne \overrightarrow 0$ e $\gamma \left( t \right) \in B$ para todo $t$ em $I$ (a exist\^encia de uma tal curva \'e garantida pelo teorema das fun\c c\~oes impl\'icitas). Da continuidade de $\gamma$, segue que existe $\delta > 0$ tal que

\[
t \in \left] {t_0  - \delta ,t_0  + \delta } \right[ \Rightarrow \gamma \left( t \right) \in B \cap V
\]

Assim, para todo $t \in \left] {t_0  - \delta ,t_0  + \delta } \right[$ tem-se

\[
f\left( {\gamma \left( t \right)} \right) \leqslant f\left( {\gamma \left( {t_0 } \right)} \right)
\]

Logo, $t_0$ \'e ponto m\'inimo de m\'aximo local de $F\left( t \right) = f\left( {\gamma \left( t \right)} \right)$ e da\'i $F'\left( {t_0 } \right) = 0$, ou seja,

\[
\left( 1 \right)\nabla f\left( {\gamma \left( {t_0 } \right)} \right).\gamma '\left( {t_0 } \right) = 0
\]

Por outro lado, de $\gamma \left( t \right) \in B$ para todo $t \in I$ seque que

\[
g\left( {\gamma \left( t \right)} \right) = 0{\text{ e }}h\left( {\gamma \left( t \right)} \right) = 0
\]

para todo $t$ em $I$; da\'i

\[
\left( 2 \right)\nabla g\left( {\gamma \left( {t_0 } \right)} \right).\gamma '\left( {t_0 } \right) = 0{\text{ e }}\nabla h\left( {\gamma \left( {t_0 } \right)} \right).\gamma '\left( {t_0 } \right) = 0
\]

De $(1)$ e $(2)$, tendo em vista que $\gamma '\left( {t_0 } \right) \ne \overrightarrow 0$ e $\nabla g\left( {\gamma \left( {t_0 } \right)} \right) \wedge \nabla h\left( {\gamma \left( {t_0 } \right)} \right) \ne \overrightarrow 0$ resulta que existem reais $\lambda _0$ e $\mu _0$ tais que

\[
\nabla f\left( {\gamma \left( {t_0 } \right)} \right) = \lambda _0 \nabla g\left( {\gamma \left( {t_0 } \right)} \right) + \mu _0 \nabla h\left( {\gamma \left( {t_0 } \right)} \right)
\]


% inserir figura da pag 916 vol 2
\end{dem}


\chapter{Fun\c c\~oes de V\'arias Vari\'aveis Reais a Valores Vetoriais} \label{chap05}

\section{Fun\c c\~oes de V\'arias Vari\'aveis Reais a Valores Vetoriais} \label{sec23}

\begin{defn}
Seja $A \subset \R^n$ uma fun\c c\~ao $f:A \subset \R^n \to \R^m$ \'e uma regra que associa um \'unico vetor $f(x)$ de $\R^m$ a cada vetor $x$ de $A \subset \R^n$ se $x = \left( {x_1 ,...,x_n } \right)$ escrevemos

\[
\begin{gathered}
  f:A \to \R^m  \hfill \\
  \left( {x_1 ,...,x_n } \right) \to f\left( {x_1 ,...,x_n } \right) = \left( {f^1 \left( {x_1 ,...,x_n } \right),f^2 \left( {x_1 ,...,x_n } \right),...,f^m \left( {x_1 ,...,x_n } \right)} \right) \hfill \\
\end{gathered}
\]

\end{defn}

\begin{ex}
$f:\R^n  \to \R^m$ dada por $f\left( {x,y} \right) = \left( {x,y,x^2  + y^2 } \right)$ \'e uma fun\c c\~ao com dom\'inio $\R^2$ e com valores em $\R^3$. Esta fun\c c\~ao transforma o par ordenado $(x,y)$ na terna $(x,y,x^2+y^2)$. A imagem de $f$ \'e o conjunto $\left\{ {\left( {x,y,x^2  + y^2 } \right):\left( {x,y} \right) \in \R^2 } \right\}$ que \'e igual a $\left\{ {\left( {x,y,z} \right) \in \R^3 :z = x^2  + y^2 ,\left( {x,y} \right) \in \R^2 } \right\}$. A imagem de $f$ coincide, ent\~ao, com o gr\'afico da fun\c c\~ao dada por $z=x^2+y^2$.

%inserir figura da pag 2 Guidorizzi vol 3.
\end{ex}

\begin{ex}
\textit{(Coordenadas polares.)} Seja a fun\c c\~ao $f:\R^2  \to \R^2$ dada por $\left( {r,\theta } \right) \mapsto f\left( {r,\theta } \right) = \left( {r\cos \theta ,r\sin \theta } \right)$, onde $r\cos \theta = x$ e $r\sin \theta = y$.

\tkzfig{fig47}{$f$ transforma a reta $r$ na circunfer\^encia $(r \cos \theta, r \sin \theta)$}
\end{ex}

\section{Campo Vetorial} \label{sec24}

\begin{defn}
Um \textit{campo vetorial}\index{Campo!vetorial} em $A \subset \R^n$ \'e uma fun\c c\~ao $f:A \to \R^n$.
\end{defn}

% figura 48 pag 7 vol 3

\begin{ex}
Represente geometricamente o campo vetorial dado por $f(x,y)=(x,y)$.
\end{ex}

\begin{sol}
$\left\| {f\left( {x,y} \right)} \right\| = \sqrt {x^2  + y^2 }$; segue que a intensidade do campo \'e a mesma nos pontos de uma mesma circunfer\^encia de centro na origem. Observe que a intensidade do campo no ponto $(x,y)$ \'e igual ao raio da circunfer\^encia, de centro na origem, que passa por este ponto.

% figura 49 pag 8
\end{sol}

\begin{defn}
Seja $f:A \subset \R^n  \to \R^m$ \'e cont\'inua em $x_0$ se dada $B_\varepsilon  \left( {f\left( {x_0 } \right)} \right)$ existe $B_g \left( {x_0 } \right)$ tal que $f\left( {B_g \left( {x_0 } \right)} \right) \subset B_\varepsilon  \left( {f\left( {x_0 } \right)} \right)$.
\end{defn}

\begin{teo}
Sejam $x_0 \in A \subset \R^n$ e $f:A \to \R^m$, $f\left( x \right) = \left( {f^1 \left( x \right),...,f^m \left( x \right)} \right)$. Ent\~ao, $f$ \'e cont\'inua em $x_0$ se, e somente se, $f^i$ \'e cont\'inua em $x_0, \forall i = 1,...,m$.
\end{teo}

\begin{defn}
Seja $f:A \subset \R^n  \to \R^m$, com $A$ aberto. Dizemos que $f$ \'e diferenci\'avel em $x_0 \in A$ se existe $L_{x_0 } :\R^n  \to \R^m$, aplica\c c\~ao linear tal que

\[
\mathop {\lim }\limits_{\left\| h \right\| \to 0} \frac{{f\left( {x_0  + h} \right) - f\left( {x_0 } \right) - L_{x_0 } \left( h \right)}}
{{\left\| h \right\|}} = 0
\]

\end{defn}

\begin{teo}
Se $f$ \'e diferenci\'avel em $x_0$, ent\~ao, $L_{x_0}$ \'e \'unica e sua matriz na base can\^onica \'e

\[
\left[ {\frac{{\partial f^j }}
{{\partial x_i }}\left( {x_0 } \right)} \right]_{m \times n}
\]

com $i = 1,...,n$ e $j = 1,...,m$.
\end{teo}


\section{Rotacional} \label{sec25}

\begin{defn}
\begin{sloppypar}
Seja $A \subset \R^3$ aberto e $f:A \to \R^3$ um campo diferenci\'avel, ${f\left( x \right) = \left( {P\left( x \right),Q\left( x \right),R\left( x \right)} \right)}$. O \textit{rotacional}\index{Rotacional} de $f$ \'e dado por
\end{sloppypar}

\[
{\text{rot }}f = \left| {\begin{array}{*{20}c}
   i & j & k  \\
   {\partial x} & {\partial y} & {\partial z}  \\
   P & Q & R  \\

 \end{array} } \right| = \left( {R_y  - Q_z ,P_z  - R_x ,Q_x  - P_y } \right)
\]

\end{defn}

\textbf{Obs:} Em $\R^2$, fa\c ca $R\left( x \right) \equiv 0$.

\begin{ex}
Seja $f\left( {x,y,z} \right) = \left( {xy,yz^2 ,xyz} \right)$. Calcule rot $f$.
\end{ex}

\begin{sol}
\[
{\text{rot }}f = \left| {\begin{array}{*{20}c}
   i & j & k  \\
   {\partial x} & {\partial y} & {\partial z}  \\
   {xy} & {yz^2 } & {xyz}  \\

 \end{array} } \right| = \left( {xz - 2yz,0 - yz,0 - x} \right)
\]

\end{sol}

\begin{ex}
Seja $f\left( {x,y} \right) = \left( {\cos y,\sin x} \right)$. Calcule rot $f$.
\end{ex}

\begin{sol}
\[
{\text{rot }}f = \left| {\begin{array}{*{20}c}
   i & j & k  \\
   {\partial x} & {\partial y} & {\partial z}  \\
   {\cos y} & {\sin x} & 0  \\

 \end{array} } \right| = \left( {0,0,\cos x + \sin y} \right)
\]

\end{sol}

\begin{defn}
Seja $f:A \subset \R^{2,3}  \to \R^{2,3}$ \'e \textit{irrotacional} se ${\text{rot }}f \equiv 0$.
\end{defn}

\begin{ex}
Seja $f:\R^2  \to \R^2$ tal que $\left( {x,y} \right) \mapsto \left( {0,Q\left( {x,y} \right)} \right)$ onde $\frac{{\partial Q}}{{\partial x}} = 0$.

Desenhe um campo satisfazendo as condi\c c\~oes dadas e calcule rot $f$.
\end{ex}

\begin{sol}
% figura 50

\[
{\text{rot }}f = \left| {\begin{array}{*{20}c}
   i & j & k  \\
   {\partial x} & {\partial y} & {\partial z}  \\
   0 & Q & 0  \\

 \end{array} } \right| = \left( {0,0,0} \right)
\]

\end{sol}

\begin{ex}
Se $f$ \'e o campo de velocidades no escoamento de um flu\'ido em $\R^2$ para cada $y_0$ fixado e $x > x_0$ temos, $\left\| {f\left( {x_0 ,y_0 } \right)} \right\| \leqslant \left\| {f\left( {x,y_0 } \right)} \right\|$, ent\~ao, um disco giraria no sentido anti-hor\'ario se $Q\left( {x,y} \right) > 0$.
\end{ex}

\begin{ex}
Considere $v = \left( {P\left( {x,y} \right),Q\left( {x,y} \right)} \right)$ de classe $C^1$, velocidade de um flu\'ido bidimensional. Sejam $A$ e $B$ part\'iculas do flu\'ido.
\end{ex}

\begin{sol}
% figura 51

\[
\begin{gathered}
  \delta \left( t \right) = \left\| {A\left( t \right) - B\left( t \right)} \right\| \Rightarrow \delta \left( 0 \right) = h \hfill \\
  A\left( t \right) = \left( {x_1 \left( t \right),y_1 \left( t \right)} \right);B\left( t \right) = \left( {x_2 \left( t \right),y_2 \left( t \right)} \right) \hfill \\
  y_2 \left( t \right) - y_1 \left( t \right) = \delta \left( t \right)\sin \theta _h \left( t \right) \hfill \\
  y_2 '\left( t \right) - y_1 '\left( t \right) = \delta '\left( t \right)\sin \theta _h \left( t \right) + \delta \left( t \right)\cos \theta _h \left( t \right).\theta _h '\left( t \right) \hfill \\
\end{gathered}
\]

Em $t = 0,\theta _h \left( 0 \right) = 0,\delta \left( 0 \right) = h$, temos,

\[
\begin{gathered}
  y_2 '\left( 0 \right) = Q\left( {x_0  + h,y_0 } \right) \hfill \\
  y_1 '\left( 0 \right) = Q\left( {x_0 ,y_0 } \right) \hfill \\
\end{gathered}
\]

Ent\~ao,

\[
\begin{gathered}
  \theta _h '\left( 0 \right) = \frac{{Q\left( {x_0  + h,y_0 } \right) - Q\left( {x_0 ,y_0 } \right)}}
{h} \hfill \\
  \mathop {\lim }\limits_{h \to 0} \theta _h '\left( 0 \right) = \frac{{\partial Q}}
{{\partial x}}\left( {x_0 ,y_0 } \right) \hfill \\
\end{gathered}
\]

Se o movimento \'e r\'igido e com velocidade angular constante $\omega$, ent\~ao,

\[
\omega  = \frac{{\partial Q}}{{\partial x}}\left( {x_0 ,y_0 } \right)
\]

Analogamente, para $c(t)$, com $c\left( 0 \right) = \left( {x_0 ,y_0  + k} \right)$, temos

\[
\omega  = \mathop {\lim }\limits_{k \to 0} \varphi _k '\left( 0 \right) =  - \frac{{\partial P}}
{{\partial y}}\left( {x_0 ,y_0 } \right)
\]

Ent\~ao,

\[
2\omega  = \frac{{\partial Q}}
{{\partial x}}\left( {x_0 ,y_0 } \right) - \frac{{\partial P}}
{{\partial y}}\left( {x_0 ,y_0 } \right) = \left\langle {{\text{rot }}v,e_3 } \right\rangle
\]

\'e o m\'odulo do rotacional.
\end{sol}

\newpage 

\begin{ex}
$\displaystyle f\left( {x,y} \right) = \left( {\frac{{ - x}}{{x^2  + y^2 }},\frac{{ - y}}{{x^2  + y^2 }}} \right)$
\end{ex}

\begin{sol}
$\left\| f \right\|^2  = \displaystyle \frac{1}{{x^2  + y^2 }}$, que \'e constante nos c\'irculos.

rot $f = 0$. Tem velocidade angular nula ao longo de retas.
\end{sol}

\begin{ex}
Seja $f\left( {x,y} \right) = \left( { - y,x} \right)$. Calcule o rotacional.
\end{ex}

\begin{sol}
\'E tangente a circunfer\^encia de raio $\sqrt {x^2  + y^2 }$. As part\'iculas descrevem essas circunfer\^encias.

% figura 52

\[
\begin{gathered}
  {\text{rot }}f = \left| {\begin{array}{*{20}c}
   i & j & k  \\
   {\partial x} & {\partial y} & {\partial z}  \\
   { - y} & x & 0  \\

 \end{array} } \right| = \left( {0,0,2} \right) \hfill \\
   \Rightarrow \omega  = 1 \hfill \\
\end{gathered}
\]

\end{sol}

\section{Divergente} \label{sec26}

\begin{defn}
Seja $f:A \subset \R^n  \to \R^n$ um campo diferenci\'avel. O \textit{divergente}\index{Divergente} $f$ \'e dado por

\[
{\text{div }}f = \sum\limits_{i = 1}^n {\frac{{\partial f^i }}
{{\partial x_i }}\left( x \right)}  \in \R
\]

\end{defn}

\textbf{Obs:} Seja $f:A \subset \R^n  \to \R$ de classe $C^2$. $\nabla f$ \'e um campo diferenci\'avel, ent\~ao,

\[
\begin{gathered}
  \nabla f = \left( {\frac{{\partial f}}
{{\partial x_1 }},...,\frac{{\partial f}}
{{\partial x_n }}} \right) \hfill \\
  {\text{div }}\left( {\nabla f} \right) = \frac{{\partial ^2 f}}
{{\partial x_1^2 }} + \frac{{\partial ^2 f}}
{{\partial x_2^2 }} + ... + \frac{{\partial ^2 f}}
{{\partial x_n^2 }} = \Delta f{\text{ }}\left( {{\text{Laplaciano}}} \right) \hfill \\
\end{gathered}
\]

\begin{defn}
Seja $f:B \to \R^m ,B \subset \R^n$ fechado \'e diferenci\'avel se existem $\Omega  \subset \R^n$ aberto, $B \subset \Omega$ e $g_i :\Omega  \to \R^m$ diferenci\'avel tal que $\left. g \right|_B  = f$ ($g$ restrito a $B$).
\end{defn}

\begin{ex}
Seja $f\left( {x,y} \right) = \left( {0,Q\left( {x,y} \right)} \right),\displaystyle \frac{{\partial Q}}{{\partial y}}\left( {x,y} \right) > 0$. Calcule o rotacional e o divergente.
\end{ex}

\begin{sol}
\[
\begin{gathered}
  {\text{rot }}f = \frac{{\partial Q}}
{{\partial x}} \hfill \\
  {\text{div }}f = \frac{{\partial Q}}
{{\partial y}} > 0 \hfill \\
\end{gathered}
\]

\end{sol}

\begin{ex}
$f\left( {x,y} \right) = \left( {P\left( {x,y} \right),Q\left( {x,y} \right)} \right)$ de classe $C^1$.
\end{ex}

\begin{sol}
% figura 53 pag 25

Seja $v\left( t \right)$ o volume da figura $A(t)B(t)C(t)D(t)$.

\[
\begin{gathered}
  v\left( 0 \right) = hk \hfill \\
  v\left( t \right) \approx \left\| {A\left( t \right) - B\left( t \right)} \right\|.\left\| {A\left( t \right) - C\left( t \right)} \right\| \hfill \\
  A\left( t \right) \approx \left( {x_0  + tP\left( {x_0 ,y_0 } \right),y_0  + tQ\left( {x_0 ,y_0 } \right)} \right) \hfill \\
  B\left( t \right) \approx \left( {x_0  + h + tP\left( {x_0  + h,y_0 } \right),y_0  + tQ\left( {x_0  + h,y_0 } \right)} \right) \hfill \\
  C\left( t \right) \approx \left( {x_0  + tP\left( {x_0 ,y_0  + k} \right),y_0  + k + tQ\left( {x_0 ,y_0  + k} \right)} \right) \hfill \\
  P\left( {x_0  + h,y_0 } \right) \approx P\left( {x_0 ,y_0 } \right) + hP_x \left( {x_0 ,y_0 } \right) \hfill \\
  Q\left( {x_0  + h,y_0 } \right) \approx Q\left( {x_0 ,y_0 } \right) + hQ_x \left( {x_0 ,y_0 } \right) \hfill \\
  P\left( {x_0 ,y_0  + k} \right) \approx P\left( {x_0 ,y_0 } \right) + kP_y \left( {x_0 ,y_0 } \right) \hfill \\
  Q\left( {x_0 ,y_0  + k} \right) \approx Q\left( {x_0 ,y_0 } \right) + kQ_y \left( {x_0 ,y_0 } \right) \hfill \\
  v\left( t \right) = \left\| {\overline {A\left( t \right)B\left( t \right)}  \wedge \overline {A\left( t \right)C\left( t \right)} } \right\| = ... =  \hfill \\
   = hk + hktP_x \left( {x_0 ,y_0 } \right) + hktQ_y \left( {x_0 ,y_0 } \right) + hkt^2 \left( {P_x Q_y  - P_y Q_x } \right)\left( {x_0 ,y_0 } \right) \hfill \\
  \frac{{v\left( t \right) - v\left( 0 \right)}}
{t} = hk\left( {P_x  + Q_y  + t\left( {P_x Q_y  - P_y Q_x } \right)} \right) \hfill \\
\end{gathered}
\]

fazendo $t \to 0$, temos:

\[
\begin{gathered}
  v'\left( 0 \right) = hk\left( {P_x \left( {x_0 ,y_0 } \right) + Q_y \left( {x_0 ,y_0 } \right)} \right) \hfill \\
   = hk.{\text{div }}f = v\left( 0 \right).{\text{div }}f \hfill \\
\end{gathered}
\]

\end{sol}


\chapter{Integrais Duplas} \label{chap06}

\section{A Integral de Riemann} \label{sec27}

\begin{defn}
Seja $R = \left[ {a_i ,b_i } \right]^n ,i = 1,...,n$ um ret\^angulo em $\R^n$. Uma \textit{parti\c c\~ao} $P$ de $R$ \'e uma escolha $x^k  = \left( {x_1^k ,...,x_n^k } \right),x_i^k  \in \left[ {a_i ,b_i } \right],x_{i - 1}^k  < x_i^k$.

\[
\begin{gathered}
  \left| P \right| = \max {\text{vol }}R_i  \hfill \\
  \left| P \right| \to 0 = {\text{vol }}R_i  \to 0,\forall i \hfill \\
\end{gathered}
\]

\end{defn}

\begin{defn}
Seja $f:R \subset \R^n  \to \R$ limitada ($\left| {f\left( x \right)} \right| < M,\forall x \in \R$). Definimos a \textit{soma inferior} e \textit{soma de Riemann} para $f$ relativa a uma parti\c c\~ao de $P$ de $R$ por

inferior: $s\left( {f,P} \right) = \sum\limits_{R_i }^{} {m_i {\text{vol }}R_i } ,m_i  = \inf f\left( x \right),x \in \R_i$

superior: $S\left( {f,P} \right) = \sum\limits_{R_i }^{} {M_i {\text{vol }}R_i } ,M_i  = \sup f\left( x \right),x \in \R_i$
\end{defn}

\textbf{Obs:}

\begin{enumerate}[(i)]
  \item $s\left( {f,P} \right) \leqslant S\left( {f,P} \right)$
  \item se $P'$ \'e refinamento de $P$, ent\~ao,

\[
s\left( {f,P} \right) \leqslant s\left( {f,P'} \right) \leqslant S\left( {f,P'} \right) \leqslant S\left( {f,P} \right)
\]

\end{enumerate}

\begin{defn}
Seja $f:R \subset \R^n \to \R$ limitada e $P$ parti\c c\~ao de $R$. Definimos a \textit{integral inferior} e \textit{superior} de $f$ sobre $R$ por

\[
\int_{-R} {f(x)dx} = \sup s(f,P),\,P \text{ parti\c c\~ao de } R
\]

e

\[
\int_{R}^- {f(x)dx} = \inf S(f,P),\,P \text{ parti\c c\~ao de } R
\]

\end{defn}

Dizemos que $f$ \'e integr\'avel sobre $R$ se $\int_{-R} {f(x)dx}  = \int_R^ -  {f(x)dx}$ e escrevemos $\int_R f(x) dx$.

\textbf{Obs}: $\int_{-R} f = \int_R^- f$ sempre existem e $\int_{-R} f \mei \int_R^- f$.

\begin{ex}
Seja $f:R \subset \R^n \to \R, f(x) = k$.

Seja $P$ parti\c c\~ao de $R$. $s(f,P) = \sum\limits_{R_i \in P}^{} {m_i {\text{vol }}R_i} = k \sum\limits_{R_i \in P}^{} {{\text{vol}}R_i} = k \text{vol } R$

\[
\begin{gathered}
   \Rightarrow \int_ -  {f(x)dx}  = \sup s(f,P) = k{\text{vol}}R \hfill \\
  S(f,P) = \sum\limits_{R_i  \in P}^{} {M_i {\text{vol}}R_i }  = k{\text{vol}}R \hfill \\
   \Rightarrow \int_R^ -  {f(x)dx}  = \inf S(f,P) = k{\text{vol}}R \hfill \\
   \Rightarrow \int_R {f(x)dx}  = k{\text{vol}}R \hfill \\ 
\end{gathered} 
\]

\end{ex}

\begin{ex}
Seja $f:[0,1]^2 = R \to \R$ tal que

\begin{equation*}
  f(x,y) =
  \begin{cases}
    \hfill 0 	& \mbox{, se } (x,y) \in \Q^2 \cap R \\
    \hfill 1 	& \mbox{, se } (x,y) \notin \Q^2 \cap R
  \end{cases}
\end{equation*}

\end{ex}

\begin{sol}
Seja $P$ parti\c c\~ao de $[0,1]^2$.

\[
\begin{gathered}
  s(f,P) = \SomaR[R_i \in P]{\mathop {\cancelto{m_i}{0}} \text{vol}R_i} = 0 \hfill \\
  \imp \int_{ - R} {f(x)dx}  = \sup s(f,P) = 0 \hfill \\
  S(f,P) = \SomaR[R_i \in P]{\mathop {\cancelto{M_i}{1}} \text{vol}R_i} = \text{vol}R = 1 \hfill \\
  \imp \int_R^ -  {f(x)dx}  = \inf S(f,P) = 1 \hfill \\
  \imp \int_{ - R} {f(x)dx}  \ne \int_R^ -  {f(x)dx}  \hfill \\ 
\end{gathered} 
\]

Portanto, $f$ n\~ao \'e Riemann integr\'avel.
\end{sol}

\begin{ex}
Seja $f:[0,1]^2 \to \R$ tal que

\begin{equation*}
  f(x,y) =
  \begin{cases}
    \hfill 1 	& \mbox{, se } x \ne y \\
    \hfill 0 	& \mbox{, se } x = y
  \end{cases}
\end{equation*}
\end{ex}

\begin{sol}
Dado $\varepsilon > 0$ considere ret\^angulos de centro $(x,x)$ e lados $\mfrac{\varepsilon}{\sqrt n}$.

Seja $P$ uma parti\c c\~ao de $[0,1]^2$ que cont\'em tais ret\^angulos.

\[
\begin{gathered}
  S(f,P) = \SomaR[R_j \ne R_i]{\mathop {\cancelto{M_j}{1}} \text{vol}R_j} + \SomaR[R_i]{\mathop {\cancel{M_i}}\nolimits^1 \text{vol}R_i} \hfill \\
   = \SomaR[R_j \ne R_i]{\text{vol}R_i} + n.\frac{{\varepsilon ^2 }}
{n} \hfill \\
   = \SomaR[R_j \ne R_i]{\text{vol}R_i} + \varepsilon ^2  \hfill \\
  s(f,P) = \SomaR[R_j \ne R_i]{\mathop {\cancelto{m_j}{1}} \text{vol}R_j} + \SomaR[R_i]{\mathop {\cancelto{m_i}{0}} \text{vol}R_i} = \SomaR[R_j \ne R_i]{\text{vol}R_j} \hfill \\ 
\end{gathered} 
\]

Se $P'$ \'e refinamento de $P$, temos:

\[
\begin{gathered}
  \SomaR[R_j \ne R_i]{\text{vol}R_j} \mei s(f,P') \mei S(f,P') \mei \SomaR[R_j \ne R_i]{\text{vol}R_j} + \varepsilon^2 \hfill \\
  \imp \left| {\int_{-} f - \int^{-} f} \right| < \varepsilon^2 ,\forall \varepsilon > 0 \hfill \\
  \imp \int_{-} f = \int^{-} f = \text{vol}R = 1 \hfill \\ 
\end{gathered} 
\]

\end{sol}

\begin{defn}
$A \subset \R^n$ tem \textit{conte\'udo nulo} se dado $\varepsilon > 0$ existem ret\^angulos $R_1,\ldots,R_n$ tais quem

\[
\Soma{\text{vol}R_i} < \varepsilon \qquad \text{e} \qquad A \subset \bigcup\limits_{i=1}^n {R_i}
\]
\end{defn}

\begin{teo}
Sejam $f:R \subset \R^n \to \R$ uma fun\c c\~ao limitada $D=\{ x \in R: f \text{ \'e descont\'i?ua em $x$} \}$. Ent\~ao, $f$ \'e Riemann integr\'avel se, e somente se, $D$ tem conte\'udo nulo.
\end{teo}

\begin{prop}
Sejam $f,g:R \subset \R^n \to \R$ fun\c c\~oes integr\'aveis. Ent\~ao

\begin{enumerate}[a)]
 \item $\int_R (f + g) = \int_R f + \int_R g$
 \item $\int_R kf = k \int_R f, k \in \R$
 \item $f \mai 0 \imp \int_R f \mai 0$
 \item $f \mei g \imp \int_R f \mei \int_R g$
\end{enumerate}

\end{prop}

\begin{dem}
Seja $P$ parti\c c\~ao de $R$. $m_i (f) = \mathop {\inf}\limits_{x \in R_i} f(x), R_i \in P$ e $M_i (f) = \mathop {\sup}\limits_{x \in R_i} f(x), R_i \in P$.

\begin{enumerate}[a)]
 \item Temos

\[
\begin{gathered}
  m_i (f) + m_i (g) \mei m_i (f + g) \mei M_i (f + g) \mei M_i (f) + M_i (g) \hfill \\
  \imp \int_{-} f  + \int_{-} g \mei \int_{-} {(f + g)} \mei \int^{-}  {(f + g)} \mei \int^{-}  f  + \int^{-}  g  \hfill \\
  \imp \int_{-} {(f + g)}  = \int^{-}  {(f + g)} \imp(f + g) \text{ \'e integr\'avel e } \int {(f + g)} = \int f + \int g. \hfill \\ 
\end{gathered} 
\]

 \item $m_i (kf) = k m_i (f)$ e $M_i (kf) = k M_i (f) \imp kf$ \'e integr\'avel e $\int kf = k \int f$.

 \item $f$ integr\'avel $\imp \int_{-} f = \int^{-} f$

\[
f \mai 0 \imp m_i (f) \mai 0 \text{ e } s(f,P) \mai 0 \imp \int_{-} f = \int f \mai 0
\]

 \item $f \mei g \imp g - f \mai 0$

\[
\begin{gathered}
  (c) \imp \int {(g - f) \mai 0}  \hfill \\
  (a) \imp \int g  + \int { - f}  \hfill \\
  (b) \imp \int g  - \int f \mai 0 \hfill \\
  \imp \int f \mei \int g  \hfill \\ 
\end{gathered} 
\]

\end{enumerate}

\end{dem}

\begin{defn}
Sejam $B \subset \R^n$ um conjunto limitado e $f:B \to \R$ limitada. Definimos $\int_B f = \int_R f.\mathcal{X}_B$, onde $R$ \'e o ret\^angulo de $\R^n, B \subset R$ e

\begin{equation*}
  \mathcal{X}_B (x) =
  \begin{cases}
    \hfill 1 	& \mbox{, se } x \in B \\
    \hfill 0 	& \mbox{, se } x \notin B
  \end{cases}
\end{equation*}
\end{defn}

\begin{prop}
Sejam $B \subset \R^n$ tem conte\'udo nulo e $f:B \to \R$ uma fun\c c\~ao limitada. Ent\~ao, $\int_B f = 0$.
\end{prop}

\begin{dem}
$B$ tem conte\'udo nulo implica que dado $\varepsilon > 0, B \subset \bigcup\limits_{i = 1}^n {R_i }$, com $\Soma{\text{vol}R_i} < \varepsilon$.

$B$ limitado implica que existe $R$ ret\^angulo de $\R^n$ tal que $B \subset R$.

\[
\begin{gathered}
  \left| {\int_B f} \right| = \left| {\int_R f.\mathcal{X}_B} \right| \hfill \\
  \left| {\int_B f} \right| \mei \int_R |f| \mei 0 + \sup f. \Soma{\text{vol}R_i} \mei \sup f. \varepsilon \hfill \\
  \imp \int_B f = 0 \hfill
\end{gathered} 
\]

\end{dem}

\begin{cor}
Sejam $f,g: B \subset \R^n \to \R$ e $D = \left\{ {x \in B:f(x) \ne g(x)} \right\}$. Se $D$ tem conte\'udo nulo, ent\~ao $\int_B f = \int_B g$.
\end{cor}

\begin{dem}

\[
\begin{gathered}
  \int_B (f-g) = \int_{B \setminus D} \mathop {\cancelto{\left( f-g \right)}{0}} + \int_D \mathop {\cancelto{\left( f-g \right)}{0}} = 0 \hfill \\
  \imp \int_B f = \int_B g \hfill \\ 
\end{gathered} 
\]

\end{dem}

\begin{defn}
Seja $A \subset \R^n$ um conjunto para o qual $\int_A 1$ exista, ent\~ao definimos $\text{vol}A = \int_A 1$.
\end{defn}

\begin{teo}[valor intermedi\'ario para integral]\index{Teorema!do valor intermedi\'ario para integral}
Sejam $B \subset \R^n$ compacto e conexo por caminhos e $f:B \to \R$ cont\'inua e integr\'avel em $B$. Ent\~ao existe $x_0 \in B$ tal que $\int_B f = f(x_0)\text{vol}B$.
\end{teo}

\begin{dem}
\begin{sloppypar}
$f$ \'e cont\'inua e $B$ \'e compacto, ent\~ao existem $x_1, x_2 \in B$ tal que ${f(x_1) \mei f(x) \mei f(x_2), \forall x \in B}$.
\end{sloppypar}

\[
\begin{gathered}
  \imp \int_B {f(x_1)} \mei \int_B {f(x)} \mei \int_B {f(x_2)} \hfill \\
  \imp f(x_1)\text{vol}B \mei \int_B {f(x)} \mei f(x_2)\text{vol}B \hfill \\ 
\end{gathered} 
\]

$\text{vol}B = 0 \imp \int_B {f(x)} = 0$ e qualquer $x_0 \in B$ satisfaz o enunciado.

\[
\text{vol}B \ne 0 \imp f(x_1) \mei \frac{\int_B {f(x)}}{\text{vol}B} \mei f(x_2)
\]

$B$ conexo por caminhos implica que, $\exists \gamma:[a,b] \to B$ cont\'inua com $\gamma(a) = x_1$ e $\gamma(b) = x_2$.

Considere $g(t) = (f \circ \gamma)(t)$. $g:[a,b] \to \R$ \'e cont\'inua.

\[
g(a) = f(x_1) \mei \frac{\int_B {f(x)}}{\text{vol}B} \mei f(x_2) = g(b)
\]

\begin{sloppypar}
O Teorema do valor intermedi\'ario para $g$ implica que $\exists t_0 \in [a,b]$ tal que ${g(t_0) = \frac{\int_B {f(x)}}{\text{vol}B}}$.
\end{sloppypar}

\[
f\underbrace {\left( {\gamma (t_0 )} \right)}_{x_0  \in B} = \frac{{\int_B f }}{\text{volB}}
\]

$\int_B f = f(x_0)\text{vol}B, x_0 = \gamma (t_0)$ \'e o ponto procurado.
\end{dem}

\begin{ex}
Como calcular $\int_B {\frac{x^2}{x^2 + y^2}}$, onde $B = \left\{ (x,y) \in \R^2: 0 < x^2 + y^2 \mei 1 \right\}$?
\end{ex}

\begin{dem}
Note que a integral n\~ao \'e limitada em $B$.

\begin{equation*}
  g(x,y) =
  \begin{cases}
    f(x,y) 	& \mbox{, se } (x,y) \in B \\
    0 	& \mbox{, se } (x,y) = (0,0)
  \end{cases}
\end{equation*}

$g:B \cup \{ 0,0 \} \to \R$ \'e limitada. $D = \{ (x,y) \in \R^2: f(x,y) \ne g(x,y) \}$.

$D = \{(0,0)\}$ tem conte\'udo nulo.

Portanto, $\int_B f = \int_B g$.
\end{dem}


\section{Teorema de Fubini}\label{sec28}

\begin{teo}[Fubini]\index{Teorema!de Fubini}
\begin{sloppypar}
Sejam $A_1 \subset \R^m$ e $A_2 \subset \R^n$ ret\^angulos e ${f:A_1 \times A_2 \to \R}$ integr\'avel. Sejam ainda, $f_x: A_2 \to \R$ dada por $f_x(y) = f(x,y)$ e $\varphi(x) = \int_{-A_2} f_x$ e $\Psi(x) = \int_{A_2}^{-} f_x$. (Neste caso, $f_x$ n\~ao \'e derivada parcial \'e fibra de $x$, ou seja, mant\'em $x$ fixo e integra $y$.) Ent\~ao, $\varphi(x)$ e $\Psi(x)$ s\~ao integr\'aveis em $A_1$ e
\end{sloppypar}

\[
\int_{A_1 } {\phi (x)}  = \int {\psi (x)}  = \int\limits_{A_1  \times A_2 } {f(x,y)} 
\]

ou seja,

\[
\int\limits_{A_1  \times A_2 } f  = \int_{A_1 } {\int_{ - A_2 } {f(x,y)} }  = \int_{A_1 } {\int_{A_2 }^ -  {f(x,y)} } 
\]
\end{teo}

\begin{cor}
Se $f:A_1 \times A_2 \to \R$ \'e integr\'avel, ent\~ao

\[
\int_{A_1 } {\int_{A_2 }^ -  f }  = \int_{A_2 } {\int_{A_1 }^ -  f }  = \int\limits_{A_1  \times A_2 } f
\]
\end{cor}

\begin{cor}
  \begin{sloppypar}
   O teorema de Fubini vale para qualquer $\xi:A_1 \to \R$ tal que ${\varphi(x) \mei \xi(x) \mei \Psi(x)}$.
  \end{sloppypar}
\end{cor}

\begin{ex}
Seja $f:[0,1]^2 \to \R$ tal que

\begin{equation*}
  f(x,y) =
  \begin{cases}
    \hfill 0 	& \mbox{, se } x \ne \mfrac{1}{2} \\
    \hfill 1 	& \mbox{, se } x = \mfrac{1}{2} \mbox{ e } y \in \Q \cap [0,1] \\
    \hfill 0 	& \mbox{, se } x = \mfrac{1}{2} \mbox{ e } y \in (\R \setminus \Q) \cap [0,1]
  \end{cases}
\end{equation*}

Calcule $\int_{[0,1]^2} f$.
\end{ex}

\begin{sol}

\tkzfigonly{figexFubini}{}

O conjunto de descontinuidade $D = \{ (x,y) \in R: x = \mfrac{1}{2} \}$ tem conte\'udo nulo, ent\~ao $f$ \'e integr\'avel, logo, vale o teorema de Fubini.

\[
\int\limits_{[0,1]^2 } f  = \int\limits_{[0,1]} {\left( {\int\limits_{ - [0,1]} {f_x dy} } \right)} dx = 0
\]

\end{sol}


\section{Integrais de Linha} \label{sec29}
\index{Integral!de linha}
\begin{defn}
Sejam $f:\R^n \to \R^n$ um campo vetorial cont\'inuo e $\gamma :\left[ {a,b} \right] \to \R^n$ uma curva de classe $C^1$ por partes. Definimos a \textit{integral de linha} de $f$ sobre $\gamma$ por

\[
\int_\gamma  f d\gamma  = \int\limits_a^b {\left\langle {f\left( {\gamma \left( t \right)} \right),\gamma '\left( t \right)} \right\rangle dt}
\]
\end{defn}

\begin{defn}
Seja $\gamma_1$ e $\gamma_2$ duas curvas tais que existe $g:[a,b] \to [c,d]$ de classe $C^1$ com $\gamma_2 (t) = \gamma_1 (g(t))$ e $g'(t) \ne 0$. Dizemos que $\gamma_2$ \'e \textit{reparametriza\c c\~ao} de $\gamma_1$ se $u = g(t)$, ent\~ao, $\frac{du}{dt} = g'(t)$

\[
\imp \gamma_2 ' (t) = \gamma_1 ' (g(t)).g'(t)
\]
\end{defn}

Se $g'(t) > 0$ dizemos que $g$ preserva orienta\c c\~ao e $g'(t) < 0$ troca a orienta\c c\~ao.

\begin{ex}
Seja $\gamma_1 (t) = (\cos t, \sin t), t \in [0,2\pi]$.

\[
\begin{gathered}
  g(s) =  - 2s,s \in [0,2\pi ] \hfill \\
  \gamma _2 (s) = \gamma _2 (g(s)) = \left( {\cos ( - 2s),\sin ( - 2s)} \right) \hfill \\ 
\end{gathered} 
\]
\end{ex}

\begin{teo}
Seja $F:\varOmega \subset \R^n \to \R^n$ campo cont\'inuo e $\varOmega$ aberto, $\gamma_1: [a,b] \to \varOmega$ e $\gamma_2: [c,d] \to \varOmega$, tal que $\gamma_2$ \'e reparametriza\c c\~ao de $\gamma_1$.

\begin{enumerate}[(i)]
 \item Se $\gamma_1$ e $\gamma_2$ tem mesma orienta\c c\~ao, ent\~ao $\int_{\gamma _1 } {Fd\gamma _1 }  = \int_{\gamma _2 } {Fd\gamma _2 }$.
 \item Se $\gamma_1$ e $\gamma_2$ tem orienta\c c\~oes opostas, ent\~ao $\int_{\gamma _1 } {Fd\gamma _1 }  = -\int_{\gamma _2 } {Fd\gamma _2 }$.
\end{enumerate}

\end{teo}

\begin{dem}
Se $\gamma_2$ \'e reparametriza\c c\~ao de $\gamma_1$, ent\~ao $\exists g: [c,d] \to [a,b]$ de classe $C^1 (g'(t) > 0)$ tal que $\gamma_2 (t) = \gamma_1 (g(t))$. Fazendo $u = g(t)$, temos

\begin{eqnarray*}
\int\limits_{\gamma _1 } {Fd\gamma _1 } &=& \int_a^b {\left\langle {F \circ \gamma _1 ,\gamma _1 '} \right\rangle dt}  \hfill \\
&=& \int_c^d {\left\langle {F \circ \gamma _1 (g(t)),\gamma _1 '(g(t)) \cdot g'(t)} \right\rangle dt}  \hfill \\
&=& \int_c^d {\left\langle {F\left( {\gamma _2 (t)} \right),\gamma _2 '(t)} \right\rangle dt}  \hfill \\
\int\limits_{\gamma _1 } {Fd\gamma _1 } &=& \int\limits_{\gamma _2 } {Fd\gamma _2 }  \hfill \\ 
\end{eqnarray*}

\end{dem}

\begin{defn}
\begin{sloppypar}
Uma curva $\gamma: [a,b] \to \R^n$ \'e $C^1$ por partes se existem ${t_0 = a < t_1 < \ldots < t_n = b}$ tal que $\mathop {\left. \gamma  \right|}\nolimits_{]t_{i - 1} ,t_i [}$ \'e de classe $C^1$.
\end{sloppypar}

\tkzfigonly{fig25}

Nesse caso, definimos $\int_\gamma F d_\gamma = \Soma{\int_I {\mathop {\left. {Fd\gamma } \right|}\nolimits_{[t_{i - 1} ,t_i ]} }}$, onde $I = \gamma_{[t_{i - 1} ,t_i ]}$.
\end{defn}

\begin{ex}
Sejam $F(x,y) = (-y,x)$ e $\gamma$ um tri\^angulo de v\'ertices $(0,0),(1,0),(1,1)$ percorrido no sentido anti-hor\'ario. Calcule $\gamma(t)$.

\tkzfigonly{fig26}
\end{ex}

\begin{sol}
\[
\begin{gathered}
  \gamma _1 (t) = (t,0),t \in [0,1] \hfill \\
  \gamma _2 (t) = (1,t),t \in [0,1] \hfill \\
  \gamma _3 (t) = (1,1) + t( - 1, - 1) = (1 - t,1 - t) \hfill \\ 
\end{gathered} 
\]

\begin{eqnarray*}
  \int_\gamma  {Fd\gamma }  &=& \int_{\gamma _1 } {Fd\gamma _1 }  + \int_{\gamma _2 } {Fd\gamma _2 }  + \int_{\gamma _3 } {Fd\gamma _3 }  \hfill \\
   &=& \int_0^1 {\left\langle {(0,t),(1,0)} \right\rangle dt}  + \int_0^1 {\left\langle {( - t,1),(0,1)} \right\rangle dt}  + \int_0^1 {\left\langle {(t - 1,1 - t),( - 1, - 1)} \right\rangle dt}  \hfill \\
   &=& 0 + 1 + 0 = \boxed1 \hfill \\ 
\end{eqnarray*}

\end{sol}

\newpage 

\begin{prop}
Seja $F = (P,Q)$ campo de classe $C^1$ e $\gamma$ um ret\^angulo de lados $(0,0),(0,a),(0,b),(a,b)$. Seja $B$ o ret\^angulo fechado de lados dados por $\gamma$. Se $\gamma$ \'e percorrido no sentido anti-hor\'ario, ent\~ao

\[
\int_\gamma  {Pdx + Qdy}  = \int_B {\left( {\frac{{\partial Q}}
{{\partial x}} - \frac{{\partial P}}
{{\partial y}}} \right)dxdy} 
\]

\tkzfigonly{fig27}
\end{prop}

\begin{dem}
Para $y$ fixado, temos

\[
\int_B {\frac{{\partial Q}}
{{\partial x}}dxdy}  = \int_0^b {\left( {\int_0^a {\frac{{\partial Q}}
{{\partial x}}dx} } \right)dy}  = \int_0^b {Q(a,y) - Q(0,y)dy} 
\]

Para $y$ fixado, temos

\[
 - \int_B {\frac{{\partial P}}
{{\partial y}}dxdy}  -  = \int_0^a {\left( {\int_0^b {\frac{{\partial P}}
{{\partial y}}dy} } \right)dx}  =  - \int_0^a {P(x,b) - P(x,0)dx} 
\]

Por outro lado,

\begin{eqnarray*}
  \int_\gamma  {Qdy} &=& \cancel{\mathop {\int_{\gamma _1 } {Qdy} }\nolimits^0}  + \int_{\gamma _2 } {Qdy}  + \cancel{\mathop {\int_{\gamma _3 } {Qdy} }\nolimits^0}  + \int_{\gamma _4 } {Qdy}  \hfill \\
   &=& \int_0^b {Q(a,t)dt}  - \int_0^b {Q(0,t)dt}  \hfill \\
   &=& \int_0^b {\left[ {Q(a,t) - Q(0,t)} \right]dt}  \hfill \\ 
\end{eqnarray*}

E

\begin{eqnarray*}
  \int_\gamma  {Pdx} &=& \int_{\gamma _1 } {Pdx} + \cancel{\mathop {\int_{\gamma _2 } {Pdx} }\nolimits^0}  + \int_{\gamma _3 } {Pdx} + \cancel{\mathop {\int_{\gamma _4 } {Pdx} }\nolimits^0}  \hfill \\
   &=& \int_0^a {P(t,0)dt}  - \int_0^a {P(t,b)dt}  \hfill \\
   &=& \int_0^a {\left[ {P(t,0) - P(t,b)} \right]dt}  \hfill \\ 
\end{eqnarray*}

\[
\int_B {\frac{{\partial Q}}
{{\partial x}} - \frac{{\partial P}}
{{\partial y}}dxdy}  = \int_\gamma  {Qdy}  + \int_\gamma  {Pdx}  = \int_B {Pdx + Qdy} 
\]

\end{dem}

\section{Campos Conservativos} \label{sec30}

\begin{defn}
Um campo $F: \R^n \to \R^n$ \'e \emph{conservativo}\index{Campo!conservativo} se existe $\varphi: \R^n \to \R$ tal que $\bigtriangledown \varphi = F$.
\end{defn}

\begin{teo}
Sejam $n=2,3$ e $F:\varOmega \subset \R^n \to \R^n$ um campo $C^1$. Se $F$ \'e conservativo, ent\~ao $\rot F \equiv 0$ em $\varOmega$.
\end{teo}

\subsection*{1-Formas diferenciais}

\begin{defn}
  \begin{sloppypar}
    Uma 1-forma diferencial em $\R^n$ \'e uma express\~ao do tipo ${\omega  = F'dx^1  + F^2 dx^2  +  \ldots  + F^n dx^n}$ \'e associada a um campo $F = \left( F^1, \ldots, F^n \right)$ de $\R^n$ em $\R^n$.
  \end{sloppypar}
\end{defn}

\begin{defn}
Uma 1-forma diferencial \'e \textit{exata} se existe $\varphi: \varOmega \subset \R^n \to \R$ tal que $\nabla \varphi = F = \left( F^1, \ldots, F^n \right)$.
\end{defn}

Em $\R^2$ ou $\R^3$ \'e necess\'ario que $\rot F \equiv 0$ para $\omega$ ser exata.

\begin{ex}
$\omega = 2xdx + 2ydy$ \'e exata. De fato, $\varphi(x,y) = x^2 + y^2$.
\end{ex}

\begin{ex}
$\omega  = \underbrace y_Pdx + \underbrace {2x}_Qdy$ n\~ao \'e exata. De fato, $2 = \frac{{\partial Q}}{{\partial x}} \ne \frac{{\partial P}}{{\partial y}} = 1$.
\end{ex}

Sejam $F:\varOmega \subset \R^n \to \R^n$ campo conservativo (ou $\omega$ exata) e $\gamma:[a,b] \to \varOmega$ curva de classe $C^1$.

\begin{eqnarray*}
  \int_\gamma {Fd\gamma }  &=& \int_a^b {\left\langle {F \circ \gamma ,\gamma '} \right\rangle dt}  \hfill \\
   &=& \int_a^b {\left\langle {\nabla \varphi  \circ \gamma ,\gamma '} \right\rangle dt}  \hfill \\
   &=& \int_a^b {\left( {\varphi  \circ \gamma } \right)'(t)dt}  \hfill \\
   &=& \varphi  \circ \gamma (b) - \varphi  \circ \gamma (a) \hfill \\ 
\end{eqnarray*}

\begin{ex}
Seja $F(x,y) = \left( {\frac{x}{{x^2  + y^2 }},\frac{y}{{x^2  + y^2 }}} \right)$. $\gamma$ fechada de classe $C^1$ em $\R^2 \setminus \{ (0,0) \}$.

\[
\begin{gathered}
  \int_\gamma Fd\gamma = 0 \hfill \\
  \varphi(x,y) = \frac{1}{2} \ln(x^2 + y^2) \hfill \\ 
\end{gathered} 
\]
\end{ex}

\begin{ex}
Seja $F(x,y) = \left( \frac{-y}{x^2 + y^2},\frac{x}{x^2 + y^2} \right)$.

\[
\begin{gathered}
  \gamma(t) = (\cos t, \sin t), t \in [0,2\pi] \hfill \\
  \int_\gamma Fd\gamma = \int_0^{2\pi} {\underbrace {\left\langle {F \circ \gamma ,\gamma '} \right\rangle }_1dt} = 2\pi \hfill \\ 
\end{gathered} 
\]

Portanto, $F$ n\~ao \'e conservativo.
\end{ex}

\begin{teo}
Seja $\varOmega \subset \R^n$ aberto conexo por caminhos e $F:\varOmega \to \R^n$ campo cont\'inuo tal que $\int_\gamma Fd\gamma$ n\~ao depende de $\gamma$ entre dois pontos dados. Para $a,x \in \varOmega$, $\varphi (x) = \int_{\br{ax}  = \gamma } {Fd\gamma }$ \'e um potencial de $F$.
\end{teo}

\begin{teo}
Seja $F:\varOmega \subset \R^n \to \R^n$ um campo cont\'inuo em $\varOmega$ aberto e conexo por caminhos. S\~ao equivalentes:

\begin{enumerate}[(i)]
 \item $F$ \'e conservativo.
 \item $\int_\gamma Fd\gamma = 0, \forall \gamma$ de classe $C^1$ fechado.
 \item $\int_\gamma Fd\gamma$ s\'o depende dos extremos de $\gamma$.
 \item $\omega$ definida por $F$ \'e exata.
\end{enumerate}

\end{teo}

\begin{defn}
Seja $\varOmega$ conexo por caminhos. $\varOmega$ \'e simplesmente conexo se para toda curva $\gamma:[a,b] \to \varOmega$ cont\'inua e fechada existe $H:[a,b] \times [0,1] \to \varOmega$ cont\'inua tal que

\[
\left\{ \begin{gathered}
  H(t,0) = \gamma (t) \hfill \\
  H(t,1) = \gamma (a) \hfill \\ 
\end{gathered}  \right.
\]

\end{defn}

\begin{teo}
Se $F:\varOmega \to \R^n$ tem $\rot F \equiv 0$ e $\varOmega$ \'e simplesmente conexo, ent\~ao $F$ \'e conservativo.
\end{teo}

%*******************************************************
\appendix
\chapter{Avalia\c{c}\~oes} \label{chapAv}

\section{Avalia\c{c}\~ao 01} \label{secP1}

\textbf{Grupo 1: C\'alculo em uma vari\'avel real.}

\begin{enumerate}
  \item Sejam $f$ uma fun\c{c}\~ao real deriv\'avel, com derivada cont\'inua e $a<b$ n\'umeros reais tais que $f(a) = f(b) = 0$ e $f'(a) f'(b) > 0$. Prove que existe $c \in \left] {a,b} \right[$ tal que $f(c)=0$.

\begin{sol}
$f'(a)>0$ e $f'(b)>0$ existe $B_{\delta _1 }(a)$ tal que $f'\left( {B_{\delta _1 } \left( a \right)} \right) > 0$, implica que, $f\left( {B_{\delta _1 } \left( a \right) \cap \left] {a,b} \right[} \right) > 0$.

$f'(b)>0$ existe $B_{\delta _2 }(b)$ tal que $f'\left( {B_{\delta _2 } \left( b \right)} \right) > 0$, implica que, $f\left( {B_{\delta _2 } \left( b \right) \cap \left] {a,b} \right[} \right) < 0$.

$f$ deriv\'avel, implica que, $f$ \'e cont\'inua, implica, pelo TVI, que $c \in \left] {a,b} \right[$ tal que $f(c) = 0$.
\end{sol}

  \item Sejam $f$ uma fun\c{c}\~ao real deriv\'avel em um intervalo aberto $I$ e $a<b$ n\'umeros reais tais que $f'(a) < 0$ e $f'(b) > 0$. Prove que existe $c \in \left] {a,b} \right[$ tal que $f'(c)=0$. (Note que nada se sabe a respeito da continuidade de $f'$.)

      \textit{Dica.} No caso de $f(b) = f(a)$ o resultado segue do Teorema de Rolle. Se $f(b) \ne f(a)$ considere todas as poss\'iveis ordens entre $0,f'(a),f'(b)$ e $\displaystyle \frac{{f\left( b \right) - f\left( a \right)}}{{b - a}}$. Em cada caso construa uma fun\c{c}\~ao cont\'inua $g(x)$ relacionada com a defini\c{c}\~ao usual de derivada de $f$ tal que $g(a)g(b)<0$. Use o Teorema do Valor Intermedi\'ario para $g$. Agora, use o Teorema do Valor M\'edio para $f$ num intervalo conveniente e conclua o resultado.

      \textbf{Obs:} O resultado acima pode ser generalizado: nas mesmas condi\c{c}\~oes, mas com $f'(a) \ne f'(b)$, para cada $d \in \left[ {f'\left( a \right),f'\left( b \right)} \right]$ existe $c \in \left[ {a,b} \right]$ tal que $f'(c)=d$. Admitindo essa generaliza\c{c}\~ao conclua que se $f'(x)$ \'e crescente (ou decrescente) num intervalo, ent\~ao, $f'$ \'e cont\'inua nesse intervalo.

\begin{sol}
$f$ deriv\'avel em $a$, implica que

\[
\begin{gathered}
f\left( x \right) = f\left( a \right) + \left( {x - a} \right)f_1 \left( x \right) \hfill \\
f'\left( a \right) = f_1 \left( a \right) < 0 \hfill \\
\end{gathered}
\]

onde $f_1 (x)$ \'e cont\'inua em $a$.

$\Rightarrow \exists B_\delta  \left( a \right)$ tal que $f_1 \left( {B_\delta  \left( a \right)} \right) < 0$

\[
\begin{gathered}
x \in B_\delta  \left( a \right) \Rightarrow f\left( x \right) < f\left( a \right) \hfill \\
\left( {x > a} \right) \hfill \\
\end{gathered}
\]

$f$ deriv\'avel em $b$

\[
\begin{gathered}
f\left( x \right) = f\left( b \right) + \left( {x - b} \right)f_1 \left( x \right) \hfill \\
f'\left( b \right) = f_1 \left( b \right) > 0 \hfill \\
\end{gathered}
\]

onde $f_1 (x)$ \'e cont\'inua em $b$.

$\Rightarrow \exists B_{\delta _2 }  \left( b \right)$ tal que $f_1 \left( {B_{\delta _2 } \left( b \right)} \right) > 0$

\[
\begin{gathered}
x \in B_{\delta _2 } \left( b \right) \Rightarrow f\left( x \right) < f\left( b \right) \hfill \\
\left( {x < b} \right) \hfill \\
\end{gathered}
\]

$f(a)$ e $f(b)$ n\~ao s\~ao m\'inimos.

$f$ cont\'inua em $\left[ {a,b} \right]$, implica, pelo Teorema de Fermat, que $f$ tem m\'inimo no interior, $c \in \left] {a,b} \right[ \Rightarrow f'\left( c \right) = 0$.

\uline{2$^\underline{\texrm{a}}$ vers\~ao} \textit{n\~ao tenho o enunciado da quest\~ao que foi adaptada em sala de aula.}

\begin{enumerate}[(a)]

\item Se $f(b) \ne f(a)$, ent\~ao

\[
\begin{gathered}
  0,f'\left( a \right),f'\left( b \right),\frac{{f\left( b \right) - f\left( a \right)}}
{{b - a}} \hfill \\
  f'\left( a \right) < 0 < f'\left( b \right) < \frac{{f\left( b \right) - f\left( a \right)}}{{b - a}} \hfill \\
\end{gathered}
\]

(\'e uma das possibilidades), analisemos dois casos:

\begin{enumerate}[(i)]
    \item $f'\left( a \right) < 0 < \displaystyle \frac{{f\left( b \right) - f\left( a \right)}}{{b - a}}$

    \item $\displaystyle \frac{{f\left( b \right) - f\left( a \right)}}{{b - a}} < 0 < f'\left( b \right)$
\end{enumerate}

Em:

\begin{enumerate}[(i)]
\item Temos

\begin{equation*}
g(x) = \left\{ \begin{array}{cl}\displaystyle
                        \frac{{f(x)  - f(a) }}{{x-a}} & \textrm{se } x \ne a\\
f'(a) & \textrm{se } x=a \end{array}\right.
\end{equation*}

$g$ \'e cont\'inua, ent\~ao,

\[
\begin{gathered}
g\left( a \right) = f'\left( a \right) < 0 \hfill \\
g\left( b \right) = \frac{{f\left( b \right) - f\left( a \right)}}{{b - a}} > 0 \hfill \\
\end{gathered}
\]

Ent\~ao, $\exists x_1  \in \left] {a,b} \right[$ tal que $g\left( {x_1 } \right) = \displaystyle \frac{{f\left( {x_1 } \right) - f\left( a \right)}}{{x - a}} = 0$

Pelo TVM, $c \in \left] {a,b} \right[$ tal que $\displaystyle \frac{{f\left( {x_1 } \right) - f\left( a \right)}}{{x - a}} = f'\left( c \right)$.

\item An\'alogo.
\end{enumerate}

\item Dado $d \in \left] {f'\left( a \right),f'\left( b \right)} \right[$ existe $c \in \left] {a,b} \right[$ tal que $f'\left( c \right) = d$.

Defina

\[
\begin{gathered}
g\left( x \right) = f\left( x \right) - dx \hfill \\
g'\left( x \right) = f'\left( x \right) - d \hfill \\
                g'\left( a \right) = f'\left( a \right) - d < 0\left( {d > f'\left( a \right)} \right) \hfill \\
g'\left( b \right) = f'\left( b \right) - d > 0 \hfill \\
\end{gathered}
\]

Implica que, $\exists c \in \left] {a,b} \right[$ tal que $g'\left( c \right) = 0$. Ent\~ao,

\[
\begin{gathered}
0 = g'\left( c \right) = f'\left( c \right) - d \hfill \\
\Rightarrow f'\left( c \right) = d \hfill \\
\end{gathered}
\]

Ainda, $f'$ crescente, implica que, $f'$ \'e cont\'inua.

\end{enumerate}

\end{sol}

\newpage 

\textbf{Grupo 2: C\'alculo em v\'arias vari\'aveis reais.}

  \item Justifique a exist\^encia ou n\~ao dos limites abaixo. Determine-os, se poss\'ivel.

\begin{enumerate}[(a)]
\item $\mathop {\lim }\limits_{\left( {x,y} \right) \to \left( {0,0} \right)} \displaystyle \frac{{x^2 y}}{{x^4  + y^2 }}$

\begin{sol}
Seja

\[
\begin{gathered}
  \gamma _1 \left( t \right) = \left( {t,0} \right) \Rightarrow \left( {f \circ \gamma _1 } \right)\left( t \right) = \frac{{t^2 0}}
{{t^4  + 0}} = 0 \hfill \\
  \mathop {\lim }\limits_{t \to 0} \left( {f \circ \gamma _1 } \right)\left( t \right) = 0 \hfill \\
  \gamma _2 \left( t \right) = \left( {t,t^2 } \right) \Rightarrow \left( {f \circ \gamma _2 } \right)\left( t \right) = \frac{{t^2 t^2 }}
{{t^4  + t^4 }} = \frac{1}
{2} \hfill \\
  \mathop {\lim }\limits_{t \to 0} \left( {f \circ \gamma _2 } \right)\left( t \right) = \frac{1}
{2} \hfill \\
\end{gathered}
\]

Portanto, o limite n\~ao existe.
\end{sol}

\item $\mathop {\lim }\limits_{\left( {x,y} \right) \to \left( {0,0} \right)} \displaystyle \frac{{x^3 + y^3}}{{x^2  + y^2 }}$

\begin{sol}
\[
\begin{gathered}
  \mathop {\lim }\limits_{\left( {x,y} \right) \to \left( {0,0} \right)} \frac{{x^3  + y^3 }}
{{x^2  + y^2 }} \hfill \\
  \left( {\frac{{x^3  + y^3 }}
{{x^2  + y^2 }} = \frac{{x^3 }}
{{x^2  + y^2 }} + \frac{{y^3 }}
{{x^2  + y^2 }} = x\frac{{x^2 }}
{{x^2  + y^2 }} + y\frac{{y^2 }}
{{x^2  + y^2 }}} \right) \hfill \\
  \mathop {\lim }\limits_{\left( {x,y} \right) \to \left( {0,0} \right)} \left( {x\underbrace {\frac{{x^2 }}
{{x^2  + y^2 }}}_{{\text{limitada}}} + y\underbrace {\frac{{y^2 }}
{{x^2  + y^2 }}}_{{\text{limitada}}}} \right) = 0 \hfill \\
\end{gathered}
\]

\end{sol}

\end{enumerate}

\item Sejam $f:\R^3 \to \R$ dada por $f(x,y,z)=x^2+y^2-z$ e $\gamma$ a curva em $\R^3$ dada pela interse\c{c}\~ao das superf\'icies $x^2+y^2=4$ e $z=y^2$.

\begin{enumerate}[(a)]
	\item Determine uma parametriza\c{c}\~ao para $\gamma$.

	\item Determine a derivada de $f$ ao longo do vetor tangente de $\gamma$ para um certo $t_0$, ou seja, calcule $\displaystyle \frac{{\partial f}}{{\partial y'}}\left( {\gamma \left( {t_0 } \right)} \right)$.

	\item Determine os pontos de m\'aximo e m\'inimo de $\left( {f \circ \gamma } \right)\left( t \right)$.

\end{enumerate}

\begin{sol}
\begin{enumerate}[(a)]
  \item Note que as superf\'icies dadas representam um cilindro e uma par\'abola, respectivamente, ent\~ao a parametriza\c{c}\~ao de $\gamma$ \'e dada por

\[
\begin{gathered}
  \gamma \left( t \right) = \left( {x\left( t \right),y\left( t \right),z\left( t \right)} \right) \hfill \\
  x^2 \left( t \right) + y^2 \left( t \right) = 4 \hfill \\
  z\left( t \right) = y^2 \left( t \right) \hfill \\
  \left\{ \begin{gathered}
  x\left( t \right) = 2\cos t \hfill \\
  y\left( t \right) = 2\sin t \hfill \\
  z\left( t \right) = 4\sin ^2 t \hfill \\
\end{gathered}  \right.t \in \left[ {0,2\pi } \right] \hfill \\
\end{gathered}
\]

  \item Por defini\c{c}\~ao, poderiamos usar

\[
\frac{{\partial f}}
{{\partial \gamma '}}\left( {\gamma \left( {t_0 } \right)} \right) = \mathop {\lim }\limits_{h \to 0} \frac{{f\left( {\gamma \left( {t_0 } \right) + h\gamma '\left( {t_0 } \right)} \right) - f\left( {\gamma \left( {t_0 } \right)} \right)}}
{h}
\]

Mas tamb\'em podemos usar

\[
\begin{gathered}
  \frac{{\partial f}}
{{\partial u}}\left( {x_0 } \right) = \left\langle {\nabla f\left( {x_0 } \right),u} \right\rangle  \hfill \\
  \frac{{\partial f}}
{{\partial \gamma '}}\left( {\gamma \left( {t_0 } \right)} \right) = \left\langle {\nabla f\left( {\gamma \left( {t_0 } \right)} \right),\gamma '\left( {t_0 } \right)} \right\rangle  \hfill \\
  \nabla f\left( {x,y,z} \right) = \left( {2x,2y, - 1} \right) \Rightarrow \nabla f\left( {\gamma \left( {t_0 } \right)} \right) = \left( {4\cos t,4\sin t, - 1} \right) \hfill \\
  \gamma '\left( {t_0 } \right) = \left( { - 2\sin t_0 ,2\cos t_0 ,8\sin t_0 \cos t_0 } \right) \hfill \\
  \therefore \frac{{\partial f}}
{{\partial \gamma '}}\left( {\gamma \left( {t_0 } \right)} \right) =  - 8\sin t_0 \cos t_0  \hfill \\
\end{gathered}
\]


  \item Temos

\[
\begin{gathered}
  \gamma :\left[ {0,2\pi } \right] \to \R^3  \hfill \\
  f:\R^3  \to \R \hfill \\
  f \circ \gamma :\left[ {0,2\pi } \right] \to \R \hfill \\
  \frac{d}
{{dt}}\left( {f \circ \gamma } \right)\left( t \right) = 0 \hfill \\
\end{gathered}
\]

\uline{$1^\circ$ modo}

\[
\begin{gathered}
  \left( {f \circ \gamma } \right)\left( t \right) = 4\cos ^2 t + 4\sin ^2 t - 4\sin ^2 t = 4\cos ^2 t \hfill \\
  \frac{d}
{{dt}}\left( {f \circ \gamma } \right)\left( t \right) =  - 8\cos t\sin t \hfill \\
   \Rightarrow  - 8\cos t\sin t = 0 \hfill \\
   \Rightarrow \cos t = 0{\text{ e }}\sin t = 0 \hfill \\
  \therefore t = {\raise0.5ex\hbox{$\scriptstyle \pi $}
\kern-0.1em/\kern-0.15em
\lower0.25ex\hbox{$\scriptstyle 2$}},{\raise0.5ex\hbox{$\scriptstyle {3\pi }$}
\kern-0.1em/\kern-0.15em
\lower0.25ex\hbox{$\scriptstyle 2$}},0,\pi  \hfill \\
\end{gathered}
\]

Por

\[
\frac{{d^2 }}
{{dt^2 }}\left( {f \circ \gamma } \right)\left( t \right) = 8\sin ^2 t - 8\cos ^2 t = 8\cos 2t
\]

Ent\~ao

\[
\begin{gathered}
  t = {\raise0.5ex\hbox{$\scriptstyle \pi $}
\kern-0.1em/\kern-0.15em
\lower0.25ex\hbox{$\scriptstyle 2$}} \Rightarrow \left( {f \circ \gamma } \right)''\left( {{\raise0.5ex\hbox{$\scriptstyle \pi $}
\kern-0.1em/\kern-0.15em
\lower0.25ex\hbox{$\scriptstyle 2$}}} \right) =  - 8 \Rightarrow {\raise0.5ex\hbox{$\scriptstyle \pi $}
\kern-0.1em/\kern-0.15em
\lower0.25ex\hbox{$\scriptstyle 2$}}{\text{ max local}} \hfill \\
  t = {\raise0.5ex\hbox{$\scriptstyle {3\pi }$}
\kern-0.1em/\kern-0.15em
\lower0.25ex\hbox{$\scriptstyle 2$}} \Rightarrow \left( {f \circ \gamma } \right)''\left( {{\raise0.5ex\hbox{$\scriptstyle {3\pi }$}
\kern-0.1em/\kern-0.15em
\lower0.25ex\hbox{$\scriptstyle 2$}}} \right) =  - 8 \Rightarrow {\raise0.5ex\hbox{$\scriptstyle {3\pi }$}
\kern-0.1em/\kern-0.15em
\lower0.25ex\hbox{$\scriptstyle 2$}}{\text{ max local}} \hfill \\
  t = 0 \Rightarrow \left( {f \circ \gamma } \right)''\left( 0 \right) = 8 \Rightarrow 0{\text{ min local}} \hfill \\
  t = \pi  \Rightarrow \left( {f \circ \gamma } \right)''\left( \pi  \right) = 8 \Rightarrow \pi {\text{ min local}} \hfill \\
\end{gathered}
\]

\uline{$2^\circ$ modo}

\[
\begin{gathered}
  \frac{d}
{{dt}}\left( {f \circ \gamma } \right)\left( t \right) = \left\langle {\nabla f\left( {\gamma \left( t \right)} \right),\gamma '\left( t \right)} \right\rangle  =  - 8\sin t\cos t \hfill \\
  \frac{d}
{{dt}} = 0 \Rightarrow t = {\raise0.5ex\hbox{$\scriptstyle \pi $}
\kern-0.1em/\kern-0.15em
\lower0.25ex\hbox{$\scriptstyle 2$}},{\raise0.5ex\hbox{$\scriptstyle {3\pi }$}
\kern-0.1em/\kern-0.15em
\lower0.25ex\hbox{$\scriptstyle 2$}},0,\pi  \hfill \\
\end{gathered}
\]


\end{enumerate}
\end{sol}

\item Mostre que

\begin{equation*}
f(x,y) = \left\{ \begin{array}{cl}\displaystyle
        \frac{{x^2(y-1)}}{{x^2 + (y-1)^2}} & \textrm{se }\left( {x,y} \right) \ne \left( {0,1} \right)\\
        0 & \textrm{se }\left( {x,y} \right) = \left( {0,1} \right)\end{array}\right.
\end{equation*}

\'e cont\'inua em $(0,1)$ e possui todas as derivadas direcionais nesse ponto. A fun\c{c}\~ao $f$ \'e diferenci\'avel em $(0,1)$? Justifique.

\begin{sol}
\[
\scriptstyle{
\mathop {\lim }\limits_{\left( {x,y} \right) \to \left( {0,1} \right)} f\left( {x,y} \right) = \mathop {\lim }\limits_{\left( {x,y} \right) \to \left( {0,1} \right)} \frac{{x^2 \left( {y - 1} \right)}}
{{x^2  + \left( {y - 1} \right)^2 }} = \mathop {\lim }\limits_{\left( {x,y} \right) \to \left( {0,1} \right)} \underbrace {\scriptstyle{\frac{x^2}{x^2  + \left( {y - 1} \right)^2}}}_{{\text{limitada}}}\left( {y - 1} \right) = 0 = f\left( {0,1} \right)
}
\]

Portanto, $f$ \'e cont\'inua.

Seja $u=(a,b)$ e $a^2+b^2=1$, ent\~ao, calculando a derivada direcional, temos


\begin{eqnarray*}
  \frac{{\partial f}}{{\partial u}}\left( {0,1} \right) &=& \mathop {\lim }\limits_{h \to 0} \frac{{f\left( {\left( {0,1} \right) + h\left( {a,b} \right)} \right) - f\left( {0,1} \right)}}{h} \hfill \\
   &=& \mathop {\lim }\limits_{h \to 0} \frac{{f\left( {\left( {ha,1 + hb} \right)} \right) - f\left( {0,1} \right)}}{h} \hfill \\
   &=& \mathop {\lim }\limits_{h \to 0} \frac{{\frac{{h^2 a^2 hb}}{{h^2 a^2  + h^2 b^2 }} - 0}}{h} = \mathop {\lim }\limits_{h \to 0} \frac{{\frac{{h^3 a^2 b}}{{h^2 }}}}{h} = a^2 b \hfill \\
\end{eqnarray*}

Para verificar se $f$ \'e diferenci\'avel em $(0,1)$, fa\c{c}amos

\[
\begin{gathered}
  \mathop {\lim }\limits_{\left( {h,k} \right) \to \left( {0,0} \right)} \frac{{f\left( {\left( {0,1} \right) + \left( {h,k} \right)} \right) - f\left( {0,1} \right) - f_x \left( {0,1} \right)h - f_y \left( {0,1} \right)k}}
{{\left\| {\left( {h,k} \right)} \right\|}} =  \hfill \\
   = \mathop {\lim }\limits_{\left( {h,k} \right) \to \left( {0,0} \right)} \frac{{\frac{{h^2 k}}
{{h^2  + k^2 }} - 0 - 0h - 0k}}
{{\sqrt {h^2  + k^2 } }} =  \hfill \\
   = \mathop {\lim }\limits_{\left( {h,k} \right) \to \left( {0,0} \right)} \frac{{h^2 k}}
{{\left( {h^2  + k^2 } \right)\sqrt {h^2  + k^2 } }} \hfill \\
\end{gathered}
\]

Seja

\[
g\left( {h,k} \right) = \frac{{h^2 k}}
{{\left( {h^2  + k^2 } \right)\sqrt {h^2  + k^2 } }}
\]

ent\~ao

\[
\begin{gathered}
  \gamma _1 \left( t \right) = \left( {t,t} \right) \hfill \\
  \mathop {\lim }\limits_{t \to 0} \left( {g \circ \gamma } \right)\left( t \right) = \mathop {\lim }\limits_{t \to 0} \frac{{t^3 }}
{{2t^2 \left( {2t^2 } \right)^{{\raise0.5ex\hbox{$\scriptstyle 1$}
\kern-0.1em/\kern-0.15em
\lower0.25ex\hbox{$\scriptstyle 2$}}} }} = \mathop {\lim }\limits_{t \to 0} \frac{t}
{{2\sqrt 2 \left| t \right|}} \hfill \\
  \therefore \lim \nexists  \hfill \\
\end{gathered}
\]

\end{sol}

  \item Admita que $T(x,y) = 16-2x^2-y^2$ representa uma distribui\c{c}\~ao de temperatura no plano $xy$. Determine uma parametriza\c{c}\~ao da trajet\'oria descrita por um ponto $P$ que se desloca a partir do ponto $(1,2)$ sempre na dire\c{c}\~ao e sentido de m\'aximo crescimento da temperatura.

\begin{sol}
Seja $\gamma \left( t \right) = \left( {x\left( t \right),y\left( t \right)} \right),t \geqslant 0$

$T$ cresce mais r\'apido na dire\c{c}\~ao do vetor gradiente $\nabla T$. Ent\~ao,

\[
\begin{gathered}
  \gamma '\left( t \right) = \nabla T\left( {\gamma \left( t \right)} \right) \hfill \\
  \left( {x'\left( t \right),y'\left( t \right)} \right) = \left( { - 4x\left( t \right), - 2y\left( t \right)} \right) \hfill \\
  \left\{ \begin{gathered}
  x'\left( t \right) =  - 4x\left( t \right) \hfill \\
  y'\left( t \right) =  - 2y\left( t \right) \hfill \\
\end{gathered}  \right. \hfill \\
\end{gathered}
\]

O ponto inicial \'e $x(0)=1$ e $y(0)=2$, ent\~ao

\[
\begin{gathered}
  x' =  - 4x \hfill \\
   \Rightarrow \frac{{x'}}
{x} =  - 4 \Rightarrow \int {\frac{{x'}}
{x}dt = \int { - 4dt} }  \hfill \\
   \Rightarrow \ln \left| {x\left( t \right)} \right| =  - 4t + c_1  \hfill \\
   \Rightarrow \left| {x\left( t \right)} \right| = e^{ - 4t} e^{c_1 } {\text{ e }}\left| {y\left( t \right)} \right| = e^{ - 2t} e^{c_2 }  \hfill \\
\end{gathered}
\]

Quando $t=0$, $x(0)=1$ e $y(0)=2$, ent\~ao,

\[
\begin{gathered}
  1 = x\left( 0 \right) = e^{ - 4.0} e^{c_1 }  \Rightarrow e^{c_1 }  = 1 \Rightarrow \left| {x\left( t \right)} \right| = e^{ - 4t}  \hfill \\
  2 = y\left( 0 \right) = e^{ - 2.0} e^{c_2 }  \Rightarrow e^{c_2 }  = 2 \Rightarrow \left| {y\left( t \right)} \right| = 2e^{ - 2t}  \hfill \\
\end{gathered}
\]

Ent\~ao, $\gamma \left( t \right) = \left( {e^{ - 4t} ,2e^{ - 2t} } \right),t \geqslant 0$

Note que $x\left( t \right) = \frac{{y^2 \left( t \right)}}{4}$, ent\~ao, a imagem \'e


\end{sol}

\end{enumerate}


\newpage

\section{Avalia\c{c}\~ao 02} \label{secP2}

\textbf{Grupo 1: Regra da cadeia.}

\begin{enumerate}
  \item Seja $f:\R^2 \to \R$ uma fun\c{c}\~ao harm\^onica, isto \'e

\[
\Delta f = \frac{{\partial ^2 f}}
{{\partial x^2 }} + \frac{{\partial ^2 f}}
{{\partial y^2 }} \equiv 0
\]

Mostre que $g\left( {u,v} \right) = f\left( {u^2  - v^2 ,2uv} \right)$ \'e harm\^onica nas vari\'aveis $u$ e $v$.

\begin{sol}
Exerc\'icio
\end{sol}

  \item Suponha que $f(x,t)$, definida em $\R^2$, \'e uma fun\c{c}\~ao real de classe $C^2$ que satisfaz \`a equa\c{c}\~ao $f_{xx} = f_{tt}$ para todo $(x,t) \in \R^2$.

  \begin{enumerate}[(a)]
    \item Mostre que $g\left( {u,v} \right) = f\left( {u + v,u - v} \right)$ satisfaz $g_{uv} = 0$;
    \item Usando o item anterior determine fun\c{c}\~oes $f(x,t)$ tais que $f_{xx} = f_{tt}$. (Dica: se $g_{uv} = 0$ o que podemos dizer sobre $g_v$? A partir disso o que podemos concluir sobre $g$?)
  \end{enumerate}

\begin{sol}
  \begin{enumerate}[(a)]
    \item $g\left( {u,v} \right) = f\left( {u + v,u - v} \right)$ satisfaz $g_{uv} = 0$, ent\~ao,

\[
\begin{gathered}
  g_u  = f_x .1 + f_t .1 \hfill \\
  g_{uv}  = f_{xx} .1 + \cancel{f_{xt} \left( { - 1} \right)} + \cancel{f_{tx} .1} + f_{tt} \left( { - 1} \right) \hfill \\
  g_{uv}  = f_{xx}  - f_{tt}  = 0 \hfill \\
\end{gathered}
\]


    \item $g_{uv} = 0$, ent\~ao

\[
\begin{gathered}
  g_v  = f_1 \left( v \right) \hfill \\
  g = \int {f_1 \left( v \right)dv}  + f_2 \left( u \right) \hfill \\
  g\left( {u,v} \right) = \int {f_1 \left( v \right)dv}  + f_2 \left( u \right) \hfill \\
  g\left( {u,v} \right) = f\left( {u + v,u - v} \right) \hfill \\
  \left\{ \begin{gathered}
  x = u + v \hfill \\
  t = u - v \hfill \\
\end{gathered}  \right. \Rightarrow \left\{ \begin{gathered}
  u = \frac{{x + t}}
{2} \hfill \\
  v = \frac{{x - t}}
{2} \hfill \\
\end{gathered}  \right. \hfill \\
   \Rightarrow f\left( {x,t} \right) = \int {f_1 \left( {\tfrac{{x - t}}
{2}} \right)\tfrac{{dx - dt}}
{2}}  + f_2 \left( {\tfrac{{x + t}}
{2}} \right) \hfill \\
\end{gathered}
\]

  \end{enumerate}
\end{sol}

  \item Seja $f:\R^n \to \R$ uma fun\c{c}\~ao diferenci\'avel e $\gamma :I \subset \R \to \R^{n + 1}$ tal que a imagem de $\gamma$ esteja contida no gr\'afico de $f$. Mostre que o vetor $\gamma '\left( {t_0 } \right)$ est\'a no espa\c{c}o tangente ao gr\'afico de $f$ no ponto $\gamma \left( {t_0 } \right)$ para todo $t_0 \in I$.

\begin{sol}
\[
\begin{gathered}
  \gamma \left( t \right) = \left( {x_1 \left( t \right),...,x_{n + 1} \left( t \right)} \right) \hfill \\
  p \in {\text{graf}}f \Rightarrow \left( {x_1 ,...,x_n ,f\left( {x_1 ,...,x_n } \right)} \right) \hfill \\
   \Rightarrow \gamma \left( t \right) = \left( {x_1 \left( t \right),...,x_n \left( t \right),f\left( {x_1 \left( t \right),...,x_n \left( t \right)} \right)} \right) \hfill \\
  \overrightarrow n  = \left( {\nabla f, - 1} \right){\text{ normal ao espa\c{c}o tangente}} \hfill \\
  \gamma '\left( t \right) = \left( {x_1 '\left( t \right),...,x_n '\left( t \right),\left\langle {\nabla f,\left( {x_1 '\left( t \right),...,x_n '\left( t \right)} \right)} \right\rangle } \right) \hfill \\
  \left\langle {\overrightarrow n ,\gamma '} \right\rangle  = \left\langle {\left( {\frac{{\partial f}}
{{\partial x_1 }},...,\frac{{\partial f}}
{{\partial x_n }}, - 1} \right),\left( {x_1 ',...,x_n ',\frac{{\partial f}}
{{\partial x_1 }}x_1 ' + ... + \frac{{\partial f}}
{{\partial x_n }}x_n '} \right)} \right\rangle  = 0 \hfill \\
   \Rightarrow \overrightarrow n  \bot \gamma ' \hfill \\
   \Rightarrow \gamma ' \in T_{\gamma \left( {t_0 } \right)} {\text{graf}}f \hfill \\
\end{gathered}
\]

\end{sol}

\textbf{Grupo 2: M\'aximos e m\'inimos.}

  \item Determine os pontos do hiperbol\'oide $x^2-y^2-z^2=1$ que est\~ao mais pr\'oximos da origem. Justifique corretamente por que os pontos encontrados s\~ao de fato pontos de m\'inimo.

\begin{sol}
\[
d\left( {x,y,z} \right) = \sqrt {x^2  + y^2  + z^2 }
\]

Seja $x^2 = 1+z^2+y^2$, ent\~ao

\[
d\left( {x,y,z} \right) = \sqrt {x^2  + y^2  + z^2 }  = \sqrt {1 + 2y^2  + 2z^2 }
\]

\'e m\'inimo se $y=z=0 \Rightarrow x^2=1 \Rightarrow x=\pm 1$

Portanto, $(1,0,0)$ e $(-1,0,0)$.

\newpage 

\uline{outro modo}: Multiplicador de Lagrange

Minimizar $d^2 \left( {x,y,z} \right) = x^2  + y^2  + z^2$ sujeito a $x^2  - y^2  - z^2  = 1$. Seja $g\left( {x,y,z} \right) = x^2  - y^2  - z^2  - 1$

\[
\begin{gathered}
  \left\{ \begin{gathered}
  \nabla d^2  = \lambda \nabla g \hfill \\
  g\left( {x,y,z} \right) = 0 \hfill \\
\end{gathered}  \right. \hfill \\
  \nabla d^2  = \left( {2x,2y,2z} \right) \hfill \\
  \nabla g = \left( {2x, - 2y, - 2z} \right) \hfill \\
  \left\{ \begin{gathered}
  2x = \lambda 2x \hfill \\
  2y =  - \lambda 2y \hfill \\
  2z =  - \lambda 2z \hfill \\
  x^2  - y^2  - z^2  = 1 \hfill \\
\end{gathered}  \right. \hfill \\
   \Rightarrow y = 0 = z \hfill \\
   \Rightarrow x =  \pm 1 \hfill \\
\end{gathered}
\]

\end{sol}

  \item Sejam $f:\R^2 \to \R$ dada por $f\left( {x,y} \right) = xy\left( {4 - x^2  - y^2 } \right)$ e considere a regi\~ao $Q = \left\{ {\left( {x,y} \right) \in \R^2 :0 \leqslant x \leqslant 2{\text{ e }}0 \leqslant y \leqslant 2} \right\}$.

  \begin{enumerate}[(a)]
    \item Determine os pontos cr\'iticos de $f$ que s\~ao interiores \`a regi\~ao $Q$ e classifique-os quanto a pontos de m\'aximos locais, m\'inimos locais ou de sela;
    \item Justifique a exist\^encia e determine os valores de m\'aximo e m\'inimo absolutos de $f$ sobre $Q$.
  \end{enumerate}

\newpage 

\begin{sol}
  \begin{enumerate}[(a)]
    \item Pontos cr\'iticos em $Q$:

    $\nabla f = (0,0)$

    Seja $f\left( {x,y} \right) = 4xy - yx^3  - xy^3$

\[
\begin{gathered}
  \nabla f = \left( {4y - 3x^2 y - y^3 ,4x - x^3  - 3y^2 x} \right) \hfill \\
  \left\{ \begin{gathered}
  4y - 3x^2 y - y^3  = 0 \hfill \\
  4x - x^3  - 3y^2 x = 0 \hfill \\
\end{gathered}  \right. \hfill \\
  \left\{ \begin{gathered}
  y\left( {4 - 3x^2  - y^2 } \right) = 0 \hfill \\
  x\left( {4 - 3y^2  - x^2 } \right) = 0 \hfill \\
\end{gathered}  \right. \hfill \\
   \Rightarrow \left( {x,y} \right) = \left\{ \begin{gathered}
  \cancel{ \left( {0,0} \right) } \hfill \\
  \left( {1,1} \right) \hfill \\
  \cancel{ \left( {1, - 1} \right) } \hfill \\
  \cancel{ \left( { - 1,1} \right) } \hfill \\
\end{gathered}  \right. \hfill \\
  f_{xx}  =  - 6xy = f_{yy}  \hfill \\
  f_{xx} \left( {1,1} \right) = f_{yy} \left( {1,1} \right) =  - 6 < 0 \hfill \\
   \Rightarrow \left( {1,1} \right){\text{ max local}} \hfill \\
  \boxed{f\left( {1,1} \right) = 2} \hfill \\
\end{gathered}
\]


    \item Temos

    \begin{itemize}
      \item $f(x,0)=0,x \in \left[ {0,2} \right]$
      \item $g(x)=f(x,2)=-2x^3,x \in \left[ {0,2} \right]$

\[
\begin{gathered}
  g'\left( x \right) =  - 6x^2  = 0 \Rightarrow x = 0 \hfill \\
  x = 2{\text{ min de }}g \hfill \\
  \boxed{f\left( {2,2} \right) =  - 16} \hfill \\
\end{gathered}
\]

      \item $f(0,y)=0,y \in \left[ {0,2} \right]$
      \item $f(2,y)=-2y^3$ tem min em $(2,2)$.
    \end{itemize}

Portanto, $(1,1)$ max global e $(2,2)$ min global.

  \end{enumerate}
\end{sol}

\newpage 

\textbf{Grupo 3: Teorema da fun\c{c}\~ao impl\'icita.}

  \item Considere a fun\c{c}\~ao $F:\R^3 \to \R$ dada por

\[
F\left( {x,y,z} \right) = z^3  + 3z + 2x^4  + y^2  - x^2  - 2y
\]


  \begin{enumerate}[(a)]
    \item Mostre que a equa\c{c}\~ao $F(x,y,z)=0$ define uma fun\c{c}\~ao $z=f(x,y)$ de classe $C^2$ em todo o plano;
    \item Determine os pontos cr\'iticos de $f$;
    \item Classifique esses pontos cr\'iticos quanto a m\'aximos locais, m\'inimos locais ou selas;
    \item Escreva o polin\^omio de Taylor de ordem 1 para $f$ em torno de $(1,1)$.
  \end{enumerate}

\begin{sol}
  \begin{enumerate}[(a)]
    \item Devemos provar que $z$ \'e fun\c{c}\~ao de $(x,y)$. Fixemos $(x,y) \in \R^2$. Ent\~ao, $2x^4  + y^2  - x^2  - 2y = k \in \R$.

\[
\begin{gathered}
  z^3  + 3z + 2x^4  + y^2  - x^2  - 2y = 0 \hfill \\
  z^3  + 3z + k = 0 \hfill \\
  g\left( z \right) = z^3  + 3z + k \hfill \\
  g'\left( z \right) = 3z^2  + 3 > 0 \hfill \\
\end{gathered}
\]

        Portanto, $g$ \'e crescente e sobrejetora.

        Ent\~ao, $\exists !z_0  \in \R$ tal que $g\left( {z_0 } \right) = 0$.

\[
\begin{gathered}
  F\left( {x_0 ,y_0 ,z_0 } \right) = 0 \hfill \\
  \frac{{\partial F}}
{{\partial z}}\left( {x_0 ,y_0 ,z_0 } \right) = \left. {3z^2  + 3} \right|_{z = z_0 }  = 3z_0^2  + 3 > 0 \hfill \\
\end{gathered}
\]

        Pelo Teorema da Fun\c{c}\~ao Impl\'icita \ref{sec17}, $\exists A \subset \R^2$ e $B \subset \R$ tal que $\left( {x_0 ,y_0 } \right) \in A$ e $z_0 \in B$ e $F\left( {x,y,z} \right) = 0,\forall \left( {x,y} \right) \in A$ e $z \in B$.

        $z=f(x,y)$ e \'e de mesma diferenciabilidade que $F$.

\[
\begin{gathered}
  f_x \left( {x,y} \right) =  - \frac{{F_x }}
{{F_z }} = \frac{{ - \left( {8x^3  - 2x} \right)}}
{{3z^2  + 3}} \hfill \\
  f_y \left( {x,y} \right) =  - \frac{{F_y }}
{{F_z }} = \frac{{ - \left( {2y - 2} \right)}}
{{3z^2  + 3}} \hfill \\
\end{gathered}
\]

        onde, $z=f(x,y)$.

    \item Pontos cr\'iticos
\[
\begin{gathered}
  \nabla f = \left( {0,0} \right) \hfill \\
  \left\{ \begin{gathered}
  8x^3  - 2x = 0 \hfill \\
  2y - 2 = 0 \hfill \\
\end{gathered}  \right. \hfill \\
   \Rightarrow x = 0, \pm \tfrac{1}
{2}{\text{ e }}y = 1 \hfill \\
\end{gathered}
\]

        Pontos cr\'iticos: $\left( {0,1} \right);\left( {{\raise0.5ex\hbox{$\scriptstyle 1$}\kern-0.1em/\kern-0.15em
\lower0.25ex\hbox{$\scriptstyle 2$}},1} \right);\left( { - {\raise0.5ex\hbox{$\scriptstyle 1$}\kern-0.1em/\kern-0.15em
\lower0.25ex\hbox{$\scriptstyle 2$}},1} \right)$

    \item Temos

\[
\begin{gathered}
  f_{xx}  = \frac{{ - \left( {24x^2  - 2} \right)\left( {3z^2  + 3} \right) + \left( {8x^3  - 2x} \right).6z.z_x }}
{{\left( {3z^2  + 3} \right)^2 }} \hfill \\
  f_{xy}  = \frac{{0\left( {3z^2  + 3} \right) + \left( {8x^3  - 2x} \right).6z.z_y }}
{{\left( {3z^2  + 3} \right)^2 }} = f_{yx} {\text{ }}\left( {{\text{T}}{\text{. Schwarz}}} \right) \hfill \\
  f_{yy}  = \frac{{ - 2\left( {3z^2  + 3} \right) + \left( {2y - 2} \right).6z.z_y }}
{{\left( {3z^2  + 3} \right)^2 }} \hfill \\
\end{gathered}
\]

$f(0,1)=z_0$ tal que $z_0^3  + 3z_0  - 1 = 0$

\[
\begin{gathered}
  f_{xx} \left( {0,1} \right) > 0 \hfill \\
  f_{xy} \left( {0,1} \right) = 0 \hfill \\
  f_{yy} \left( {0,1} \right) < 0 \hfill \\
  H_f  = \left( {\begin{array}{*{20}c}
   {f_{xx} } & {f_{yy} }  \\
   {f_{xy} } & {f_{yy} }  \\

 \end{array} } \right) \hfill \\
\end{gathered}
\]

$\det H_f \left( {0,1} \right) < 0 \Rightarrow \left( {0,1} \right)$ sela

\[
\begin{gathered}
  f_{xx} \left( {{\raise0.5ex\hbox{$\scriptstyle 1$}
\kern-0.1em/\kern-0.15em
\lower0.25ex\hbox{$\scriptstyle 2$}},1} \right) > 0 \hfill \\
  f_{xy} \left( {{\raise0.5ex\hbox{$\scriptstyle 1$}
\kern-0.1em/\kern-0.15em
\lower0.25ex\hbox{$\scriptstyle 2$}},1} \right) = 0 \hfill \\
  f_{yy} \left( {{\raise0.5ex\hbox{$\scriptstyle 1$}
\kern-0.1em/\kern-0.15em
\lower0.25ex\hbox{$\scriptstyle 2$}},1} \right) < 0 \hfill \\
\end{gathered}
\]

$\det H_f \left( {{\raise0.5ex\hbox{$\scriptstyle 1$}
\kern-0.1em/\kern-0.15em
\lower0.25ex\hbox{$\scriptstyle 2$}},1} \right) > 0 \Rightarrow \left( {{\raise0.5ex\hbox{$\scriptstyle 1$}
\kern-0.1em/\kern-0.15em
\lower0.25ex\hbox{$\scriptstyle 2$}},1} \right)$ max local

\[
\begin{gathered}
  f_{xx} \left( {{\raise0.5ex\hbox{$\scriptstyle { - 1}$}
\kern-0.1em/\kern-0.15em
\lower0.25ex\hbox{$\scriptstyle 2$}},1} \right) > 0 \hfill \\
  f_{xy} \left( {{\raise0.5ex\hbox{$\scriptstyle { - 1}$}
\kern-0.1em/\kern-0.15em
\lower0.25ex\hbox{$\scriptstyle 2$}},1} \right) = 0 \hfill \\
  f_{yy} \left( {{\raise0.5ex\hbox{$\scriptstyle { - 1}$}
\kern-0.1em/\kern-0.15em
\lower0.25ex\hbox{$\scriptstyle 2$}},1} \right) < 0 \hfill \\
\end{gathered}
\]

$\det H_f \left( {{\raise0.5ex\hbox{$\scriptstyle { - 1}$}
\kern-0.1em/\kern-0.15em
\lower0.25ex\hbox{$\scriptstyle 2$}},1} \right) > 0 \Rightarrow \left( {{\raise0.5ex\hbox{$\scriptstyle { - 1}$}
\kern-0.1em/\kern-0.15em
\lower0.25ex\hbox{$\scriptstyle 2$}},1} \right)$ max local

\newpage 

    \item Temos

\begin{eqnarray*}
  P_1 \left( {x,y} \right) &=& f\left( {1,1} \right) + f_x \left( {1,1} \right)\left( {x - 1} \right) + f_y \left( {1,1} \right)\left( {y - 1} \right) \hfill \\
   &=& 0 + \left( {\tfrac{{ - 6}}{3}} \right)\left( {x - 1} \right) + 0\left( {y - 1} \right) \hfill \\
  P_1 \left( {x,y} \right) &=&  - 2\left( {x - 1} \right) \hfill \\
\end{eqnarray*}

O plano tangente ao gr\'afico \'e

\[
\begin{gathered}
  z =  - 2x - 2 \hfill \\
  2x + z + 2 = 0 \hfill \\
\end{gathered}
\]


  \end{enumerate}
\end{sol}

\end{enumerate}

% Referencias bibliograficas

\begin{thebibliography}{99}

\bibitem[1]{gui1} GUIDORIZZI, Hamilton Luiz. {\sl Um Curso de C\'alculo.} Vol 1. 5\textordfeminine\ Ed. LTC, 2001.

\bibitem[2]{gui2} GUIDORIZZI, Hamilton Luiz. {\sl Um Curso de C\'alculo.} Vol 2. 5\textordfeminine\ Ed. LTC, 2001.

\bibitem[3]{gui3} GUIDORIZZI, Hamilton Luiz. {\sl Um Curso de C\'alculo.} Vol 3. 5\textordfeminine\ Ed. LTC, 2001.

\end{thebibliography}

\printindex
\end{document} 